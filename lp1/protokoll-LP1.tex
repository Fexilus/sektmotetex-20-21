\documentclass[hidelinks]{../sektionsmote} % Temporary fix
\usepackage{digsig}
\usepackage{csquotes}

\title{Protokoll fört vid sektionsmöte}
\shorttitle{Sektionsmötesprotokoll}
\motesdag{07}
\motesmanad{10}
\motesar{2020}
\motesnr{01}
\motestid{17.15}
\motesplats{Zoom}
\verksamhetsar{20/21}

\makeheader

\begin{document}
\maketitle

\section{Mötets öppnande}
Mötet öppnas \tid av Fysikteknologsektionens Talman Ruben Seyer.


\section{Mötets behörighet och beslutsförighet}
Talman meddelar att mötet är utlyst korrekt och i tid.
Han frågar mötet om det anses vara behörigt och beslutsförigt.
\begin{beslut}
    \item anse mötet behörigt och beslutsförigt enligt stadgarna.
\end{beslut}


\section{Val av justerare}
Efter en lång tystnad nomineras Styret-medlemmarna Joseph Löfving och David Winroth till justerare.
\begin{beslut}
    \item välja Joseph Löfving och David Winroth att justera protokollet. 
\end{beslut}
Alexandru Golic reserverar mot sektionsmötets beslut.
Som skäl anför han att sektionsmötet bör vara så oberoende från Styret som möjligt.


\section{Val av rösträknare}
Alexandru Golic och Albert Vesterlund nomineras till rösträknare.
\begin{beslut}
    \item välja Alexandru Golic och Albert Vesterlund till rösträknare.
\end{beslut}


\section{Fastställande av föredragningslistan}
Presidiet har tagit emot en förfrågan från Focumateriet om att lägga till punkten \enquote{Meddelande från Focumateriet}.
Talman yrkar därför på att föredragningslistan fastställs med tillägg av punkt §11 c Focumateriet.
\begin{beslut}
    \item fastställa föredragningslistan, med tillägg: 
    \begin{itemize}
        \item §11 c Focumateriet.
    \end{itemize}
\end{beslut}


\section{Antagande av mötesordning}
Tidigare Talman David Winroth har lagt förslag på mötesordning i enlighet med punkt 5.3.1 i reglementet.
Han säger att han har gjort justeringar för att förbättra den från förra årets mötesordning.
Bland annat har han ändrat hur adjungeringar utförs.
Han erbjuder sig svara på frågor kring ändringarna, men ingen sektionsmedlem har någon fråga.
\begin{beslut}
    \item anta den förslagna mötesordningen.
\end{beslut}

\section{Adjungeringar}
Inga adjungeringar föreligger.


\section{Föregående mötesprotokoll}
Tidigare Sekreterare Emma Lundqvist informerar om att det föregående protokollet justerades och anslogs i tid i enlighet med stadgan.
\begin{beslut}
    \item lägga föregående mötesprotokoll till handlingarna. 
\end{beslut}


\section{Uppföljning av beslut}
Inga beslut föreligger


\section{Fastställande av beslut}
Ett flertal beslut har tagits av Styret under senaste läsperioden.

\subsection{Avsägelser}
\begin{beslut}
    \item fastställa sektionsstyrelsens beslut att entlediga:
    \begin{itemize}
        \item Anton Brunström som ledamot i FIF.
        \item Eva Larsson som ledamot i valberedningen.
        \item Gustav Axelsson som kandidatansvarig i SNF.
        \item David Hambraeus som nätmakare i Spidera.
    \end{itemize}
\end{beslut}

\subsection{Fyllnadsval}
\begin{beslut}
    \item fastställa sektionsstyrelsens beslut att välja:
    \begin{itemize}
        \item Jonathan Bengtsson som revisor.
        \item Frida Krohn, Natalie Friedman, Eva Larsson, Tobias Hainer och Axel Johansson som leda-möter i valberedningen.
        \item Victor Salomonsson som sekreterare i SNF.
        \item Hannes Johansson som ledamot i FIF.
        \item Christian Josefson som kandidatansvarig i SNF.
    \end{itemize}
\end{beslut}

\subsection{Godkännande av ledamöter till programråd}
\begin{beslut}
    \item fastställa sektionsstyrelsens beslut att välja:
    \begin{itemize}
        \item Tobias Gabrielii och Emelie Sjögren till F:s programråd.
        \item Linnea Hallin och Albert Vesterlund till TM:s programråd.
    \end{itemize}
\end{beslut}


\section{Meddelanden}

\subsection{Sektionsstyrelsen}
Sektionsordförande Emelie Björkman.
Hon hoppas att Nollan har haft bra mottagning och känner sig välkomna på Chalmers och sektionen.
Hon informerar om att Styret har diskuterat fram vissa områden de tänker prioritera under verksamhetsåret.
Hon informerar dessutom om att alla sociala arrangemang på sektionen är inställda eftersom flera sektionsmedlemmar haft corona-symptom under veckan.
Hittills finns ett bekräftat coronafall.
Hon manar till fysisk distansering, men inte social.

Skyddsombud Emelie Sjögren informerar om att Övergången äntligen ska målas om.
Detta kommer ske under läsvecka 1, läsperiod 2.
Under tiden de målar om kommer lokalen inte brukas för studier.
Hon informerar även om de nästkommande skyddsronderna.
De kommer ske senare i år jämfört med tidigare år, och kommer därför ske under vecka 50. Under skyddsronder utevärderas bland annat brandsäkert, och desutom kommer Emelie kunna peka ut problem som studenterna upplever.
Ett exempel på problem hon kan ta upp är otillräcklig ventilation i Övergången.
För att bistå henne i detta kommer det finns ett formulär att fylla i.
Hon vill även påminna om att om man mår dåligt eller inte vet vart man ska vända sig kan man alltid prata med henne.
Hon sitter i Styretrummet om dagarna om man vill nå henne, men går även att nå genom sin \href{mailto:styret.samo@ftek.se}{mail}.
Till sist informerar hon om att köket ska renoveras.
Detta kommer antagligen ske inför Mottagningen 2021.

Vice sektionsordförande Sofia Reiner är nöjd med mottagningen.
Hon har utfört en utvärdering av den tillsammans med de sektionsaktiva.
Hon vill uppmuntra icke sektionsaktiva medlemmar att kontakta henne med åsikter för att vidga perspektiven i utvärderingen.

\subsection{Kårledningen}
Sektionens Kårledningskontakt detta läsår är Utbildningsenhetens ordförande Gabriel Aspegrén.
Honom kan man kontakta med frågor om både utbildningen och Kårledningen.
Han nås via \href{mailto:uo@chalmersstudentkar.se}{mail}, \href{tel:+46317723912}{telefon} och och sitt kontor i Kårlednings-korridoren i kårhuset.
Kårledningen vill tacka alla sektioner för Mottagningen.
De är glada att den blev så bra trotts omständigheterna i år.
Han informerar om att CHARM kommer genomföras digitalt i år, likt många sektioners arbetsmarknadsmässor.
Han informerar även om att kåren nya VD är Roger Nordman.
Roger har läst på Kemi och suttit i CCC, vilket Gabriel ser på som bra tecken.
Dessutom informerar han om att Fullmäktige arbetat på sin tillgänglighet för gemene teknolog; Fullmäktigesammanträden livestreamas sedan september, och dessutom kommer sammanfattningar av sammanträdandet publiceras.
Han informerar även om att arbetet med att skaffa en fullstor motionshall har påbörjats.
Till sist informerar han om att det fastslagits att idrottsengagemanget på Chalmers ska utvecklas.

\subsection{Focumateriet}
Automatpirat Olof Cronquist ska dela en PowerPoint, men stöter på tekniska problem.
Han vill informera om att Focumateriet bedriver oberoende journalistik under sektmötet, och att detta går att följa på nätet.
Han tillägger även att det vanligtvis är \enquote{Safe sex lv6}, men att det denna läsperiod borde lösa sig automatiskt eftersom teknologer ska hålla två meters avstånd.

\subsection{Talmanspresidiet}
Vice talman Martin Due håller en presentation om hur man deltar på sektionsmöten.
Dessutom informerar Talman om att Talmanspresidiet föreslår en matpaus om mötet ser ut att gå över 8:00, men att beslut om ajourneringar är upp till sektionsmötet.


\section{Verksamhets- och revisionsberättelser}
Talman yrkar på en tidsbegränsning på 5 minuter per verksamhetsberättelsepresentation.
Alex Golic undrar om diskussionen inkluderas, vilket Talman svarar inte inkluderas.
Tidigare Sektionsordförande Fredrik Meisingseth anser att 5 minuter är för kort, och vill ha längre tid.
Emelie Björkman undrar om Fredrik har någon vidare motivering, eftersom hon tycker att tidsbegränsningen låter rimlig.
Fredrik svarar att han inte vill tjafsa, och mötet går till omröstning.
\begin{beslut}
    \item sätta en tidsbegränsning på 5 minuter per verksamhetsberättelsepresentation.
\end{beslut}

\subsection{Studienämnden (SNF) 19/20}

\subsubsection{Verksamhetsberättelse}
Tidigare Ordförande i SNF Albert Johansson säger att SNF gjort det de ska göra.
Enda undantaget är protokollen som färdigjusterats sent för att hålla högre justeringsnivå.
SNF har under det gångna året kollat på om det finns möjlighet att ha inspelade videoföreläsningar utöver vanliga föreläsningar i sal, men kommit fram till att det inte är värt att arbeta vidare med i nuläget.
Tidigare SNF har dessutom hjälp nya SNF inleda sitt arbete.
Han tillägger att avvikande från verksamhetsplanen arrangerades inte ett Cocktailparty under läsperiod 4 på grund av pandemin.
\begin{beslut}
    \item med godkännande lägga berättelsen till handlingarna
\end{beslut}

\subsubsection{Revisionsberättelse}
Nils säger att revisorerna tillstyrker att bordlägga ansvarsfriheten eftersom några grejer är kvar i bokföringen.
David Winroth undrar om resten av nämnden ansvarsbefrias.
Mötet vill godkänna revisionsberättelsen

\subsubsection{Ansvarsfrihet}
David har lagt yrkande på att ansvarsbefria bara vice.
Fru Sekr. påpekar att praxis brukar vara att inte ansvarsbefria någon
Mei tycker att det är onödigt att ansvarsbefria separat
David svarar att detta förslag bygger på en tidigare diskussion, och tycker att folk inte borde fortsätta vara sektionsaktiva om det inte är nödvändigt.
Mötet beslutar att dela på besluten kring ansvarsfrihet.
Mötet beviljar vice-ordf ansvarsfrihet.
Mötet bordlägger frågan om ansvarsfrihet på ordf och kassör till nästa möte.

\subsection{Focumateriet 19/20}

\subsubsection{Verksamhetsberättelse}
Therese Gardell satt som Kapten.
De har haft sköna arrangemang (ta från deras bilaga).
Automaterna.
Planerat (men inte arrat) filpperresa
Letat efter en flaska Åke.
INte snålat med sprit
Alex Golic undrar om de hittade åke. Den finns i focumaten på focrummet.
Felix undrar varför inte nya presidiet har hattar, Therese tycker att de ju har gett hattar till presidiet.
Mötet godkänner och lägger berättelsen till handlingarna

\subsubsection{Revisionsberättelse}
Herr Talman läser upp revisorernas (grej) och de tillstyrker ansvarsfrihet
Mötet godkänner och lägger till handlingar.

\subsubsection{Ansvarsfrihet}
Mötet beviljar Focumateriet ansvarsfrihet

\subsection{F6 19/20}

\subsubsection{Verksamhetsberättelse}

Jonas Bohlin var sexmästare.
Det finns inte så mycket att säga. Det gick bra, och sedan blev aspningen skadad av corona och sabbade ekonomin.
Men de arrade gasquer et-raj och gröt och sopp-luncher.
Mötet godkänner berättelsen och lägger till handlingar

\subsubsection{Revisionsberättelse}
Nils Patriksson har inte så mycket att säga, det ser bra ut.
Mötet gokänner berättelsen och lägger den till handlingar.

\subsubsection{Ansvarsfrihet}
Revisorerna tillstyrker ansvarsfrihet.
Mötet beviljar F6 19/20 ansvarsfriheten

\subsection{Djungelpatrullen 19/20}

\subsubsection{Verksamhetsberättelse}
Axel Prebensen säger att Dp fixat FOcus, klätt om soffor, nollanstäd, pubrundor, men inte sista, inkomstbortfall.
Hofflor, aspning.
Mötet godkänner berättelsen och lägger till handlingar.

\subsubsection{Revisionsberättelse}
Revisorerna tillstyrker ansvarsfrihet.
Mötet godkänner och lägger till handlingar

\subsubsection{Ansvarsfrihet}
Mötet beviljar ansvarsfrihet till DP 19/20

\subsection{Sektionsstyrelsen 19/20}

\subsubsection{Verksamhetsberättelse}
Fredrik Meisingseth får ta över då Tarek inte är där.
De har styrt sektionen, det har gått bra.
Mötet godkänner och lägger till handlingar.

\subsubsection{Budgetutfall}
Benny är inte där, så Mei kör.
Det gick rätt bra ekonomiskt på sektionen.
De lyckades inte spendera pengar då nån grabb i Kina åt fladdermöss.
Godkänner och lägger till handlingar.

\subsubsection{Revisionsberättelse}
Nils säger att det ser bra ut, men att de behöver fixa några sista grejer.
Mötet godkänner och lägger till handlingar.

\subsubsection{Ansvarsfrihet}
Revisorer tillstyrker bordläggande till nästkommande möte.
David Winroth tycker återigen, med diskusion från ett år sedan som bakgrund, att ansvarsfriheten ska delas upp mellan ekonomiskt ansvariga och övriga förtroendeinvalda.
Mötet beslutar att dela på besluten.
Mötet bordlägger frågan om ansvarsfrihet för ekonomiskt ansvariga till nästkommande möte
Mötet beviljar övriga medlemmar ansvarsfrihet.


\section{Verksamhetsplaner och budget}

\subsection{Sektionsstyrelsen 20/21}

\subsubsection{Verksamhetsplan}
VPn är kul (tycker Bämelie)
Den skiljer sig mycket från den preleminära mycket
De har valt att bara ha med saker de brinner för och de kan genomföra
Jack har en fråga (lägger till handlingar) om TMs studiebevakning
Emelie håller med om att punkten är viktig. Den ströks dock pga. sin formulering. DEt stod att styret skulle kontrollera SNFs arbete, vilket de ogillade. DE vill gärna stötta SNF i frågan, men tycker att en sån punkt hör hemma hos dem.
I diskussion med SNF så har de sagt att de tror att de kommer arbeta med frågan pga. deras stora TM-andel.
Hon poängterar att deras punkt om jämstäldhet inkluderar denna typ av frågor.
Mötet fastställer verksamhetsplanen.

\subsubsection{Budget}
Finns ändringsyrkande från David redan innan.
David säger att budgeten ser annorlunda ut i år.
Han har lagt in föreningarnas budgettar i sektionens budget, vilket ger både större inkomstpunkter och utgiftspunkter.
Ändringsyrkandena är att SNF fick 500 kronor fel och lite smågrejer.
Mötet bifaller ändringsyrkandet.
Mötet fastställer budgeten med ändringsyrkandet.\footnote{Detta beslut rivs upp senare under mötet, se sidan \pageref{uppriven-budget}.}

\subsection{Verksamhetsplan Studienämnden (SNF) 20/21}
Oliver Thim (ordf.).
Fortsätta försöka få god svarsfrekvens på kursenkäter
Kursval ska bli tydligare
bevaka psykosoc
God kontakt med PA
Utveckla studiebevakning på master
I år tillkommer en punkt om att covid och EiB inte ska påverka studiekvalitén.
Mötet faställer verksamhetsplanen
    
\subsection{Verksamhetsplan Focumateriet 20/21}
Gustav Hallberg hörs lite dåligt på luren
Det låter som att han går runt, sedan blir han mute:ad
Vid ett andra försöker han presentera planen, men det enda ordet som går att ursklija är ''flipperturnering''.
Mötet fastställer verksamhetsplanen.

\subsection{Verksamhetsplan F6 20/21}
Ludvig gissar att alla på sektmötet läst planen
Extra fokus i år på att synas under rådande restriktioner.
Mötet besultar att fastställa F6 verksamhetsplan.

\subsection{Verksamhetsplan Djungelpatrullen 20/21}
Isabella Tepp säger att den är lik 19/20
Hon har fixat stavfel, och har coronaanpassat planen lite.
Mötet fastställer verksamhetplanen.


\section{Fyllnadsval}

\subsection{Valberedningen}

\subsubsection{Val av 1 ledamot}
Ingen söker.
Mötet beslutar om att vakantsätta posten.


\section{Personval}

\subsection{Studienämnden (SNF)}

\subsubsection{Val av årskursrepresentant år 1}
Samuel Martinsson söker.

Samuel söker går första året på TM.
Bemelie undar varför han är intreserad av SNFs arbete.
Som kursutvärderare fick han lite insikt, och tycker att det verkar kul att få bättre inblick och representera år 1.

Mötet väljer in Samuel som årskursrep.

\begin{ofraga} \label{uppriven-budget}
    David vill göra en ny ändring i budget, 200 kr mer till SNFs teambuilding.
    Mötet river upp gamla beslutet, fastställer nya budgeten.
\end{ofraga}

\subsection{Balnågonting}

\subsubsection{Val av 0--5 ledamöter}
Sökande:
Isac Johnsson
Albert Vesterlund

Isac är 19, F körkort.
Söker för att han var överaskad av balen. ASpningen sålde honom med fokus på projekt.

Albert är 20 TM2, har körkort
Puffade balen, tyckte det var kul.
Tycker sektionens middag var trevlig, tror den är rolig att arra.
Alexandru säger: Man ska bjuda in Bert INge, genom "Gamla tekniska fysiker", vad tycker Albert om dem?
Albert tycker att de är trevliga herrar, hoppas att de kan ge spons.

Mötet beslutar att välja in 5 ledamöter varav 3 vakanta
Mötet beslutar båda två valda

\section{Propositioner}

\section{Motioner}

\subsection{Alex kan inte baka till sektionsmöten}
Albert väljer att inte lyfta motionen.
"F6" lyfter motionen kollektivt
Styret yrkar på avslag eftersom det inte åligger bakis och det är logistiskt svårt.
Jonas Bohlin för F6 20/21s talan. De stöttar motionen starkt.
Mötet avslår motionen i sin helhet.
Ludwig Gustavsson reserverar sig

\section{Övriga frågor}

\section{Dumvästutdelning}
Joseph undrar varför David bara är med en gång, Talmanspresidiet svarar att de borde fixat det.
Alexandru vill ändra på bracketen, men Martin svarar att det inte skulle vara rättvist, för stor chans etc.
Mötet tycker om bracketen, så vi kör på den.

Emelie (F6) vill nominera Ruben Fri för att han slarvade bort dumvästen två gånger om, och sedan hittade den i sitt eget skåp
Nån nollk (skog) Krukan köpte för mycket ost, och sedan åkte han och köpte mer ost när den gamla fanns kvar
Fredrik nominerar Elsa (sushi): pufftack, bowling, elsa gör första kastet och träffar resultatskärmen ovanför bowlingbanan. Det var sur personal men gratis nachos.
Joseph har nominerat David Winroth eftersom han (läser innantill om nyckel, coronasymptom)
Albert Vesterlund har nominerat David Winroth (läser innantill om SNFs budget), Linnea (SNF-kass) förtydligar att budgeten var fine tills David skulle komma igång att ändra den.
Hugo Spencer nominerar Pontus Gustavsson eftersom han sa [citat] på nollupdragsredovisningen.
Alexandru nominerar Talmanspresidiet eftersom Ruben hade 3000 ping och gjorde en omröstning jobbigare via sektionsmöteshemsidan

Mötet beslutar att Sushi var dummast
Ruben presenterar förra dumheten med dugga
Mötet beslutar att Sushi var tillräckligt dum.


\section{Mötets avslutande}
Mötet avslutas 20.10 av Talman.

\clearpage
\section*{Signaturer}
En signatur på den här sidan avser hela sektionsmötesprotokollet \themote. Det sker elektroniskt via digitalt ID. Kontakta talmanspresidiets sekreterare på \href{mailto:talman.sekr@ftek.se}{\texttt{talman.sekr@ftek.se}} vid kontroll eller övriga frågor. 

\vspace{4cm}

\begin{minipage}{0.45\textwidth}
    \begin{center}
        \digsigfield{4.5cm}{1.5cm}{NN}
        \line(1,0){150}\\
        \footnotesize NN\\ %%Talman
        Talman\\[3cm]
        \digsigfield{4.5cm}{1.5cm}{NN}
        \line(1,0){150}\\
        \footnotesize NN\\ %%Justerare
        Justerare
        \end{center}
        \end{minipage}
        \begin{minipage}{0.45\textwidth}
        \begin{center}
        \digsigfield{4.5cm}{1.5cm}{NN} 
        \line(1,0){150}\\
        \footnotesize NN\\ %%Mötessekreterare
        Sekreterare\\[3cm]
        \digsigfield{4.5cm}{1.5cm}{NN}
        \line(1,0){150}\\
        \footnotesize NN\\ %%Justerare
        Justerare
    \end{center}
\end{minipage}

\clearpage
\begin{bilagor}
    \bilaga{Mötesordning}{motesordning.pdf}
    \bilaga{Beslut att fastställa}{beslut.pdf}
    
    \bilaga{Verksamhetsberättelse Studienämnden (SNF) 19/20}{snf/vb.pdf}
    \bilaga{Revisionsberättelse Studienämnden (SNF) 19/20}{snf/rb.pdf}
    \bilaga{Verksamhetsberättelse Focumateriet 19/20}{foc/vb.pdf}
    \bilaga{Revisionsberättelse Focumateriet 19/20}{foc/rb.pdf}
    \bilaga{Verksamhetsberättelse F6 19/20}{f6/vb.pdf}
    \bilaga{Revisionsberättelse F6 19/20}{f6/rb.pdf}
    \bilaga{Verksamhetsberättelse Djungelpatrullen 19/20}{dp/vb.pdf}
    \bilaga{Revisionsberättelse Djungelpatrullen 19/20}{dp/rb.pdf}
    \bilaga{Verksamhetsberättelse Sektionsstyrelsen 19/20}{styret/vb.pdf}
    \bilaga{Revisionsberättelse Sektionsstyrelsen 19/20}{styret/rb.pdf}
    \bilaga{Budgetutfall 19/20}{styret/budgetutfall.pdf}
    
    \bilaga{Verksamhetsplan Studienämnden (SNF) 20/21}{snf/vp.pdf}
    \bilaga{Verksamhetsplan F6 20/21}{f6/vp.pdf}
    \bilaga{Verksamhetsplan Djungelpatrullen 20/21}{dp/vp.pdf}
    \bilaga{Verksamhetsplan Focumateriet 20/21}{foc/vp.pdf}
    
    \bilaga{Verksamhetsplan Sektionsstyrelsen 20/21}{styret/vp.pdf}
    \bilaga{Budget 20/21}{styret/budget.pdf}
    
    \bilaga{Motion om ''Alex kan inte baka till sektionsmöten''}{motion/a.pdf}
    \bilaga{Motionssvar till ''Alex kan inte baka till sektionsmöten''}{motion/a_svar.pdf}
\end{bilagor}

\end{document}