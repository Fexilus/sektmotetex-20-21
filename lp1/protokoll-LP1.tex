\documentclass[hidelinks]{../sektionsmote} % Temporary fix
\usepackage{digsig}
\usepackage{csquotes}

\title{Protokoll fört vid sektionsmöte}
\shorttitle{Sektionsmötesprotokoll}
\motesdag{07}
\motesmanad{10}
\motesar{2020}
\motesnr{01}
\motestid{17.15}
\motesplats{Zoom}
\verksamhetsar{20/21}

\makeheader

\begin{document}
\maketitle

\section{Mötets öppnande}
Mötet öppnas \tid av Fysikteknologsektionens Talman Ruben Seyer.


\section{Mötets behörighet och beslutsförighet}
Talman Ruben Seyer meddelar att mötet är utlyst korrekt och i tid.
Han frågar mötet om det anses vara behörigt och beslutsförigt.
\begin{beslut}
    \item anse mötet behörigt och beslutsförigt enligt stadgarna.
\end{beslut}


\section{Val av justerare}
Efter en lång tystnad nomineras Styret-medlemmarna Joseph Löfving och David Winroth till justerare.
\begin{beslut}
    \item välja Joseph Löfving och David Winroth till justerare.
\end{beslut}
Alexandru Golic reserverar mot beslutet.
Som skäl anför han att sektionsmötet bör vara så oberoende från Styret som möjligt.


\section{Val av rösträknare}
Alexandru Golic och Albert Vesterlund nomineras till rösträknare.
\begin{beslut}
    \item välja Alexandru Golic och Albert Vesterlund till rösträknare.
\end{beslut}


\section{Fastställande av föredragningslistan}
Presidiet har tagit emot en förfrågan från Focumateriet om att lägga till punkten \enquote{Meddelande från Focumateriet}.
Talman Ruben Seyer yrkar därför på att föredragningslistan fastställs med tillägg av punkt §11 c Focumateriet.
\begin{beslut}
    \item fastställa föredragningslistan, med ändringar: 
    \begin{itemize}
        \item Lägga till punkt §11 c Focumateriet.
        \item Flytta punkt §11 c Talmanspresidiet till punkt §11 d Talmanspresidiet
    \end{itemize}
\end{beslut}


\section{Antagande av mötesordning}
Tidigare Talman David Winroth har lagt \hyperlink{bilagor/motesordning.pdf.1}{förslag på mötesordning} i enlighet med punkt 5.3.1 i reglementet.
Han säger att han har gjort justeringar för att förbättra den från förra årets mötesordning.
Bland annat har han ändrat hur adjungeringar utförs.
Han erbjuder sig svara på frågor kring ändringarna, men ingen sektionsmedlem har någon fråga.
\begin{beslut}
    \item anta den \hyperlink{bilagor/motesordning.pdf.1}{förslagna mötesordningen}.
\end{beslut}

\section{Adjungeringar}
Inga adjungeringar föreligger.


\section{Föregående mötesprotokoll}
Tidigare Sekreterare Emma Lundqvist informerar om att det föregående protokollet justerades och anslogs i tid i enlighet med stadgan.
\begin{beslut}
    \item lägga föregående mötesprotokoll till handlingarna. 
\end{beslut}


\section{Uppföljning av beslut}
Inga beslut föreligger.


\section{Fastställande av beslut}
Ett flertal \hyperlink{bilagor/beslut.pdf.1}{beslut} har tagits av Styret under senaste läsperioden.

\subsection{Avsägelser}
\begin{beslut}
    \item fastställa sektionsstyrelsens beslut att entlediga:
    \begin{itemize}
        \item Anton Brunström som ledamot i FIF.
        \item Eva Larsson som ledamot i Valberedningen.
        \item Gustav Axelsson som Kandidatansvarig i SNF.
        \item David Hambraeus som Nätmakare i Spidera.
    \end{itemize}
\end{beslut}

\subsection{Fyllnadsval}
\begin{beslut}
    \item fastställa sektionsstyrelsens beslut att välja:
    \begin{itemize}
        \item Jonathan Bengtsson som Revisor.
        \item Frida Krohn, Natalie Friedman, Eva Larsson, Tobias Hainer och Axel Johansson som ledamöter i Valberedningen.
        \item Victor Salomonsson som Sekreterare i SNF.
        \item Hannes Johansson som ledamot i FIF.
        \item Christian Josefson som Kandidatansvarig i SNF.
    \end{itemize}
\end{beslut}

\subsection{Godkännande av ledamöter till programråd}
\begin{beslut}
    \item fastställa sektionsstyrelsens beslut att välja:
    \begin{itemize}
        \item Tobias Gabrielii och Emelie Sjögren till F:s programråd.
        \item Linnea Hallin och Albert Vesterlund till TM:s programråd.
    \end{itemize}
\end{beslut}


\section{Meddelanden}

\subsection{Sektionsstyrelsen}
Sektionsordförande Emelie Björkman hoppas att Nollan har haft en bra Mottagning och känner sig välkomna på Chalmers och sektionen.
Hon informerar om att Styret har diskuterat fram vissa områden de tänker prioritera under verksamhetsåret.
Hon berättar dessutom om att alla sociala arrangemang på sektionen är inställda eftersom flera sektionsmedlemmar haft corona-symptom under veckan.
Hittills finns ett bekräftat coronafall.
Hon manar till fysisk distansering, men inte social.

Skyddsombud Emelie Sjögren informerar om att Övergången äntligen ska målas om.
Detta kommer ske under läsvecka 1, läsperiod 2.
Under tiden de målar om kommer lokalen inte brukas för studier.
Hon berättar även om de nästkommande skyddsronderna.
De kommer ske senare i år jämfört med tidigare år, och kommer därför ske under vecka 50. Under skyddsronder utvärderas bland annat brandsäkerhet, och dessutom kommer Emelie kunna peka ut problem som studenterna upplever.
Ett exempel på problem hon kan ta upp är otillräcklig ventilation i Övergången.
För att bistå henne i detta kommer det finns ett formulär att fylla i.
Hon vill även påminna om att om man mår dåligt eller inte vet vart man ska vända sig kan man alltid prata med henne.
Hon sitter i Styretrummet om dagarna om man vill nå henne, men går även att nå genom sin \href{mailto:styret.samo@ftek.se}{mail}.
Till sist informerar hon om att köket ska renoveras.
Detta kommer antagligen ske inför Mottagningen 2021.

Vice sektionsordförande Sofia Reiner är nöjd med Mottagningen.
Hon har utfört en utvärdering av den tillsammans med de sektionsaktiva.
Hon vill uppmuntra icke sektionsaktiva medlemmar att kontakta henne med åsikter för att vidga perspektiven i utvärderingen.

\subsection{Kårledningen}
Sektionens Kårledningskontakt detta läsår är Utbildningsenhetens ordförande Gabriel Aspegrén.
Honom kan man kontakta med frågor om både utbildningen och Kårledningen.
Han nås via \href{mailto:uo@chalmersstudentkar.se}{mail}, \href{tel:+46317723912}{telefon} och och sitt kontor i Kårlednings-korridoren i kårhuset.
Kårledningen vill tacka alla sektioner för Mottagningen.
De är glada att den blev så bra trots omständigheterna i år.
Han informerar om att CHARM kommer genomföras digitalt i år, likt många sektioners arbetsmarknadsmässor.
Han berättar även att kårens nya VD är Roger Nordman.
Roger har läst på Kemi och suttit i CCC, vilket Gabriel ser på som bra tecken.
Dessutom meddelar han att Fullmäktige har arbetat på sin tillgänglighet för gemene teknolog; Fullmäktigesammanträden livestreamas sedan september, och dessutom kommer sammanfattningar av sammanträdandena publiceras.
Han berättar även om att arbetet med att skaffa en fullstor motionshall har påbörjats.
Till sist informerar han om att det fastslagits att idrottsengagemanget på Chalmers ska utvecklas.

\subsection{Focumateriet}
Automatpirat Olof Cronquist ska dela en PowerPoint, men stöter på tekniska problem.
Han vill informera om att Focumateriet bedriver oberoende journalistik under sektmötet, och att detta går att följa på nätet.
Han tillägger även att det vanligtvis är \enquote{Safe Sex LV6}, men att det denna läsperiod borde lösa sig automatiskt eftersom teknologer ska hålla två meters avstånd.

\subsection{Talmanspresidiet}
Vice talman Martin Due håller en presentation om hur man deltar på sektionsmöten.
Dessutom informerar Talman Ruben Seyer om att Talmanspresidiet föreslår en matpaus om mötet ser ut att gå över 20.00, men att beslut om ajourneringar är upp till sektionsmötet.


\section{Verksamhets- och revisionsberättelser}
Talman Ruben Seyer yrkar på en tidsbegränsning på 5 minuter per verksamhetsberättelsepresentation.
Alex Golic undrar om diskussionen inkluderas, vilket Ruben svarar inte inkluderas.
Tidigare Sektionsordförande Fredrik Meisingseth anser att 5 minuter är för kort, och vill ha längre tid.
Emelie Björkman undrar om Fredrik har någon vidare motivering, eftersom hon tycker att tidsbegränsningen låter rimlig.
Fredrik svarar att han inte vill tjafsa, och mötet går till omröstning.
\begin{beslut}
    \item sätta en tidsbegränsning på 5 minuter per verksamhetsberättelsepresentation.
\end{beslut}

\subsection{Studienämnden (SNF) 19/20}

\subsubsection{Verksamhetsberättelse}
Tidigare Ordförande i SNF Albert Johansson säger att SNF har gjort det de ska göra.
Enda undantaget är protokollen som färdigjusterats sent för att hålla högre justeringsnivå.
SNF har under det gångna året kollat på om det finns möjlighet att ha inspelade videoföreläsningar utöver vanliga föreläsningar i sal, men kommit fram till att det inte är värt att arbeta vidare med i nuläget.
Tidigare SNF har dessutom hjälp nya SNF inleda sitt arbete.
Han tillägger att, avvikande från verksamhetsplanen, arrangerades inte ett Cocktailparty under läsperiod~4 på grund av pandemin.
\begin{beslut}
    \item med godkännande lägga \hyperlink{bilagor/snf/vb.pdf.1}{verksamhetsberättelsen} till handlingarna.
\end{beslut}

\subsubsection{Revisionsberättelse}
Revisor Nils Patriksson tillstyrker bordläggning av ansvarsfriheten eftersom bokföringen inte är färdiggjord.
David Winroth undrar om resten av nämnden ansvarsfrias.
Han får svaret att tillstyrkanden på bordläggning gäller de förtroendevalda.
\begin{beslut}
    \item med godkännande lägga \hyperlink{bilagor/snf/rb.pdf.1}{revisionsberättelsen} till handlingarna.
\end{beslut}

\subsubsection{Ansvarsfrihet}
David yrkar på att ansvarsfriheten ska delas upp mellan de ekonomiskt ansvariga (Ordförande och Kassör) och resterande (Vice ordförande).
Sekreterare Felix Augustsson påpekar att praxis är att inte ansvarsbefria någon vid inkompletta bokföringar.
Fredrik Meisingseth tycker att det är onödigt att ansvarsbefria Vice ordförande separat.
David svarar att detta förslag bygger på en diskussion efter röstning om ansvarsfrihet förra året.
Han tycker att folk inte borde fortsätta vara sektionsaktiva efter sitt sittandeår om det inte är nödvändigt, vilket man tekniskt sett gör om ansvarsfrihet inte beviljas.
\begin{beslut}
    \item dela på besluten kring ansvarsfrihet mellan ekonomiskt ansvariga och resterande.
\end{beslut}
\begin{beslut}
    \item bevilja tidigare Vice ordförande Albin Ahlbäck ansvarsfrihet för det gångna verksamhetsåret.
\end{beslut}
\begin{beslut}
    \item bordlägga frågan om ansvarsfrihet för tidigare Ordförande Albert Johansson och tidigare Kassör Anton Wikström till nästkommande sektionsmöte.
\end{beslut}

\subsection{Focumateriet 19/20}

\subsubsection{Verksamhetsberättelse}
Tidigare Kapten Therese Gardell säger att Focumateriet har anordnat sköna arrangemang så som Foc Kör, Flipperturnering, Flipperkväll och Sektionsaktivabastu under det gångna året.
De har även tagit hand om automaterna och planerat (men inte arrat) en flipperresa.
Till sist har de även letat efter en flaska Åke, och sett till att inte snåla med spriten under slutet av verksamhetsåret.
Alex Golic undrar om de hittade en flaska Åke.
Therese svarar att den finns i Focumaten i Focrummet.
Sekreterare Felix Augustsson undrar varför det nya presidiet inte har fått hattar.
Therese tycker att de har gett hattar till presidiet, men får invändningen att det bara var det tidigare presidiet som har fått hattar.
\begin{beslut}
    \item med godkännande lägga \hyperlink{bilagor/foc/vb.pdf.1}{verksamhetsberättelsen} till handlingarna.
\end{beslut}

\subsubsection{Revisionsberättelse}
Talman Ruben Seyer läser upp revisorernas \hyperlink{bilagor/foc/rb.pdf.1}{revisionsberättelse} och informerar om att de tillstyrker ansvarsfrihet.
\begin{beslut}
    \item med godkännande lägga \hyperlink{bilagor/foc/rb.pdf.1}{revisionsberättelsen} till handlingarna.
\end{beslut}

\subsubsection{Ansvarsfrihet}
\begin{beslut}
    \item bevilja tidigare Kapten Therese Gardell, tidigare Automatpirat Alex Bökmark samt tidigare Kistväktare Olof Cronquist ansvarsfrihet för det gångna verksamhetsåret.
\end{beslut}

\subsection{F6 19/20}

\subsubsection{Verksamhetsberättelse}
Tidigare Sexmästare Jonas Bohlin tycker inte att det finns så mycket att säga.
Han tyckte att det gick bra.
Sedan kom corona, vilket skadade aspningen och sabbade ekonomin.
Men det till trots arrade de Gasquer, ET-raj samt gröt- och sopp-luncher.
\begin{beslut}
    \item med godkännande lägga \hyperlink{bilagor/f6/vb.pdf.1}{verksamhetsberättelsen} till handlingarna.
\end{beslut}

\subsubsection{Revisionsberättelse}
Revisor Nils Patriksson har inte så mycket att säga eftersom han tycker att bokföringen ser bra ut.
\begin{beslut}
    \item med godkännande lägga \hyperlink{bilagor/f6/rb.pdf.1}{revisionsberättelsen} till handlingarna.
\end{beslut}

\subsubsection{Ansvarsfrihet}
Talman Ruben Seyer förtydligar att revisorerna tillstyrker ansvarsfrihet.
\begin{beslut}
    \item bevilja tidigare Sexmästare Jonas Bohlin, tidigare Sexreterare Frida Olsson samt tidigare Kassör Julia Hammare ansvarsfrihet för det gångna verksamhetsåret.
\end{beslut}

\subsection{Djungelpatrullen 19/20}

\subsubsection{Verksamhetsberättelse}
Tidigare Överste Axel Prebensen säger att DP har fixat Focus, klätt om soffor, hållit i Nollanstäd samt arrat Pubrundor alla läsperioder utom sista då de istället fick ett budgetbortfall.
Dessutom har de lagat Hofflor och arrat en aspning.
\begin{beslut}
    \item med godkännande lägga \hyperlink{bilagor/dp/vb.pdf.1}{verksamhetsberättelsen} till handlingarna.
\end{beslut}

\subsubsection{Revisionsberättelse}
Revisorerna tillstyrker ansvarsfrihet.
\begin{beslut}
    \item med godkännande lägga \hyperlink{bilagor/dp/rb.pdf.1}{revisionsberättelsen} till handlingarna.
\end{beslut}

\subsubsection{Ansvarsfrihet}
\begin{beslut}
    \item bevilja tidigare Överste Axel Prebensen, tidigare Rustmästare Frida Krohn samt tidigare Skattmästare Johan Bruhn ansvarsfrihet för det gångna verksamhetsåret.
\end{beslut}

\subsection{Sektionsstyrelsen 19/20}

\subsubsection{Verksamhetsberättelse}
Tidigare Sektionsordförande Fredrik Meisingseth får ta över eftersom tidigare Vice sektionsordförande Tarek Alhaskir inte är närvarande.
Han meddelar att Styret har styrt sektionen, vilket har gått bra.
\begin{beslut}
    \item med godkännande lägga \hyperlink{bilagor/styret/vb.pdf.1}{verksamhetsberättelsen} till handlingarna.
\end{beslut}

\subsubsection{Budgetutfall}
Tidigare Kassör Tobias Gabrielii är inte närvarande, så än en gång får Fredrik ta över.
Han berättar att det gick rätt bra ekonomiskt på sektionen.
Däremot lyckades de inte spendera pengar under året eftersom nån grabb i Kina åt fladdermussoppa.
\begin{beslut}
    \item med godkännande lägga \hyperlink{bilagor/styret/budgetutfall.pdf.1}{budgetutfallet} till handlingarna.
\end{beslut}

\subsubsection{Revisionsberättelse}
Revisor Nils Patriksson säger att bokföringen ser bra ut, men att det finns några sista detaljer som behöver fixas innan ansvarsfrihet beviljas.
\begin{beslut}
    \item med godkännande lägga \hyperlink{bilagor/styret/rb.pdf.1}{revisionsberättelsen} till handlingarna.
\end{beslut}

\subsubsection{Ansvarsfrihet}
Nils tillstyrker bordläggning av ansvarsfrihet till nästkommande sektionsmöte.
David Winroth tycker återigen, med diskussionen från ett år sedan som bakgrund, att ansvarsfriheten ska delas upp mellan ekonomiskt ansvariga och övriga förtroendevalda.
\begin{beslut}
    \item dela på besluten kring ansvarsfrihet mellan ekonomiskt ansvariga och resterande.
\end{beslut}
\begin{beslut}
    \item bordlägga frågan om ansvarsfrihet för tidigare Sektionsordförande Fredrik Meisingseth och tidigare Kassör Tobias Gabrielii till nästkommande sektionsmöte.
\end{beslut}
\begin{beslut}
    \item bevilja tidigare Vice sektionsordförande Tarek Alhaskir, tidigare Sekreterare Hannes Bergström, tidigare Skyddsombud Sara Nordin Hällgren samt tidigare Informationsansvarig Alexandru Golic ansvarsfrihet för det gångna verksamhetsåret.
\end{beslut}

\section{Verksamhetsplaner och budget}

\subsection{Sektionsstyrelsen 20/21}

\subsubsection{Verksamhetsplan}
Sektionsordförande Emelie Björkman tycker att verksamhetsplaner är kul.
\hyperlink{bilagor/styret/vp.pdf.1}{Årets verksamhetsplan} skiljer sig mycket från den preliminära verksamhetsplanen.
Styret har valt att bara ta med saker de brinner för och tror de kan genomföra under året.
Sektionsordförande 18/19 Jack Vahnberg har skickat in en fråga skriftligt.
Den lyder:
\enquote{%
    I preliminära verksamhetsplanen för styret 20/21 ingick en punkt med namnet \enquote{utvärdera studiebevakning av Teknisk Matematik}, som saknas i den slutgiltiga VP:n.
    Personlig anser jag att TM's studiebevakning kan behöva en utvärdering och revidering, och saknar punkten.
    Hur kommer det sig att ni valde att utelämna den i den slutgiltiga VP:n?%
}
Emelie håller med om att punkten är viktig.
Den ströks dock på grund av sin formulering.
Formuleringen antydde att Styret skulle kontrollera SNF:s arbete, vilket de ogillade.
Styret vill gärna stötta SNF i frågan, men tycker att en sådan punkt hör hemma i SNF:s verksamhetsplan.
I diskussion med SNF har det även framkommit att SNF tror att de kommer ha ett sådant fokus naturligt i år, eftersom de ha en hög andel TM-studenter i nämnden.
Till sist poängterar Emelie att Styrets punkt om jämställdhet inkluderar även denna typ av frågor.
\begin{beslut}
    \item fastställa \hyperlink{bilagor/styret/vp.pdf.1}{verksamhetsplanen} för Sektionsstyrelsen 20/21.
\end{beslut}

\subsubsection{Budget}
Kassör David Winroth lägger ett \hyperlink{bilagor/styret/ny-budget.pdf.1}{ändringsyrkande på budgeten}.
Han informerar om att budgeten ser annorlunda ut i år eftersom han har lagt in föreningarnas budgetar i sektionens budget.
Detta ger både större inkomstpunkter och utgiftspunkter.
Ändringsyrkandet består till största del av en korrektion på 500kr mindre till SNF.
\begin{beslut}
    \item bifalla ändringsyrkandet.
    \item fastställa \hyperlink{bilagor/styret/ny-budget.pdf.1}{budgeten med ändringsyrkandet}.\footnote{Detta beslut rivs upp senare under mötet, se sidan \pageref{uppriven-budget}.}\label{budget-orginalbeslut}
\end{beslut}

\subsection{Verksamhetsplan Studienämnden (SNF) 20/21}
Ordförande i SNF Oliver Thim säger att årets SNF kommer fortsätta försöka få god svarsfrekvens på kursenkäter.
Dessutom ska kursvalen bli tydligare.
Årets SNF vill även fortsätta bevaka det psykosociala läget på sektionen.
Under året ska det hållas god kontakt med programansvariga, och SNF vill arbeta för att utveckla studiebevakning på master-nivå.
Till sist informerar han om att det i år tillkommer en punkt om att pandemin och \enquote{Ekonomi i balans} inte ska påverka studiekvalitén.
\begin{beslut}
    \item fastställa \hyperlink{bilagor/snf/vp.pdf.1}{verksamhetsplanen} för SNF 20/21.
\end{beslut}
    
\subsection{Verksamhetsplan Focumateriet 20/21}
Kapten Gustav Hallberg försöker presentera Focumateriets verksamhetsplan, men hörs lite dåligt på luren.
Det låter som att han går runt, sedan stängs hans mikrofon av.
Han gör ett andra försök att presentera planen, men det enda ordet som går att urskilja är \enquote{flipperturnering}.
\begin{beslut}
    \item fastställa \hyperlink{bilagor/foc/vp.pdf.1}{verksamhetsplanen} för Focumateriet 20/21.
\end{beslut}

\subsection{Verksamhetsplan F6 20/21}
Sexmästare Ludwig Gustavsson gissar att alla på sektionsmötet läst deras verksamhetsplan.
Det till trots vill han förtydliga att det i år är ett extra fokus på att synas på sektionen, på grund av rådande restriktioner.
\begin{beslut}
    \item fastställa \hyperlink{bilagor/f6/vp.pdf.1}{verksamhetsplanen} för F6 20/21.
\end{beslut}

\subsection{Verksamhetsplan Djungelpatrullen 20/21}
Överste Isabella Tepp säger att årets verksamhetsplan är lik föregående års.
Hon har rättat stavfel och corona-anpassat planen lite.
\begin{beslut}
    \item fastställa \hyperlink{bilagor/pd/vp.pdf.1}{verksamhetsplanen} för Djungelpatrullen 20/21.
\end{beslut}


\section{Fyllnadsval}

\subsection{Valberedningen}

\subsubsection{Val av 1 ledamot}
Ingen söker.
\begin{beslut}
    \item vakantsätta 1 ledamot i valberedningen.
\end{beslut}


\section{Personval}

\subsection{Studienämnden (SNF)}

\subsubsection{Val av årskursrepresentant år 1}
Det finns en sökande:
\begin{itemize}
    \item Samuel Martinsson.
\end{itemize}

Samuel går första året på TM.
Emelie Björkman undrar varför han är intresserad av SNF:s arbete.
Samuel svarar att han som kursutvärderare upplevde att han fick lite inblick i skolans och sektionens verksamhet, och tycker att det verkar kul att få bättre inblick och representera årskurs 1.
\begin{beslut}
    \item välja in Samuel Martinsson som årskursrepresentant år 1 i SNF.
\end{beslut}

\begin{ofraga} \label{uppriven-budget}
    Kassör David Winroth vill göra en ny ändring i budget.
    Han yrkar på att ge SNF 200kr mer till teambuilding.
    \begin{beslut}
        \item riva upp \hyperref[budget-orginalbeslut]{beslutet om att fastställa budgeten med ändringsyrkande}.
        \item fastställa \hyperlink{bilagor/styret/ny-budget2.pdf.1}{budgeten med det tidigare ändringsyrkandet, samt vidare ändringar}:
        \begin{itemize}
            \item Ge SNF 2700kr i teambuilding-budget.
            \item Justera de budgetposter som påverkas till följd av denna ändring.
        \end{itemize}
    \end{beslut}
\end{ofraga}

\subsection{Balnågonting}

\subsubsection{Val av 0--5 ledamöter}
Det finns två sökande:
\begin{itemize}
    \item Isac Johnsson
    \item Albert Vesterlund.
\end{itemize}

Isac är 19, går första året på F och har körkort.
Han söker eftersom han blev överraskad av balen.
Balnågontings aspning sålde in sig hos honom, eftersom det då framkom att det är ett stort fokus på projekt.

Albert är 20, går andra året på TM och har körkort
Han puffade balen och tyckte det var kul.
Han tycker även att Sektionens middag var trevlig och tror den är rolig att arra.
Alexandru Golic nämner att man som Balnågonting ska bjuda in Bert-Inge Hogsved genom \enquote{Gamla tekniska fysiker}.
Han undrar vad Albert tycker om dem.
Albert tycker att de är trevliga herrar och hoppas att de kan sponsra middagen.
\begin{beslut}
    \item välja in 5 ledamöter i Balnågonting, varav 3 vakanta.
    \item välja in Isac Johnsson och Albert Vesterlund som ledamöter i Balnågonting.
\end{beslut}

\section{Propositioner}
Inga propositioner föreligger.

\section{Motioner}

\subsection{Alex kan inte baka till sektionsmöten}
Motionären Albert Johansson väljer att inte lyfta \hyperlink{bilagor/motion/a.pdf.1}{motionen}.
\enquote{F6} väljer däremot att i hans ställe lyfta motionen kollektivt.
Styret yrkar i \hyperlink{bilagor/motion/a-svar.pdf.1}{sitt motionssvar} på avslag eftersom det inte åligger Bakisclubben att baka till denna läsperiod, samt att bullutkörningen är logistiskt svår.
Jonas Bohlin säger sig föra F6 20/21:s talan.
Han säger att de stöttar motionen starkt.
\begin{beslut}
    \item avslå \hyperlink{bilagor/motion/a.pdf.1}{motionen} i sin helhet.
\end{beslut}
Ludwig Gustavsson reserverar sig mot sektionsmötets beslut, utan angivet skäl.

\section{Övriga frågor}
Inga övriga frågor föreligger.

\section{Dumvästutdelning}
\begin{ofraga}
    Talman Ruben Seyer yrkar på 10 minuters ajournering.
    \begin{beslut}
        \item inte ajournera mötet.
    \end{beslut}
    Talman Seyer väljer då att ajournera mötet i 5 minuter, i enlighet med §17 i den antagna mötesordningen.
\end{ofraga}

\noindent Nomineringarna är som följer:
\begin{itemize}
\item \textbf{Ruben Frilund} nomineras av Emelie Lemann.\\
Ruben är nuvarande Dumvästinnehavare, och har slarvat bort Dumvästen två gånger om sedan förra mötet.
Lyckligtvis är Dumvästen är nu tillbaka hos honom efter att han hittat den i sitt egna skåp.

\item \textbf{Mathias ''Crux'' Arvidsson} nomineras av Fredrik Skoglund.\\
Mathias köpte till Mottagningens först grilltillfälle 400 individuellt inplastade skivor Cheese-food (innehållande 25\% ost).
Vid nästkommande tillfälle köper Mathias, även efter de grillandes klagomål, 400 individuellt inplastade skivor Cheese-food.
Detta trotts att 200 skivor kvarstod från föregående grillning.

\item \textbf{Elsa ''\$ushi'' Danielsson} nomineras av Fredrik Skoglund.\\
Elsa arrangerade tillsammans med resterande FnollK och F6 ett pufftack i form av bowling.
Elsa, som saknar tidigare erfarenhet av bowling, bestämmer sig för att göra kvällens första kast.
Otursamt nog missar hon banan, och träffar istället TV-skärmen 2 meter ovanför banan.

\item \textbf{David Winroth} nomineras av Joseph Löfving.\\
David tog i sin sjuka Sekreterare Josephs ställe emot en av sektionens viktigaste nycklar när föregående landsfader skulle flytta till Österrike.
Efter att ha mottagit nyckeln funderar David på om den får plats i myntinkastet till nyckelkassan.
För att reda ut det trycker han prompt ner nyckeln i myntinkastet, men upptäcker snabbt att han inte får ut den eftersom nyckeln har fastnat.

\item \textbf{David Winroth} nomineras av Albert Vesterlund.\\
David lyckades inte skicka in en korrekt budget, och behövde därför skicka in ett ändringsyrkande.
Men även ändringen var felaktig, vilket ledde till ett upprivet beslut och ytterligare ett ändringsyrkande.

\item \textbf{Pontus Gustavsson d.ä} nomineras av Hugo Spencer.\\
Pontus sa med unmute:ad mikrofon på nolluppdragsredovisningen \enquote{Man får hålla tungan rätt i mun, tills Mottagningen är slut, då kan man hålla den i någon annans mun}.

\item \textbf{Talmanspresidiet} nomineras av Alexandru Golic.\\
Talmanspresidiet hade 3000 ping, och behövde därför gå till sluten votering trots att de klubbat beslutet från en omröstning.
\end{itemize}

\begin{beslut}
    \item Elsa ''\$ushi'' Danielsson var dummast.
\end{beslut}
Efter noggrann jämförelse med föregående dumvästinnehavare Ruben Frilund går mötet till beslut.
\begin{beslut}
    \item ge Elsa ''\$ushi'' Danielsson Dumvästen.
\end{beslut}


\section{Mötets avslutande}
Mötet avslutas 20.10 av Talman.

\clearpage
\section*{Signaturer}
\label{sec:sig}
\addcontentsline{toc}{section}{\nameref{sec:sig}}
En signatur på den här sidan avser hela sektionsmötesprotokollet \themote. Det sker elektroniskt via digitalt ID. Kontakta talmanspresidiets sekreterare på \href{mailto:talman.sekr@ftek.se}{\texttt{talman.sekr@ftek.se}} vid kontroll eller övriga frågor. 

\vspace{4cm}
\begin{center}
    \makebox[12cm][c]{
        \parbox{12cm}{
            \signatur{Ruben Seyer}{Talman} \hfill
            \signatur{Felix Augustsson}{Sekreterare}
            \newline
            \signatur{Joseph Löfving}{Justerare} \hfill
            \signatur{David Winroth}{Justerare}
            \vspace{1.5em}
        }
    }
\end{center}

\clearpage
\begin{bilagor}
    \bilaga{Mötesordning}{motesordning.pdf}
    \bilaga{Beslut att fastställa}{beslut.pdf}
    
    \bilaga{Verksamhetsberättelse Studienämnden (SNF) 19/20}{snf/vb.pdf}
    \bilaga{Revisionsberättelse Studienämnden (SNF) 19/20}{snf/rb.pdf}
    \bilaga{Verksamhetsberättelse Focumateriet 19/20}{foc/vb.pdf}
    \bilaga{Revisionsberättelse Focumateriet 19/20}{foc/rb.pdf}
    \bilaga{Verksamhetsberättelse F6 19/20}{f6/vb.pdf}
    \bilaga{Revisionsberättelse F6 19/20}{f6/rb.pdf}
    \bilaga{Verksamhetsberättelse Djungelpatrullen 19/20}{dp/vb.pdf}
    \bilaga{Revisionsberättelse Djungelpatrullen 19/20}{dp/rb.pdf}
    \bilaga{Verksamhetsberättelse Sektionsstyrelsen 19/20}{styret/vb.pdf}
    \bilaga{Budgetutfall 19/20}{styret/budgetutfall.pdf}
    \bilaga{Revisionsberättelse Sektionsstyrelsen 19/20}{styret/rb.pdf}
    
    \bilaga{Verksamhetsplan Sektionsstyrelsen 20/21}{styret/vp.pdf}
    \bilaga{Budget 20/21}{styret/budget.pdf}
    \bilaga{Budget 20/21 (Ändringsyrkande 1)}{styret/ny-budget.pdf}
    \bilaga{Budget 20/21 (Ändringsyrkande 2)}{styret/ny-budget2.pdf}
    \bilaga{Verksamhetsplan Studienämnden (SNF) 20/21}{snf/vp.pdf}
    \bilaga{Verksamhetsplan Focumateriet 20/21}{foc/vp.pdf}
    \bilaga{Verksamhetsplan F6 20/21}{f6/vp.pdf}
    \bilaga{Verksamhetsplan Djungelpatrullen 20/21}{dp/vp.pdf}
    
    \bilaga{Motion om ''Alex kan inte baka till sektionsmöten''}{motion/a.pdf}
    \bilaga{Motionssvar till ''Alex kan inte baka till sektionsmöten''}{motion/a-svar.pdf}
\end{bilagor}

\end{document}