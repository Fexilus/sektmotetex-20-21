\documentclass[hidelinks]{sektionsmote}
\usepackage{digsig}
\usepackage{csquotes}
\usepackage{censor}
\usepackage{soul}

\title{Protokoll fört vid sektionsmöte}
\shorttitle{Sektionsmötesprotokoll}
\motesdag{25}
\motesmanad{2}
\motesar{2021}
\motesnr{03}
\motestid{17.23}
\motesplats{Zoom}
\verksamhetsar{20/21}

\makeheader

\begin{document}
\maketitle

\section{Mötets öppnande}
Mötet öppnas \tid av Fysikteknologsektionens Talman Ruben Seyer.


\section{Mötets behörighet och beslutsförighet}
Talman Ruben Seyer meddelar att mötet är utlyst korrekt och i tid.
Han frågar mötet om det anses vara behörigt och beslutsförigt.
\begin{beslut}
    \item anse mötet behörigt och beslutsförigt enligt stadgarna.
\end{beslut}


\section{Val av justerare}
Tarek Alhaskir och Mikael Eriksson nomineras till justerare.
\begin{beslut}
  \item välja Tarek Alhaskir och Mikael Eriksson till justerare.
\end{beslut}


\section{Val av rösträknare}
Tarek Alhaskir och Mijo Thoresson nomineras till rösträknare.
\begin{beslut}
  \item välja Tarek Alhaskir och Mijo Thoresson till rösträknare.
\end{beslut}


\section{Fastställande av föredragningslista}
Adam Johansson yrkar på att flytta motion §15 c Vice i FIF till mellan punkt §12 Verksamhetsplaner och punkt §13 Personval.

Tarek Alhaskir yrkar på att lägga till ett meddelande §10 d från sitt nya valförbund.

Vice sektionsordförande Sofia Reiner yrkar på att flytta proposition §14 e Budgetändring för köksrenovering och riktade satsningar till mellan §12 Verksamhetsplaner och punkt §13 Personval.
\begin{beslut}
  \item fastställa föredragningslistan med följande ändringar:
  \begin{itemize}
    \item Lägga till punkt §10 d Nytt valförbund
    \item Skapa en ny punkt §13 tidigarelagda ärenden och ändra numreringen på efterkommande punkter.
    \item Flytta proposition §14 e Budgetändring för köksrenovering och riktade satsningar till punkt §13 a.
    \item Flytta motion §15 c Vice i FIF till punkt §13 b.
  \end{itemize}
\end{beslut} % TODO: Länkar (till sections?)


\section{Adjungeringar}
Inga adjungeringar föreligger.


\section{Föregående mötesprotokoll}
Sekreterare Felix Augustsson informerar om att det föregående protokollet justerades och anslogs i tid i enlighet med stadgan.
\begin{beslut}
    \item lägga föregående mötesprotokoll till handlingarna. 
\end{beslut}


\section{Uppföljning av beslut}
Inga beslut föreligger.


\section{Fastställande av beslut}
\subsection{Fyllnadsval}
\begin{beslut}
  \item fastställa sektionsstyrelsens beslut att välja:
  \begin{itemize}
      \item Viktor Lilja som ledamot i FARM.
  \end{itemize}
\end{beslut}


\section{Meddelanden}
\subsection{Sektionsstyrelsen}
Skyddsombud Emelie Sjögren meddelar att många svarade på studentbarometern.
Sektionen hade trots det 3:e högst svarsfrekvens.
Resultatet ska användas under ett möte med högskolan som kommer att handla om hur teknologer både ska må och trivas bättre.

Sektionskassör David Winroth delger att sektionen trotts pandemin har dragit in mer pengar än vad som spenderas.
Under verksamhetsårets andra kvartal gick sektionen 10 tusen kronor plus.
Han tillägger att det därför är bra om sektionsmötet röstar igenom propositionen gällande budgetändringar.

Vice sektionsordförande Sofia Reiner informerar om att Focus kök ska renoveras givet att tidigare nämnd proposition röstas igenom.
Enligt plan kommer det nya köket då vara färdigt innan Mottagningen 2021.

\subsection{Kårledningen}
Sektionens kårledningskontakt Gabriel Aspegrén säger att ansökan om till nästa års kårledning just har stängt.
Han meddelar även att kårledningen öppnat upp för att arr ska kunna arrangeras på JA Pripps enligt samma regler som gäller för JA:s dagliga verksamhet.
För mer information om det bör man kontakta vSO. % TODO: Länk + namn
Han informerar även om att fullmäktige sammanträder på onsdag.
Under mötet kommer bland annat dubbelt sektionsmedlemskap diskuteras.
Vidare kan teknologer kandidera för att själva sitta i fullmäktige, vilket kårledningen eller sektionens styrelse kan informera mer om.
I och med att en ny kårledning ska väljas in jobbar sittande kårledningen på en ny preliminär verksamhetsplan, vilket man kan kontakta dem kring.
Till sist påminner han om att man kan skicka in ansökningar till Mastercard-stipendiet. % TODO: Länk + kanske rätt namn

\subsection{Focumateriet}
Automatpirat Olof Cronquist påminner om att det är \enquote{Safe Sex LV6}, vilket avståndshållande borde lösa.
Focumateriet kommer dessutom som vanligt bedriva oberoende journalistik på \newline\href{https://focumateriet.wordpress.com}{\texttt{focumateriet.wordpress.com}}.


\subsection{Nytt valförbund}
Tarek Alhaskir informerar om att han är med i en grupp F-teknologer som har startat ett valförbund till Fullmäktige.
Man kan höra av sig till honom om man har intresse av att vara med i valförbundet, eller om man har åsikter om vilka frågor de borde driva.


\section{Verksamhets- och revisionsberättelser}
\subsection{FARM 2020}
\subsubsection{Verksamhetsberättelse}
Tidigare vice ordförande Valter Schütz berättar att pandemin har begränsat antalet fysiska arr, men att FARM istället haft digitala lunchföreläsningar och FARM-weeks.
FARM är även kontaktat företag, och har nu över 170 företagskontakter för framtida mässor.
I övrigt har de även publicerat alumniintervjuer, införskaffat sponsorer till Mottagningen och arrangerat under Mottagningen.
Bland annat arrangerade de en uppskattad räkfrossa som Valter hoppas arrangeras även kommande år.
Till sist påminner han om att ett av målen för FARM under 2020 var att öka lönsamheten, vilket de drivit genom prishöjningar.
Prishöjningarna innebär att FARM 2020 även utan FARM-mässan drog in lika mycket som FARM 2019.

\begin{beslut}
  \item med godkännande lägga verksamhetsberättelsen till handlingarna.
\end{beslut}


\subsubsection{Revisionsberättelse}
Lekmannarevisor Johan Bruhn påpekar att det är vissa fel i räkenskaperna som härstammar från verksamhetsåret 18/19.
Detta problem är dock under kontroll, och revisorerna tillstyrker därmed ansvarsfrihet.

\begin{beslut}
  \item med godkännande lägga revisionsberättelsen till handlingarna.% TODO: link
\end{beslut}


\subsubsection{Ansvarsfrihet}
\begin{beslut}
  \item bevilja tidigare ordförande Richard Svensson, tidigare vice ordförande Valter Schütz samt tidigare kassör Filip Rydin ansvarsfrihet för det gångna verksamhetsåret.
\end{beslut}


\subsection{FnollK 2020}
\subsubsection{Verksamhetsberättelse}
Tidigare ordförande Carl Strandby berättar att några mål från verksamhetsplanen för FnollK 2020 inte är uppfyllda på grund av pandemin.
Bland annat gick det inte att uppfylla målen om att Nollan skulle få träffa andra phaddergrupper och sektionens kommittéer, och minskandet av FnollK:s arbetsbelastning.
Målet om att alla Nollan ska klara sin första tenta gick även sämre än vanligt.
Carl påpekar att det 2020 blev fyra gånger så många arrangemang, eftersom varje storphaddergrupp fick var sitt av varje arrangemang.
Under ett år med extraordinära omständigheter som mångdubblade arbetsbelastningen för sektionsaktiva känner FnollK 2020 därför sig stå i tacksamhetsskuld till sektionen.

\begin{beslut}
  \item med godkännande lägga verksamhetsberättelsen till handlingarna.
\end{beslut}


\subsubsection{Revisionsberättelse}
Lekmannarevisor Johan Bruhn säger att bokföringen är korrekt, och att inget utanför ramarna för vad FnollK bör göra har gjorts.
Även FnollK har vissa räkenskapsfel som härstammar från 2018 eller 2019, men likt för FARM tillstyrker revisorerna även här ansvarsfrihet.

\begin{beslut}
  \item med godkännande lägga revisionsberättelsen till handlingarna.% TODO: link
\end{beslut}


\subsubsection{Ansvarsfrihet}
\begin{beslut}
  \item bevilja tidigare ordförande Carl Strandby, tidigare vice ordförande Mathias Arvidsson samt tidigare kassör Elsa Danielsson ansvarsfrihet för det gångna verksamhetsåret.
\end{beslut}


\section{Verksamhetsplaner}
\subsection{Verksamhetsplan FARM 2021}
Ordförande Eric Carlsson berättar att årets plan är lik förra årets.
FARM kommer jobba med att arrangera på distans med företag.
De tänker även genomföra en mässa på distans.
FARM kommer att fortsätta sitt arbete med företagskontakter, som FARM 2020 gjorde mycket lättare.
Till sist tänker FARM 2021 stärka sin bild på sektionen, bland annat genom att ha en mer ordentlig aspning.

\begin{beslut}
  \item fastställa verksamhetsplanen för FARM 2021.% TODO: link
\end{beslut}


\subsection{Verksamhetsplan FnollK 2021}
Ordförande Emma Ödman säger att FnollK 2021 vill att Nollan ska klara sin första tenta och hitta en kompis.
De ska även lära känna Chalmers och de sektionsaktiva.
Till skillnad från tidigare år har FnollK 2021 även lagt till text angående anpassning till pandemirestriktioner.

\begin{beslut}
  \item fastställa verksamhetsplanen för FnollK 2021.
\end{beslut}


\section{Tidigarelagda ärenden}

\subsection{Proposition: Budgetändring för köksrenovering och riktade satsningar}
Sektionskassör David Winroth säger att Sektionsstyrelsen vill göra flera budgetändringar.
Dels handlar det om en riktad satsning som uppstått eftersom programmen har pengar över från förra året.
Dessa pengar ska bland annat läggas på brädspel och flipper.
Sammanlagt handlar det om 117 000kr som läggs till både som inkomst och utgift.
Focus kök ska även renoveras.
Den halvan av kostnaden som sektionen står för kommer dras direkt från programbidragen, och kommer uppgå till ungefär 250 000kr.

\begin{beslut}
  \item bifalla propositionen i sin helhet.
\end{beslut}


\subsection{Motion: Vice i FIF}
Ordförande i FIF Adam Johansson läser upp motionen.% TODO: link

Ledamot i FIF 2018 Ajsa Cuprija håller med om att motionen är behövd.

Nina Ekelund trodde att Vice ordförande existerade när hon aspade, och blev besviken när hon insåg att det inte fanns.
Hon tycker att det är naturligt att lägga till.

Filip Rydin håller med Ajsa.

Ilma Aase, som aspat FIF, tror en Vice ordförande kan bidra till FIF:s utveckling.

Eftersom Talman Ruben Seyer inte har koll så läses Sektionsstyrelsen motionssvar upp lite i efterhand. % TODO: Link
Sektionskassör David Winroth tar mot sin egen vilja på sig uppläsandet, och meddelar att de yrkar på bifall.

\begin{beslut}
  \item bifalla motionen i sin helhet. %TODO: link
\end{beslut}

\begin{ofraga}
  Ordförande i FIF Adam Johansson yrkar på att direktjustera motionen och att lägga till invalet i föredragningslistan.
  \begin{beslut}
    \item direktjustera beslutet att bifalla Motion om Vice i FIF.%TODO: link
    \item riva upp föredragningslistan.
    \item fastställa föredragningslistan med tidigare föreslagna ändringar samt följande ändring:% TODO: Link
    \begin{itemize}
      \item Lägga till punkt §14 b ii Val av vice ordförande.
    \end{itemize}
  \end{beslut}
\end{ofraga}


\section{Personval}
\subsection{Fabiola}
\subsubsection{Val av 1--10 ledamöter}
Det finns 5 sökande:
\begin{itemize}
    \item Julia Jansson
    \item Isabella Tepp
    \item Ajsa Cuprija
    \item Alma Blombäck
    \item Mathilda Gustavsson.
\end{itemize}

\paragraph{Julia Jansson} är 21, tm17 och har körkort.
Hon vill främja sammanhållningen bland kvinnor på sektionen, och tycker att \textit{Brudar och bärs} är kul.

\paragraph{Isabella Tepp} är 20, f19 och har körkort.
Hon tycker att Fabiola verkar kul, eftersom de som nuvarande medlemmarna i Fabiola är goa och det är bra att Fabiola finns.

Tarek Alhaskir undrar vilka erfarenheter Isabella kan ta med sig från DP till Fabiola.
Isabella svarar att hennes DuP-erfarenheter kan utnyttjas under \textit{Brudar och bärs}.

\paragraph{Ajsa Cuprija} är 22, f17 och har körkort.
Hon vill arrangera tillsammans med JämK, andra sektioner och andra föreningar på sektionen.
Hon är helt enkelt taggad på att arrangera.

\paragraph{Alma Blombäck} är 22, f17 och har körkort.
Alma har suttit i Fabiola tidigare, tycker om att sitta i Fabiola och vill därför sitta om i Fabiola.

\paragraph{Mathilda Gustavsson} är 21 tm19, och har körkort.
Hon har suttit i Fabiola under det gångna året, vilket hon tycker har varit roligt trotts att all verksamhet bedrivits digitalt.
Hon vill fortsätta jobba med att få bra gemenskap bland kvinnor på sektionen.

\begin{beslut}
  \item välja in 10 ledamöter i Fabiola.
  \item välja Julia Jansson, Isabella Tepp, Ajsa Cuprija, Alma Blombäck, Maja Gustavsson till ledamöter i Fabiola, samt vakantsätta 5 platser.
\end{beslut}


\subsection{Fysikteknologsektionens Idrottsförening, FIF}
\subsubsection{Val av ordförande}
Det finns 1 sökande:
\begin{itemize}
    \item Filip Rydin.
\end{itemize}

\paragraph{Filip Rydin} är 20, f19 och har körkort.
Han tycker att FIF verkar kul, och fick en bra känsla av föreningen under aspningen.

Ordförande i FIF Adam Johansson undrar hur Filip tänker kring att arrangera under pandemin.
Filip svarar att alla som aspar vill arrangera mycket, men att de till att börja med får ha arrangemangen digitalt.
Senare under året hoppas han att det kommer gå att ha fysiska arr i mindre grupper.

Kassör i FIF 2018 Tarek Alhaskir undrar om Filip och de andra asparna har några tankar som grupp kring det intresset för FIF.
Dessutom undrar han vad nästa motsvarighet till en skidresa blir.
Filip svarar på den andra frågan att de vill ha fallskärmshoppning.
På den första frågan svarar han att asparna inte har pratat så mycket om mängden arrangemang.

\begin{beslut}
  \item välja in Filip Rydin som ordförande i FIF.
\end{beslut}


\subsubsection{Val av vice ordförande}
Det finns 1 sökande:
\begin{itemize}
    \item Ilma Aase.
\end{itemize}

\paragraph{Ilma Aase} är 21, tm20 och har inte körkort.
Hon söker FIF eftersom hon tycker att det är kul med sport
Initialt tänkte hon söka Ordförande, men tyckte att motionens beskrivning på Vice ordförande passade henne bättre.

Ledamot i FIF 2018 Ajsa Cuprija undrar vilket Chalmersmästerskap Ilma helst vill arrangera.
Ilma svarar spökboll, vilket dock redan arrangeras av en annan sektion.
Men hon spelade innebandy när hon var yngre, så kanske det om det inte finns.
Efter ett tips från Ajsa landar hon på att hon kan sno CM i spökboll.

Ledamot i FIF Ludvig Rodung undrar hur hon ser på att hålla scribl.io-CM nu när det är svårt att ha fysiska arrangemang.
Det tycker Ilma låter jättebra.

Philip Öhman undrar om Ilma går med på att arrangera scribl.io-CM eftersom hon alltid vinner på scribl.io.
Han undrar även om hon kan låta någon annan vinna om det blir scribl.io-CM.
Ilma svarar att hon är för hets och tävlar för att vinna, så hon kan inte låta någon annan vinna.

\begin{beslut}
  \item välja in Ilma Aase som vice ordförande i FIF.
\end{beslut}


\subsubsection{Val av kassör}
Det finns 1 sökande:
\begin{itemize}
    \item Simon Rödén.
\end{itemize}

\paragraph{Simon Rödén} är 20, tm20 och har körkort.
Han tycker aspningen var kul och vill aktivera sektionen.

Erik Levin undrar om Simon har blivit dömd för skattebrott eller förskingring.
Simon svarar att han inte har blivit dömd får något av dessa brott.

Kassör i FIF Samuel Winqvist undrar om han kan få gratis choklad efter varje FIF-arrangemang nästa år.
Simon svarar att han hellre satsar på en skidresa om det går.

Hannes Johansson säger att FIF har mycket pengar över, och undrar vad det ska gå till om en skidresa inte går att genomföra.
Simon svarar att fallskärmshoppning är ett pandemisäkrare alternativ.

Tobias Wallström frågar om fallskärm kommer ingå vid ett sådant arrangemang.
Simon svarar att det ingår

\begin{beslut}
  \item välja in Simon Rödén som kassör i FIF.
\end{beslut}


\subsubsection{Val av 0--6 ledamöter}
Det finns 6 sökande:
\begin{itemize}
    \item Hugo Rådegård
    \item Marcus Johansson
    \item Nina Ekelund
    \item Andreas Erlandsson
    \item Philip Öhman
    \item Alvin Gustafsson
\end{itemize}

\paragraph{Hugo Rådegård} är 19, f20 och har körkort.
Han tycker om sport, speciellt badminton och vill arrangera på sektionen.
Han tycker att det varit en bra grupp som aspat.

Ledamot i FIF Amandus Reimer säger att många sittande i FIF spelar badminton.
Han undrar därför om Hugo kan anordna en badmintonturnering.
Hugo svarar att det går när det inte längre är en pandemi.

\paragraph{Marcus Johansson} är 19, f20 och har körkort.
Han tycker att FIF:s arrangemang under hösten varit roliga, idrottar mycket och vill arrangera som sittande under det kommande året.

Ledamot i FIF Eric Le undrar vilken onsdagssport som blir vanligast förekommande om Marcus får bestämma.
Marcus svarar att badminton fungerar bra.

\paragraph{Nina Ekelund} är 19, f20 och har körkort.
Hon söker eftersom aspningen var kul och för att hon trivs med de andra asparna.
Hon tycker att träning gör så att hon mår bättre när hon är stressad, vilket man lätt blir när man går F eller TM.
Har påstår sig ha många idéer på arrangemang som FIF kan genomföra.

Molly Sigfridsdotter undrar om det kan arrangeras Chalmersmästerskap i pulkaåkning nästa vinter.
Nina svarar att hon föreställer sig en tävling likt kappflygningen under Mottagningen; alla lag får bygga en pulka, och sen ser man vem som åker längst.

Eric Le vill höra mer om Ninas idéer på arrangemang.
Nina säger att bara vissa av hennes idéer är bra.
För att nämna några funderar hon på Just Dance på distans, tipspromenad på söndagar och att utvidga vad man klassar som sport.
Man kan till exempel börja arrangera e-sport.
Sammanfattningsvis går mycket att lösa med avstånd och vidgat tänk.

\paragraph{Andreas Erlandsson} är 21, f20 och har körkort.
Han söker eftersom han tycker att sport är bra, vilket gör FIF till ett klockrent val.
Han tycker att gänget som aspar är gött, och tror det blir riktigt bra om de blir invalda.

Ordförande i FIF Adam Johansson undrar vilken inofficiell post i FIF som Andreas vill ha under 2021.
Andreas tycker att alla poster verkar kul, men lutar mot sekreterare eller materialansvarig.

Ledamot i FIF Ludvig Rodung undrar om Andreas vill arrangera med systerföreningarna KfIK och ifK.
Andreas är positivt inställd, eftersom han tycker att det är kul med samarbete.
Han tror att arrangemangen kan bli bra för alla involverade sektioner.

\paragraph{Philip Öhman} är 23, tm20 och har bara moppekörkort.
Han tycker att FIF är en viktig förening, och att det är kul att få komma in lite till skolan även när studierna är på distans.
Han vill jobba med informationsspridning om han sitter i FIF, och att få upp kul bilder inför arrangemang likt de som FIF 2020 haft.

Adam Johansson säger att Philip är bra på Rocket League, och undrar därför vem i FIF 2020 som Philip tror är bäst på Rocket League.
Philip tror mest på David eftersom han bäst på vanlig fotboll och är FIFA-nörd.
Därefter tror han att Adam är näst bäst.

Hannes Johansson påstår att Philip är bra på Among Us, och därmed är bra på att ljuga.
Med detta som bakgrund och med Philips senaste uttalande i åtanke undrar han om det går att lita på Philip.
Philip svarar att han inte söker en förtroendepost, så det är okej om sektionen inte litar på honom.
Han ska ändå bara lägga upp memes.

\paragraph{Alvin Gustafsson} är 20, f20 och har körkort.
Han söker eftersom FIF hade kul arrangemang under Mottagningen.
Han gillar idrott och har sportat mycket som ung.

Adam Johansson undrar vilken inofficiell post Alvin är intresserad av.
Alvin svarar att han skulle vilja arrangera klassiska högstadieidrottsgrenar.

Hannes Johansson påstår att det är ett stort ansvar att arrangera varje onsdag.
Han undrar därför vilken sport som ska spelas på onsdagar, och varför Alvins svar är fotboll.
Alvin svarar att han har innebandy som favoritsport, men att han kollar på fotboll mycket.
Han kan därför tänka sig att arrangera även fotboll.

\begin{beslut}
  \item välja in 6 ledamöter i FIF.
  \item välja Hugo Rådegård, Marcus Johansson, Nina Ekelund, Andreas Erlandsson, Philip Öhman och Alvin Gustafsson till ledamöter i FIF.
\end{beslut}

\begin{ofraga}
  Sekreterare Felix Augustsson vill ajournera mötet i 25 minuter.
  \begin{beslut}
    \item ajournera mötet till 20.08
  \end{beslut}
\end{ofraga}


\subsection{Foton}
\subsubsection{Val av 1--6 ledamöter}
Det finns 6 sökande:
\begin{itemize}
    \item Jacob Burman
    \item Victor Enevold
    \item Alma Lund
    \item Annie Carlsson
    \item Axel Flordal
    \item Tess Ellertsson
\end{itemize}

\paragraph{Jacob Burman} är 21, tm19 och har körkort.
Som F6 har han sett Fotons arbete, och tycker det verkar mysigt.
Att man som Foton får gratis mat är även ett plus.

Ledamot i Foton Albert Vesterlund undrar om Jacob känner sig som partikel eller våg när han vaknar.
Jacob svarar våg.

Emma Ödman undrar vilket Mottagningsarrangemang som Jacob helst vill fotografera.
Jacob svarar F6 arrangemang.

Tobias Wallström undrar hur man tar bästa screenshot:en.
Jacob svarar att han använder Gyazo så att han alltid är redo.

\paragraph{Victor Enevold} är 20, f20 och har inte körkort.
Han tycker det är kul att fotografera och vill fånga fler roligheter på bild.
Han vill även använda drönare och undervattenskamera om tillfälle ges.

Ledamot i Foton David Winroth undrar vad Victor har för vision för Nollan-bilder.
Victor svarar att ju galnare bilder, desto bättre.
Bilderna borde dessutom sättas upp där de syns väl.

Ledamot i Foton Anton Alm har hört rykten om att sektionen har mycket pengar, och undrar om Victor vill äska pengar för en drönare.
Det vill Victor.

\paragraph{Alma Lund} är 25, f16 och har körkort.
Hon vill sitta i Foton för att fotografera folk som har det kul.
Hon tycker även det är kul att arbeta med bilderna efter att de är tagna.

Alber Vesterlund undrar hur många A4-sidor lång Alma tycker att en äskning om ny kamera borde vara.
Alma svarar att en kortare äskning är en bättre äskning, så en A4-sida.

Anton Alm undrar vilket Almas favoritfilter på Instagram är.
Alma svarar \#nofilter.

Karin Hult tycker att Alma som äldre har sett hur Nollan-bilder brukar se ut, och undrar om hon kan ordna så att de sätts upp igen.
Alma tycker att Nollan-bilder är kul och vill försöka fixa det.

\paragraph{Annie Carlsson} är 19, f20 och har körkort.
Hon vill engagera sig och tycker om fotografering.

David Winroth undrar hur många sidor en äskning på 15 000kr för en ny kamera som Annie skriver kommer vara.
Annie svarar att om hon bara skriver en sida så kan hon lägga tiden hon sparar på att fotografera mer.

Ledamot i Foton Ida Ekmark undrar vilken den bästa ställningen är.
Att fota i alltså.
Annie tycker att den bästa ställningen är den vanliga: stående framifrån.

\paragraph{Axel Flordal} är 18, tm20 och har körkort.
Han tycker att det är kul att både ta och kolla på bilder.
Han vill se till att det finns bilder genom att ta dem.

Albert Vesterlund påstår att tidigare Foton både mer och mindre bra på att ta gruppfoton.
Han undrar därför hur man bäst fotograferar en foton.
Axel att man antingen kan ta bilder på varandra, eller be en patet hjälpa till.

Alexandru Golic undrar hur många sidor en äskning på 18 000kr borde vara.
Axel svarar en halv.

\paragraph{Tess Ellertsson} är 20, f20 och har inte körkort.
Hon tycker att det är roligt att fotografera och vill lära sig mer.
Hon tycker att det från aspningen verkat som att Foton har haft det kul på arrangemangen de gått på.

David Winroth påstår att de nuvarande ledamöterna i Foton har varierande mötesnärvaro.
Han undrar därför hur många procent av möten Tess tänker gå på?
Tess svarar 100\%.
Albert Vesterlund undrar hur mycket Tess tänker skjuta på att editera bilder.
Tänker hon låta bilderna ligga på sin hårddisk fram tills det är två veckor kvar tills hon går av likt en viss sittande i Foton?
Tess svarar att hon kan ge editeringsansvaret till någon annan.

Anton Alm undrar vilket arrangemang Tess är extra taggad på att fotografera.
Tess svarar att hon ser fram emot att fotografera hela Mottagningen, men framför allt gyckelsittningen.

\begin{beslut}
  \item välja in 6 ledamöter i Foton.
  \item välja Jacob Burman, Victor Enevold, Alma Lund, Annie Carlsson, Axel Flordal och Tess Ellertsson till ledamöter i Foton.
\end{beslut}


\subsection{Finform}
\subsubsection{Val av chefredaktör tillika ansvarig utgivare}
Det finns 1 sökande:
\begin{itemize}
    \item Fredrik Skoglund.
\end{itemize}

\paragraph{Fredrik Skoglund} är 21, f19 och har körkort för personbil och medeltung lastbil.
Han säger att han sett fram emot Finform och nu äntligen kan axla rollen eftersom han är fri från FnollK.
Hans plan är att ta Finform till nya höjder.

Tobias Wallström undrar om Finform kan börja betala för bilder på Dragos, likt The Daily Bugle i Spindelmannen.
Fredrik har hört talas om att Finform har pengar över, och att foton på Dragos är värda sin vikt i guld.
Så det gäller att ha tillräckligt tunga foton.

Redaktör i Finform Mijo Thoresson undrar hur Fredrik kommer visa uppskattning för den som ansvarar för layouten på tidningen.
Fredrik svarar att han vill ge en tron eller ett pris till Mijo om han tar på sig det ansvaret igen.

Emma Ödman har hört talas om en Finformhusvagn, och undrar vad ska den vara till.
Fredrik säger att den kan användas som rum eftersom det är ont om plats på campus.
Det ger även Finform möjlighet att göra flygande reportage.

Vincent Udén har i en viss Modul läst om att man ska kontakta Billy, trotts att Billy inte längre var med i FnollK.
Han undrar därför hur Fredrik ska undvika liknande fel i Finform.
Fredrik tycker att fel ger karaktär, samt att det sektionsmedlemmar ska kunna kontakta Billy.

Chefredaktör Oscar Nilsson påstår att det är svårt att kalla till möten, och undrar hur Fredrik hur han ska få alla redaktörer att dyka upp när det är möte.
Fredrik tycker att det är en nyckfull fråga, och att det är svårt att locka med pizza och kall dryck på Discord.
Han har där emot en idé om en straff-quota som gör så att man till slut måste dyka upp på möten.

Redaktör i Finform 2016 Karin Hult undrar hur många citat som är rimligt att ha med per nummer.
Fredrik tycker citat är kul; de är en av de många bästa grejerna med Finform.
I slutändan anser ha att det är en fråga om plats.
Men så länge det finns mindre fonter kan man ha fler citat.
Om fonten blir tillräckligt liten kan förstoringsglas på Focus behövas.

Sektionskassör David Winroth säger att det är ett känt faktum att Finform inte spenderar pengar: det står i överlämningen han fick att man ska hetsa Finform att spendera mer.
Så han undrar därför hur Fredrik planerar att spendera pengar förutom husvagnsinköp.
Fredrik svarar att han vill lägga mer pengar i stadiet då artiklar skrivs.

Alexandru Golic påstår att rektangel är en ful form, och undrar därför en tidning som heter Finform bör ha istället.
Fredrik påpekar att rektangel är en billig form.
Romber är bättre eftersom de är sneda, men han är osäker på hur ryggen på tidningen ska göras.
Till slut bestämmer han sig för att kub är den bästa formen för tidningen att ha.

\begin{beslut}
  \item välja in Fredrik Skoglund till Chefredaktör tillika ansvarig utgivare för Finform.
\end{beslut}


\subsubsection{Val av kassör}
Det finns 1 sökande:
\begin{itemize}
    \item Ludvig Nordqvist.
\end{itemize}

\paragraph{Ludvig Nordqvist} är 20, f19 och har inte körkort.
Han söker kassör eftersom han gillar pengar.

Redaktör i Finform Mijo Thoresson undrar om Ludvig är villig att budgetera för en tron till en som gör layouten för Finform.
Ludvig svarar att det beror på om det finns pengar över; Finform behöver nämligen en ny kamera.

\begin{beslut}
  \item välja Ludvig Nordqvist till kassör i Finform.
\end{beslut}


\subsubsection{Val av 2--8 redaktörer}
Det finns 10 sökande:
\begin{itemize}
    \item Alexandru Golic
    \item Alexander Körner
    \item Mijo Thoresson
    \item Didrik Palmqvist
    \item Hampus Hansen
    \item Olof Forsberg
    \item Markus Utterström
    \item Rakel Hellberg
    \item Johan Bruhn
    \item Erik Dahlstedt
\end{itemize}

\paragraph{Alexandru Golic} är 22, f18 och har inte körkort.
Han har länge funderat på Finform eftersom han gillar dumma skämtsamma saker.
Han har inte sökt innan eftersom han suttit i Sektionsstyrelsen.
Till sist tillägger han att det är dags att skriva roliga texter.

Tobias Wallström påstår att Alex är bra på ordvitsar, och ber om ett smakprov.
Dessutom vill han få ett exempel på en artikel Alex kan tänka sig skriva.
Alex svarar \enquote{När jag var ung lovade jag att inte sätta mig, och det står jag fortfarande för}.
På den andra svarar Alex att han tycker om sina förfäder i Romarriket, och vill sätta dessa ledare mot programansvariga.
Till exempel kan man skriva om huruvida Julius Ceasar eller Jana Madjarova skulle vinna.

Joseph Löfving har hört att Alex kommer spendera tid utomlands, och undrar vad Alex kan skriva som utrikeskorrespondent.
Alex svarar att sektionsmedlemmar tycker om öl, och att det därför skulle vara intressant att skriva om öl som inte bara kommer från \censor*{8} Sverige.

Redaktör i Finform 2016 Karin Hult undrar Alex kan arrangera en Ohmsitts om pandemirestriktionerna lyfts innan de går av.
Alex svarar att han kan försöka, men vill minnas att det satt 40 personer i Finform ett år.
Så restriktionerna får vara helt lyfta innan Ohmsitts blir görbart.
Om så är fallet kan han lobba för det i eller utanför landet.

\paragraph{Alexander Körner} är 22, f18 och har körkort.
Han tycker att det är viktigt med ett annorlunda perspektiv, och vill visa att även jävig journalistik är bra.
Han är redo att sälja ut sig, och är inte rädd för att ställa de svåra frågorna och gräva.

Albert Vesterlund undrar vad Alexander tycker om höga trösklar.
Alexander svarar att tröskeln ska reflektera vad som finns bakom dörren.

Tobias Wallström undrar om Alexander har typ av artiklar han tycker om att skriva.
Alexander svarar att han skrivit om allt från raggning till överlevnadstips.
Förra året lovade han att han skulle intervjua fackspråk om ghettoslang.
I sitt kandidatarbete har han nu ett inplanerat möte med fackspråk, så om han blir invald kan han infria förra årets vallöfte det kommande året.

\paragraph{Mijo Thoresson} är 20, f19 och har körkort.
Han söker eftersom det är kul att skriva, och kan tänka sig att göra layouten i år igen.

Nyinvald chefredaktör Fredrik Skoglund undrar InDesign:er till InDesign:er vilket Mijos favoritverktyg i InDesign är.
Mijo kan inga namn på verktyg, men tycker att Ctrl-C Ctrl-V är väldigt användbart

\paragraph{Didrik Palmqvist} är \st{21} 20, f19 och har inte körkort.
Han har redan suttit ett år, vilket var kul, och känner att han har mycket kvar att göra i Finform.

Ludvig Nordqvist tycker att Didrik är 20.
Didrik svarar att han försökte sig på fake news.

\paragraph{Hampus Hansen} är 21, f20 och har körkort för bil och EU-moped.
Han vill sitta i Finform för att få en platform för att utöva chockerande och sanningsenlig journalistik.

Fredrik Skoglund undrar vad Hampus ståndpunkt på top-10-listor är.
Hampus säger att WatchMojo eller Finform kan klara av det, men för alla andra för andra är det bättre att hålla sig till top-5-listor.

\paragraph{Olof Forsberg} är 19, f20 och har körkort.
Han tycker att Finform är intressant och tycker att det är kul att skriva.

Fredrik Skoglund ponerar att Olof har obegränsat med kapital för ett reportage; vart skulle han då åka?
Om et inte var för pandemin skulle Olof intervjuat folk i Alperna, eller åkt till ett varmt land under sommaren.

\paragraph{Markus Utterström} är 21, f19 och har körkort.
Han söker eftersom det är kul med Finform, kul att skriva och kul med gött häng.

\paragraph{Rakel Hellberg} är 23, tm18 och har inte körkort.
Hon söker Finform eftersom det är viktigt med demokrati.
Till att börja med aspade hon inte eftersom hon inte har samma humor som brukar förekomma i Finform, vilket skulle göra det svårt att skriva likadana texter själv.
Efter att ha pratat med några sittande i Finform blev hon uppmuntrad att skriva om annat, och söker därför nu.
Hon tillägger att hon varken är grabb eller 666, så hon skulle helt klart bidra till mångfalden i Finform.

Axel Prebensen undrar vilken Rakels favoritjournalist är, och varför det är Alex Jones.
Rakel undrar om hon borde veta vem det är.
På Axels första fråga svarar hon Fredrik Strage; hon hoppas kunna skriva i hans stil om hon blir invald.

Karin Hult undrar vilken sorts artikel som är viktigast i ett Nollannummer.
Rakel svarar att Nollan fort fattar hur det ligger till, så det är viktigt att tidligt förmedla när det var bättre.

\paragraph{Johan Bruhn} är 22, tm18 och har körkort.
Han har suttit i Finform 2020, men skrev inte så mycket som han hade velat.
Han känner därför att han har mycket skrivande kvar i sig.

\paragraph{Erik Dahlstedt} är 20, tm20 och har körkort.
Han söker Finform eftersom det verkar kul, och för att lära känna nya människor.

Tarek Alhaskir undrar om han vill skriva nån viss typ av texter eller om han har en favoritdel av Finform.
Erik svarar att han älskar film, och skulle därför gärna skriva om film.

\begin{ofraga}
  Olof Cronqvist yrkar på att
  \begin{itemize}
    \item Riva upp föredragningslistan.
    \item Fastställa föredragningslistan med tidigare föreslagna ändringar samt följande ändring:
    \begin{itemize}
      \item Lägga till punkt §14' Ändra antalet redaktörer i Finform mellan punkt §14 d Finform och punkt §14 e Sångförmän.
    \end{itemize}
    \item Bordlägga nuvarande punkt till efter punkt §14'.
  \end{itemize}
  \begin{beslut}
    \item avslå yrkandet.
  \end{beslut}
\end{ofraga}

\begin{beslut}
  \item välja in 8 redaktörer i Finform.
  \item välja Alexandru Golic, Alexander Körner, Mijo Thoresson, Didrik Palmqvist, Hampus Hansen, Markus Utterström, Rakel Hellberg och Johan Bruhn till redaktörer i Finform.
\end{beslut}


\subsection{Sångförmän}
\subsubsection{Val av 0--6 sångförmän}
Det finns 2 sökande:
\begin{itemize}
    \item Erik Bivrin
    \item Albert Vesterlund.
\end{itemize}

\paragraph{Erik Bivrin} är 22, f18 och har körkort.
Han söker eftersom det är kul att sjunga med folk.
Han sökte även förra året, men blev bestulen på möjligheten att sjunga på grund av pandemin.

\paragraph{Albert Vesterlund} är 20, tm19 och har körkort.
Han tycker det är kul med sånger, särskilt att liva upp sittningar med sång.
Han vill dessutom utnyttja sin sjungbok.

\begin{beslut}
  \item välja in 6 sångförmän.
  \item välja Erik Bivrin och Albert Vesterlund till sångförmän, samt vakantsätta 4 platser.
\end{beslut}


\subsection{Fanfareriet}
Fanbärare Karin Hult informerar om att man i Fanfareriet hissar tyg och bär fanan vid Cortège, Mottagning och Mösspåtagning.

\subsubsection{Val av flaggmarskalk}
Det finns 1 sökande:
\begin{itemize}
    \item Albert Vesterlund.
\end{itemize}

\paragraph{Albert Vesterlund} är 20, tm19 och har körkort.
Han tycker att det är viktigt att flaggan hissas.

Tobias Wallström undrar om flaggmarskalker bär flaggor eller fanor, samt vad skillnaden är.
Albert tror att det heter fonstyget, men tillägger att man lär sig det om efter att man blivit flaggmarskalk.

Ida Ekmark undrar om Albert kommer se till att den nya flaggan införskaffas.
Det kan Albert; det borde vara enkelt eftersom designen redan är vald.

Alexandru Golic undrar om nya flaggan kan döpas till F-flaggan eller F-fanan.
Albert säger att den kan döpas om såvida det inte finns bättre namn.

\begin{beslut}
  \item välja Albert Vesterlund till flaggmarskalk i Fanfareriet.
\end{beslut}


\subsubsection{Val av 1--2 fanbärare}
Det finns 2 sökande:
\begin{itemize}
    \item Sara Wäpling
    \item Ida Ekmark.
\end{itemize}

\paragraph{Sara Wäpling} är 23, f16 och har körkort.
Hon tycker att det är kul att klä upp sig, har suttit i Balnågonting och tycker att det är kul med finklädda tillställningar.

Fanbärare Karin Hult tycker att klädkoder är förvirrande för kvinnor, och undrar därför vad högtidsdräkt innebär?
Sara svarar att klänningen då ska nå ner till golvet, inte ska vara något man går till stranden i och att den tekniskt sett ska ha täckta axlar.
Men, lägger hon till, hur många sådana balklänningar finns det egentligen?

\paragraph{Ida Ekmark} är 22, f18 och har körkort.
Hon har suttit som fanbärare i två år, tycker det är kul och vill göra det igen.

\begin{beslut}
  \item välja in 2 fanbärare i Fanfareriet.
  \item välja Sara Wäpling och Ida Ekmark till fanbärare i Fanfareriet.
\end{beslut}


\subsection{Bilnissar}
Mekanisk bilnisse Willem de Wilde berättar att bilnissarna ser till att bilen kör.
Ekonomisk bilnissen skickar dessutom fakturor.
Han utlovar även en Ohmsitts när pandemin är över.


\subsubsection{Val av ekonomisk bilnisse}
Det finns 1 sökande:
\begin{itemize}
    \item Anton Wikström.
\end{itemize}

\paragraph{Anton Wikström} är 21, f18 och har körkort.
Han har varit bilnisse i två år och kan tänka sig att ställa upp igen.

Martin Due undrar även i år om Helikoptern är bak- eller framhjulsdriven.
Anton svarar att Helikoptern nog är framhjulsdriven.

\begin{beslut}
  \item välja Anton Wikström till ekonomisk bilnisse.
\end{beslut}


\subsubsection{Val av mekanisk bilnisse}
Det finns 1 sökande:
\begin{itemize}
    \item Willem de Wilde.
\end{itemize}

\paragraph{Willem de Wilde} är 22, f18 och har körkort.
Han har suttit ett år som bilnisse, så han vet hur allt fungerar.
Han vet även till skillnad från Anton att motormontering på helikoptern är vertikal, och att Helikoptern därför är bakhjulsdriven.
Han informerar även om att helikoptern går på diesel och lite andra roliga grejer.

\begin{beslut}
  \item välja Willem de Wilde till mekanisk bilnisse.
\end{beslut}


\subsection{Blodgruppen}
Ida Ekmark informerar om att Blodgruppen främjar blodgivarkulturen på sektionen.
De arrangerar vanligtvis en blodlunch per läsperiod.
Blodgruppens uppdrag har varit svåra att genomföra under pandemin, men i övrigt är det ett bra och givande sätt att hjälpa samhället.


\subsubsection{Val av ansvarig blodutgivare}
Det finns 1 sökande:
\begin{itemize}
    \item Ida Ekmark.
\end{itemize}

\paragraph{Ida Ekmark} är 22, f18 och har körkort.
Hon tycker att Blodgruppen är viktig, och vill att den ska leva vidare.

Jacob Burman tycker att det är viktigt att veta vilken blodgrupp man tillhör.
Ida svarar att hon tillhör 0-, så hon kan ge blod till alla.

Alexandru Golic säger att hon förra året var den enda o blodgruppen som kunde ge blod, och undrar därför hur många liter blod hon kan ge på en gång?
Ida svarar att hon kan ge så mycket som krävs, så alla liter.

\begin{beslut}
  \item välja Ida Ekmark till ansvarig blodutgivare i Blodgruppen.
\end{beslut}


\subsubsection{Val av 1--4 ledamöter}
Det finns 3 sökande:
\begin{itemize}
    \item Linnea Hallin
    \item Anton Wikström
    \item Mijo Thoresson.
\end{itemize}

\paragraph{Linnea Hallin} är 21, tm18 och har körkort.
Hon tycker det är viktigt att ge blod, men kan inte ge själv och vill därför få andra att ge.
Hon tycker även att det är kul att laga blodlunch och vill att Blodgruppen ska fortsätta vara aktiv.

Tobias Wallström tycker att få andra att ge blod utan att själv ge är lite som att vilja att alla andra betalar skatt utan att betala själv.
Linnea tycker att hon har en bra ursäkt som har givit tidigare, men inte får ge längre.

\paragraph{Anton Wikström} är 21, f18 och har körkort.
Han har suttit innan, och resonerar att ju fler kockar man har i köket, desto mer spenatsoppa blir det.

\paragraph{Mijo Thoresson} är 20, f19 och har körkort.
Han tycker att det är viktigt att ge blod, så eftersom det finns platser över i Blodgruppen vill han bidra.

\begin{beslut}
  \item välja in 4 ledamöter i Blodgruppen.
  \item välja Linnea Hallin, Anton Wikström och Mijo Thoresson till ledamöter i Blodgruppen, samt vakantsätta en plats.
\end{beslut}


\subsection{Kräldjursvårdare}
Alfred Weddig Karlsson har med sig Tilde, och berättar att man som kräldjursvårdare får ta hand om Tilde.
När man får ha arrangemang igen ska man dessutom ta med sig Tilde på dessa.

\subsubsection{Val av 1 kräldjursvårdare}
Det finns 2 sökande:
\begin{itemize}
    \item Willem de Wilde
    \item Tess Ellertsson.
\end{itemize}

\paragraph{Willem de Wilde} är 22, f18 och har körkort.
Willem har ett stort intresse för kräldjur efter att ha hittat information på internet om tankekontroll.
Som kräldjursvårdare kommer han, Tilde och reptiler fortsätta kontrollera regeringen och befolkningen.
Han lovar även att ta med sig Tilde på alla arrangemang.

Kräldjursvårdare 2018 Karin Hult påpekar att kräldjur inte får finnas på Focus.
Willem svarar att han kan låta kräldjuren vara i kräldjurshörnana, också känt som styretrummet, där reptiler redan utövar sin tankekontroll.

David Winroth påpekar att sitta i Fullmäktige är ett annat sätt att få makt.
Willem tycker att sitta i Fullmäktige är lite Bill Gates-nivå på makt, och känner därför att han kan skjuta upp det i ett år.

Alexandru Golic undrar om Willem är för rädd för att säga för mycket.
Hur ser Willem till att han inte hamnar i en \enquote{olycka}?
Willem svarar att \enquote{if you cant fight them join them}.
Han tillägger även att han till skillnad från de till synes flesta på mötet kommer stå på rätt sida av historien.

Karin Hult undrar vilket kön Tilde har.
Willem svarar att han tycker att frågan är politiskt, och vill därmed inte yttra sig innan han kontaktat sin advokat.

\paragraph{Tess Ellertsson} är 20, f20 och har inte körkort.
Victor Enevold agerar ombud, och säger att Tess söker Kräldjursvårdare eftersom posten är både häftig och viktig.
Hon ser en stor utvecklingspotential; hon vill bland annat skaffa en \enquote{förbjudet med kräldjur}-skylt på Focus och erbjuda en utbildning om vad kräldjur faktiskt är.

\begin{beslut}
  \item välja Tess Ellertsson till kräldjursvårdare.
\end{beslut}


\subsection{Bakisclubben (BC)}
\subsubsection{Val av 0--6 bakisar}
Det finns 2 sökande:
\begin{itemize}
    \item Rebecka Mårtensson
    \item Hugo Rådegård.
\end{itemize}

\paragraph{Rebecka Mårtensson} är 19, f20 och har inte körkort.
Hon tycker om att baka; hon bakade bland annat 46 lussekatter i julas.
Därför tycker hon att Bakisclubben passar bra för henne.

Alexandru Golic påminner om att Rebecka är en känd från SVT, och undrar hur hon kan utnyttja det för Bakisclubbens vinning.
Rebecka svarar att det ger henne fler möjligheter att sprida bakning på sektionen nu eftersom folk nu lyssnar när hon talar.

\paragraph{Hugo Rådegård} är 19, f20 och har körkort.
Han tycker att det är väldigt kul att baka och vill göra det mer.
Han kan göra detta nya bakande för sektionen, för vem gillar inte nybakat bröd varje dag?

Felix Augustsson undrar om det innebär att han utlovar bröd varje dag till alla sektionsmedlemmar.
Hugo svarar att han inte vet om budgeten tillåter det.

Emma Ödman ponerar att han har en tillräckligt stor budget; vilken sorts bröd bakar han då?
Hugo svarar att han då skulle baka både mörkt och ljust bröd.
Hans mål skulle vara att baka ett nytt bröd varje dag, men är rädd för att det är svårt att komma på 365 olika brödsorter.
Men han tillägger att det åtminstone borde bakas till kanelbullens dag.

David Winroth påstår att Bakisclubben har urgamla traditioner såsom Bakisbrunch, och undrar om Hugo kan lova att arrangera en sådan.
Hugo svarar att han kan inte kan love någonting, men det är helt möjligt att det går att arrangera.

Adam Johansson gillar protein och tyckte därför att förra årets chokladbollar var otillräckliga.
Han undrar därför om Hugo kan lova mer protein i år.
Hugo svarar att det kanske går när det inte längre är en pandemi, men att han i nuläget inte kan säga ja eller nej utan bara gärna.

Alexandru Golic påpekar att Hugo som ledamot i FIF kommer gå minus i kalorier, men som bakis kommer gå plus i kalorier.
Så tror Hugo att han kommer gå mer plus eller minus?
Hugo svarar att han är nöjd med sin form, så han satsar på plus minus noll.
Men samtidigt innebär FIF mer arbete än Bakisclubben, så det kanske blir minus ändå.

\begin{beslut}
  \item välja in 6 bakisar i Bakisclubben.
  \item välja Rebecka Mårtensson och Hugo Rådegård till bakisar i Bakisclubben, samt vakantsätta 4 platser.
\end{beslut}

\begin{ofraga}
  Martin Due yrkar på 10 min luftpaus, dvs ajournering av mötet.
  \begin{beslut}
    \item ajournera mötet till 23.23
  \end{beslut}
\end{ofraga}


\subsection{Spidera}
\subsubsection{Val av 2--10 teknologer}
Det finns 6 sökande:
\begin{itemize}
    \item Eric Carlsson
    \item Victor Salomonsson
    \item Patrik Wallin Hybelius
    \item Viktor Wernholm
    \item Hugo Spencer
    \item Ossian Eriksson.
\end{itemize}

\paragraph{Eric Carlsson} är 23, f17 och har körkort.
Han söker Spidera eftersom det är intressant och lärorikt.
Han tycker att det är viktigt att sektionen kan ha egna tekniska lösningar, vilket han kan lösa.

\paragraph{Victor Salomonsson} är 20, tm19 och har körkort.
Han söker Spidera \enquote{för att det är nice, typ}.

\paragraph{Patrik Wallin Hybelius} är 24, går f17 och har körkort.
Han har suttit ett år, men har inte gjort så mycket han velat.
Därför vill han istället göra mer i år.

\paragraph{Viktor Wernholm} är 21, f18 och har körkort.
Han söker främst eftersom det verkar intressant; han vill lära sig om ftek och hur det går att förbättra ftek.

\paragraph{Hugo Spencer} är 17, f20 och har inte körkort av naturliga skäl.
Har inte arbetat så mycket med servrar, men gjort mycket arbete på Wikipedia, bland annat scriptande i Javascript och SQL-querying.

\paragraph{Ossian Eriksson} lyckas inte få en uppkoppling, och personintervjun kan därmed inte genomföras.

\begin{beslut}
  \item välja in 10 teknologer i Spidera.
  \item välja Eric Carlsson, Victor Salomonsson, Patrik Wallin Hybelius, Viktor Wernholm, Hugo Spencer och Ossian Eriksson till teknologer i Spider, samt vakantsätta 4 platser.
\end{beslut}


\subsection{Sektionsnörd}
Sektionsnörd David Bååw informerar om att sektionsnörden är den högsta auktoriteten när det kommer till Dragoskunskap.
Som sektionsnörd håller man även i en Dragosföreläsning, en Dragostenta, visar Fantomenfilm och organiserar Fantomentidningar.

\subsubsection{Val av 1 sektionsnörd}
Det finns 3 sökande:
\begin{itemize}
    \item Maja Rhodin
    \item Karin Hult
    \item Andreas Erlandsson.
\end{itemize}

\paragraph{Maja Rhodin} är 19, f20 och har körkort sedan 2 månader tillbaka.
Hon tycker att Dragos är cool.
Hennes far är Dragosnörd, så hon blev överlycklig när hölls en hel föreläsning om Dragos under Mottagningen.
Hon målade Dragos gestalt på ett flygplan under Mottagningen.

Sektionsnörd David Bååw undrar dels hur hon ser på traditionen att sektionsnörden aspar och söker F6, och dels vilken ring Dragos har på sin vänstra hand.
Maja svarar att hon kan fortsätta traditionen; hon aspar redan F6 och traditioner är kul.
Vidare svarar hon att vänster hand är närmast hjärtat, så på Dragos vänstra hand är ringen som han inte slår skurkar med.

Sektionsnörd 2018 Tobias Wallström undrar dels om Maja har några anmärkningar på Dragosföreläsningen, och dels vad Dragos hund heter.
Maja svarar att Dragos har en bergsvarg som heter Devil.
Föreläsningen tycker Maja borde innehålla information om Dragos märken i ansiktet.
Men framför allt borde föreläsningen visa mer Fantomenglädje istället för bara fakta.

Hugo Spencer undrar om Maja kan tänka sig att måla Dragos på mer än flygplan.
Maja svarar att hon alltid tyckt om att teckna Fantomen, så hon kan tänka sig måla Dragos på Focus om de som målar Focus vill ha hjälp.

Alexandru Golic undrar om Maja som kunnig om Fantomen vet hur bra Fantomen tål en vinkelslip.
Maja vet inte vad en vinkelslip är, och ber därför om ett förtydligande.
Olyckligtvis har Alexandru inte heller så bra koll på vad en vinkelslip är, så Maja får gissa sig fram.
Hon säger att hon aldrig läst om Fantomen och vinkelslipar, men tror att Fantomen kan klara av att hantera det.

Erik Bivrin påstår att bonuspoängen från Dragostentan försvunnit från Janas dugga, och undrar om Maja kan få tillbaka dem.
Maja tycker att både skyddshelgon och Jana är bra traditioner, så hon håller med om att bonuspoängen borde återinföras.

Samuel Martinsson undrar hur Maja ser på att göra Dragostentan svårare för att få en större utmaning.
Han undrar även vad hon tycker om Nollan som inte kan skriva namn och CID.
Maja svarar att hon är en förespråkare för att intresse för Dragoskunskap är viktigare än innehavande av Dragoskunskap.
Ribban ska givetvis inte vara för låg, men hon ser inte heller hur en hög ribba gynnar överförandet av kunskap.
På Samuels andra fråga svarar hon att hon tycker att det är lite oroande att Nollan skriver fel på den typen av fält.
Om hon blir sektionsnörd kommer hon ha en hel slide om hur man skriver sitt namn.

\paragraph{Karin Hult} är 25, f15 och har körkort.
Hon har varit intresserad av att vara sektionsnörd i några år, och har nu tillräckligt få omtentor för att kunna hålla i en bra Dragosföreläsning.
Hon nämner även att hon brukade läsa Fantomenkrönikor när hon var liten.

David Bååw nämner återigen traditionen att sektionsnörden aspar och söker F6, och undrar hur Karin ställer sig till det.
Han undrar även om Karin kan nämna några skatter som Dragos har.
Karin tycker att det räcker att hon redan suttit i F6.
Och hon svarar även att två av Dragos skatter är Excalibur och Caesars lagerkrans.

\paragraph{Andreas Erlandsson} är 21, f20 och har körkort.
Att söka sektionsnörd är för honom ett kall, då han har fått Dragos välsignelse runt matbordet som liten.
Detta kom sig då hans far brukade läsa dagsstrippar i Smålands-tidningen, och Andreas vill nu föra vidare denna välsignelse till Nollan.

David Bååw undrar dels hur Andreas ser på traditionen att sektionsnörden aspar och söker F6, och dels när Fantomenlegenden började.
Andreas svarar att Fontomenlegenden började 1536, och att FIF gör så att han inte känner att han kan sitta i F6, men att han fortfarande kan aspa.
Och man vet aldrig, F6 kanske är så roligt att han ändrar sig?

Emma Ödman tror att man som sektionsnörd ser sig själv i Dragos, och undrar därför vad Andreas och Dragos gemensamma nämnare är.
Andreas svarar att \enquote{Inte döda, bara avväpna} är ett motto som resonerar med honom.

Joel Sandås undrar när de enligt Andreas bästa seriestripparna publicerades.
Andreas svarade att perioden 2006-2007 var bra, men att det sedan kom en dipp vid 2010.
Nu för tiden anser han dock att de är bra igen.

\begin{beslut}
  \item välja Maja Rhodin till sektionsnörd.
\end{beslut}


\subsection{Game Boy}
Game Boy Mikael Eriksson informerar om att Game Boy köper in brädspel och arrangerar brädspelskvällar.

\subsubsection{Val av 0--6 Game Boys}
Det finns 5 sökande:
\begin{itemize}
    \item Viktor Salomonsson
    \item Erik Broback
    \item Tobias Gabrielii
    \item Mikael Eriksson
    \item Viktor Wernholm.
\end{itemize}

\paragraph{Viktor Salomonsson} är 20, tm19 och har körkort.
Han tycker om Game Boy och har varit på alla Game Boys arrangemang sedan han började.
Han tycker även att chips och spel är gött.

\paragraph{Erik Broback} är 20, f19 och har körkort.
Han tycker att det är kul med brädspel.
han anser att brädspelskvällarna är både roliga och bra för sektionen, och vill därför ställa upp och anordna dem.

\paragraph{Tobias Gabrielii} är 23, f17 och har körkort.
Han tycker att brädspel är kul, att arrangemang är kul och har äntligen tid att sitta i Game Boy.

Maja Rhodin undrar om den nya deluxe edition av Fantomenspelet som införskaffats till Focus är den skandinaviska versionen.
Tobias har ingen aning.

Erik Levin tycker att det är mycket nattligt med brädspelskväll.
Han påpekar att Tobias ser väldigt trött ut trots att klockan bara är 00.20
Han undrar därför hur Tobias kommer att hantera att arrangera under nattetid.
Tobias svarar att det är skillnad på ett sektionsmöte och en brädspelskväll.

Alexandru Golic påstår att Tobias är finansmannen, och undrar därför om det kommer spelas Finans om Tobias blir invald.
Tobias svarar att han tycker att Finans är kul; problemet är att nästan ingen annan tycker att det är kul.
Han kan dock tänka sig att spela det själv, om det inte vore för att Focus Finans är så sliten.

Tobias Wallström undrar vad Tobias Gabrielii tycker om spel på börsen.
Tobias Gabrielii har inte hållit på med sånt; han tycker inte att pengar och spel går ihop.
Men han påpekar att FTEK-aktien som det var på tal att sektionen skulle köpa har gått upp 50\% i värde.

\paragraph{Mikael Eriksson} är 21, tm19 och har körkort.
Han har suttit i Game Boy i ett år och tyckte det var kul.
Så när det inte såg ut att bli fyllt i år utan honom sitter han gärna om.

\paragraph{Viktor Wernholm} är 21, f18 och har körkort.
Han tycker i grunden att alla spel är kul, som exempelvis datorspel och sport, men han tycker framför allt om brädspel.
Han tror att det är stor potential att fysiska spelkvällar kommer kunna arrangeras under 2021.
Han har nya idéer om olika spelkvällskoncept man kan testa.
Till sist säger han även att han vill utveckla Game Boy om Game Boy blir en sektionsförening.

\begin{beslut}
  \item välja in 6 Game Boys i Game Boy.
  \item välja Viktor Salomonsson, Erik Broback, Tobias Gabrielii, Mikael Eriksson och Viktor Wernholm till Game Boys i Game Boy, samt vakantsätta en plats.
\end{beslut}


\subsection{Piff och Puff}
\subsubsection{Val av 0--4 Piffar}
Det finns 3 sökande:
\begin{itemize}
    \item Jonas Bohlin
    \item Sofia Reimer
    \item Ludwig Gustavsson.
\end{itemize}

\paragraph{Jonas Bohlin} är 22, f17 och har körkort.
Han vill ge tillbaka mer till sektionen, och tror att det i Piff och Puff är bra att ha någon som varit med ett tag och därför kan hjälpa med att till exempel äska.

Martin Due undrar det finns någon person som Jonas inte vill sitta med.
Jonas svarar att han inte vill sitta med Kim Jong Un, eftersom han tror att Jong Un är dålig på att äska pengar.

Samuel Martinsson undrar om det finns någon världsledare Jonas vill sitta med.
Jonas vill sitta med Stefan Löfven trotts att han kanske inte riktigt är en världsledare.
Jonas tror att Stefan är bra att ha med vid både fika och äskning.

\paragraph{Sofia Reimer} är 21, f18 och har körkort.
Hon söker eftersom det inte hänt så mycket med Piff och Puff de senaste åren, trots att hon tror att det går att göra mycket.
Hon behöver även något nytt att göra efter att hon går i pensionerar sig från Sektionsstyrelsen.
Hon vill hjälpa andra att arrangera, och tror att Piff och Puff är ett bra sätt att utnyttja den kunskap hon har samlat på sig.

Joseph Löfving säger att en av hennes medsökande påstås vara sektions mäktigaste sektionsaktiva, och undrar därför hur Sofia ska hantera maktbalansen.
Sofia svarar att hon förstår vem han menar, men att hon tror att de andra två sökande kan arbeta runt denne.
I annat fall gäller de bara att spräcka bubblan för den förstnämnda.
I slutändan tror hon att bra kommunikation kan göra alla glada.

\paragraph{Ludwig Gustavsson} är 22, f19 och har körkort.
Han tycker att det är kul att engagera sig och att hjälpa andra.

Albert Vesterlund säger att Ludwig har sektionens enligt vissa viktigaste post.
Han undrar därför vilka kompetenser Ludwig kan ta med sig från den posten.
Ludwig svarar att han har fått arrangera lite trots pandemin, och att han dessutom vet hur en äskning ser ut.

Sektionskassör David Winroth undrar om han vet hur mycket pengar det finns att äska i år.
Ludwig svarar att det bästa sättet att få reda på det är att skicka ett mail till Sektionskassören.

\begin{beslut}
  \item välja in 4 Piffar i Piff och Puff.
  \item välja Jonas Bohlin, Sofia Reimer och Ludwig Gustavsson till Piffar i Piff och Puff, samt vakantsätta en plats.
\end{beslut}


\subsection{Mastermottagningsansvarig}
Tobias Gabrielii säger att man som mastermottagningsansvarig sitter tillsammans med representanter från K och KfKb.
Det är lite jobb, mycket budget och man får dessutom betalt för sitt arbete.

\subsubsection{Val av 1 mastermottagningsansvarig}
Det finns 1 sökande:
\begin{itemize}
    \item Tarek Alhaskir.
\end{itemize}

\paragraph{Tarek Alhaskir} är 23, f17 och har inte körkort.
Han tycker att det är kul att arrangera med andra sektioner och att det är kul att visa Chalmers för utbytesstudenter.

Samuel Martinsson frågar vad det fetaste Masterarrangemanget han kan anordna är.
Tarek svarar att han kanske kan bjuda på bullar, och att han är för trött för att ge ett bra svar.

Karin Hult undrar om Tarek kan tänka sig att ha en eldkastare under Mastermottagningen
Tarek svarar att om han får vill han gärna använda en eldkastare, särskilt om det är något Karin brinner för.

\begin{beslut}
  \item välja Tarek Alhaskir till mastermottagningsansvarig.
\end{beslut}


\subsection{Frisörer}
\subsubsection{Val av 0--2 frisörer}
Det finns 6 sökande:
\begin{itemize}
    \item Joel Sandås
    \item Victor Enevold
    \item Axel Flordal
    \item Vincent Udén
    \item Mijo Thoresson
    \item Willem de Wilde.
\end{itemize}

\paragraph{Joel Sandås} är 20, f20 och har körkort.
Han älskar sten, framför allt en sten: Einsten.
Han vill göra Frisör till en viktig post, och vill informera Nollan om stenen för att föra vidare traditionen som som nu skapas.

\paragraph{Victor Enevold} är 20, f20 och har körkort.
Han sökte för att det var kul, men tror att det höga sökantalet indikerar att de finns andra sökande med större ambitioner.

\paragraph{Axel Flordal} är 18, tm20 och har körkort.
Axel tycker att det finns enorm utvecklingspotential med posten.
Han vill utforska vad stenen är, huruvida det är en riktig sten och andra filosofiska frågor.

\paragraph{Vincent Udén} är 20, f19 och har körkort.
Han tycker att det finns en viktig fråga på sektionen: Einsten är inte funnen.
Han säger att han är bekväm med stenar eftersom han håller på med klättring, och berg i stort sätt är stora stenar.
Han nämner även att hans farfar i FnollK, Grus\censor*{5}korn, har starka stenassociationer.

Emma Ödman undrar om Vincent börja leta efter Einsten, och vad som kommer stå i Nollmodulen om Einsten.
Vincent svarar att han letat utanför sin balkongdörr, vilket är ett bra ställe att börja.
Vad han skriver i modulen beror på om han blir vald till Frisör eller inte.
Det kan antingen bli två helsidor sidor eller betydligt mindre.

Isac Borghed undrar om Vincent kan klippa stenar.
Vincent svarar att man när man klättrar ibland får stenbitar att lossna, vilket kan räknas som klippning av berg.

\paragraph{Mijo Thoresson} är 20, f19 och har körkort.
Mijo är taggad på att hitta Einsten, ta Einsten till sektionen och arrangera en stor välkomstfest när pandemin är över.

Samuel Martinsson undrar vilka planer Mijo har för ceremonin.
Mijo svarar att han inte har planerat, utan väntar istället på inslag från Einsten.
Men togor och bål skulle enligt honom inte sitta fel.

\paragraph{Willem de Wilde} är 22, f18 och har körkort.
För att skicka ett budskap om att sektionen välkomnar alla varmt vill Willem ha två finländare som pratar mumin som Frisörer.
Med sax och kex ska de tillsammans fixa kalufs och välkomna folk oavsett dialekt.

David Winroth undrar om Willem kan ha en frisersalong på Focus.
Willem svarar något om att han kan fixa \enquote{muminkex}.

Jacob Burman undrar vad Willem känner kring stora hårda stenar.
Willem svarar att han är van vid sådana ting från sin tid i F6.

Emma Ödman undrar vilka framtidsplaner Willem har om han skulle bli en av de två första att innehava denna post.
Willem svarar att planen involverar en trimmer för Talman, för stenen och för den stabila kalufsen.
Willem nämner även att roliga aktiviteter på campus efter pandemin är Willems stora kall.

Isac Borghed var Willem tänker leta först eftersom det har varit svårt att hitta Einsten.
Willem svarar att stenar är stora; till exempel är Maskinsektionens bautasten stor.
Willem säger att man kan sätta en peruk på bautastenen, och sedan baxa stenen medans man pratar finlandssvenska för att ingen ska våga ifrågasätta vad som händer.

\begin{beslut}
  \item välja in 2 frisörer.
  \item välja Joel Sandås och Willem de Wilde till frisörer.
\end{beslut}


\section{Propositioner}
\subsection{Justering av FIF:s verksamhet}
Sektionskassör David Winroth säger att FIF just nu har ett verksamhetsår som sträcker sig från 1:a april och framåt.
Det blir jobbigt för sektionen att ha 3 bokföringsår, så Sektionsstyrelsen föreslår därför att invalet av FIF ska flyttas till LP2, och därmed sammanfalla med FnollK:s och FARM:s verksamhetsår.

\begin{beslut}
  \item bifalla propositionen.
\end{beslut}


\subsection{Omskapandet av Game Boy som sektionsförening}
Sekreterare i Styret Joseph Löfving säger att Game Boy de senaste åren blivit mer aktiv.
Bland annat är det nu vanligt förekommande att de säljer mat på brädspelskvällar.
Som sektförening kan de få en egen budgetpost för detta, vilket innebär att de slipper äska pengar inför varje arrangemang.

Karin Hult undrar om Game Boy borde ha inval samtidigt som FIF av samma anledning som ledde till att FIF:s inval just flyttades.
David Winroth svarar att det inte behövs eftersom Game Boy till skillnad från FIF inte kommer att sköta sin egen ekonomi.

Game Boy Mikael Eriksson säger att Game Boy under året arbetat för denna ändring, och att intresset för Game Boy kommer bestå.

\begin{beslut}
  \item bifalla propositionen.
\end{beslut}


\subsection{Uppstädning av styrdokument}
Sekreterare i Styret Joseph Löfving säger att flera saker i styrdokumenten är felaktiga, men att det mest handlar om småsaker.

\begin{beslut}
  \item bifalla propositionen.
\end{beslut}


\subsection{Miljö- och hållbarhetsåtgärder}
Sekreterare i Styret Joseph Löfving berättar att Sektionsstyrelsen tidigare skickat ut un enkät om vegetariskt kost på sektionsarrangemang.
Sektionens medlemmar hade många åsikter, vilket lett till tagit fram denna proposition.

Erik Levin yrkar på att lägga till att fågel som köps in borde ha Svensk Fågel-märkning.
Han säger att det annars kan användas soja från Sydamerika för att mata fåglar som man sedan äter, trots att fåglarna är svenska.
Fläsk- och nötkött har å andra sidan oftast inte det problemet.

Martin Due anser i egenskap av teknolog att formuleringen om likvärdigt alternativt för plastprodukter är otydlig.
Han yrkar därför på att stryka punkten om plastprodukter.

Selma Moqvist undrar vilka arrangemang som täcks.
Joseph svarar att alla arrangemang där sektionens pengar används omfattas av policyn.

Felix Augustsson anser i egenskap av teknolog även han att formuleringen om likvärdigt alternativt för plastprodukter är otydlig.
Han yrkar dock istället på att stryka undantaget för plastinköp om likvärdigt alternativ inte finns.

Tarek Alhaskir tycker att det är för sent att bestämma om frågan klockan 01.30 på kvällen eftersom beslutet påverkar många som inte är på mötet.
Yrkar yrkar därmed på bordläggning av frågan till nästkommande möte.

David Winroth anser i egenskap av teknolog att information om propositionen står med i kallelsen, och att man då om man har starka åsikter kan man vara på mötet. Dessutom tycker han att detta verka vara fallet eftersom flera sektionsmedlemmar uttryckt åsikter i frågan.

Martin Due undrar hur policyn kommer följas upp.
Kommer det finnas några påföljder om man bryter mot policyn.
Joseph Löfving svarar att det likt andra regelbrott kan leda till missförtroendeförklaring om övertrampet anses medvetet och grovt.

Selma tror inte att några fysiska arrangemang kommer ske innan nästa möte, och anser därför att det är lika bra att skjuta upp frågan.

Skyddsombud Emelie Sjögren säger att Sektionsstyrelsen inte vill belasta nästa Sektionsstyrelse med implementationen av besluten, och att de därför tycker att beslut i frågan ska tas idag.

\begin{beslut}
  \item avslå bordläggning av frågan.
  \item bifalla propositionen, med ändringsyrkandet:
  \begin{itemize}
    \item Köttprodukter som serveras ska vara av svenskt ursprung, och ha Svensk Fågel-märkning om tillämpningsbart.
  \end{itemize}
\end{beslut}


\section{Motioner}
\subsection{Tilltal av talmanspresidiet}
Motionären Emelie Sjögren presenterar motionen. %TODO: Link
Sekreterare i Styret Joseph Löfving säger att Sektionsstyrelsen yrkar på bifall eftersom de håller med motionärerna helt i frågan.

\begin{beslut}
  \item bifalla motionen i sin helhet.
\end{beslut}


\subsection{Sektionsfärg}
Motionen faller eftersom ingen lyfter den.

\section{Övriga frågor}
Inga övriga frågor föreligger.

\section{Dumvästutdelning}
Nomineringarna är som följer:
\begin{itemize}
  \item \textbf{Hugo Rådegård} nomineras av Samuel Martinsson.\\
  Hugo stod och stekte prinskorv i sitt kök när en vän dök upp i entrén och ville bli insläppt.
  Hugo inser när de kommer tillbaka till köket att han låst in sina nycklar tillsammans med korven som fortfarande steks.
  Ingen av Hugos grannar är hemma, och kundcentret är stängt, så Hugo blir tvungen att kontakta Securitas.
  Under tiden han väntar på Securitas fylls trapphuset av rökdoft.
  När han till slut blir insläppt i sitt kök inser han att det bara är ena halvan av korvarna som bränts, och delar därmed de andra halvorna med sin vän.
  
  \item \textbf{Sektionsmötet LP2} nomineras av Tobias Wallström.\\
  Sektionsmötet beslutade under läsperiod 2 att en ny funktionärspost, som helt saknar värde, skulle skapas.
  Inte nog med att sektionen redan har meningslösa inval; detta inval föll dessutom på det redan för långa sektionsmötet i läsperiod 3.

  \begin{ofraga}
    David Winroth yrkar på att Talman Ruben Seyer ska bära västen i sektionsmötets ställe denna nominering vinner.
    Joseph Löfving informerar om att detta är praxis från Operation Thunder, vilket Vice talman Martin Due håller med om.
  \end{ofraga}
  
  \item \textbf{Olof Cronqvist} nomineras av Joseph Löfving.\\
  Under kvällens gång kunde den läskunnige sektionsmedlemmen se på Focs News att Focumateriet beklagade sig väldigt över hur långt mötet skulle bli. Ändå väljer Olof Cronquist, vice ordförande i Focumateriet, att yrka på att göra en spontan reglementesändring och drog således ut på mötet själv.
  
  \item \textbf{Ruben Seyer} nomineras av David Winroth.\\
  Under kvällen valdes en av rösträknarna in efter sluten votering som denne själv granskade.
  Talman Ruben Seyer reagerade inte över huvud taget på detta.

\end{itemize}

\begin{beslut}
  \item Hugo Rådegård var dummast.
\end{beslut}
Efter noggrann jämförelse med föregående dumvästinnehavare Erik Johansson går mötet till beslut.
\begin{beslut}
  \item ge Hugo Rådegård Dumvästen.
\end{beslut}

\section{Mötets avslutande}
Mötet avslutas 02.09 av Talman Ruben Seyer.

\clearpage
\section*{Signaturer}
\label{sec:sig}
\addcontentsline{toc}{section}{\nameref{sec:sig}}
En signatur på den här sidan avser hela sektionsmötesprotokollet \themote. Det sker elektroniskt via digitalt ID. Kontakta talmanspresidiets sekreterare på \href{mailto:talman.sekr@ftek.se}{\texttt{talman.sekr@ftek.se}} vid kontroll eller övriga frågor. 

\vspace{4cm}
\begin{center}
  \makebox[12cm][c]{
    \parbox{12cm}{
      Detta är endast en kopia för granskning.
      %\signatur{Ruben Seyer}{Talman} \hfill
      %\signatur{Felix Augustsson}{Sekreterare}
      %\newline
      %\signatur{Tarek Alhaskir}{Justerare} \hfill
      %\signatur{Mikael Eriksson}{Justerare}
      %\vspace{1.5em}
    }
  }
\end{center}

\clearpage
\begin{bilagor}
  \bilaga{Beslut att fastställa}{beslut.pdf}
  
  \bilaga{Verksamhetsberättelse FARM 2020}{farm/vb.pdf}
  \bilaga{Revisionsberättelse FARM 2020}{farm/rb.pdf}
  \bilaga{Verksamhetsberättelse FnollK 2020}{fnollk/vb.pdf}
  \bilaga{Revisionsberättelse FnollK 2020}{fnollk/rb.pdf}
  
  \bilaga{Verksamhetsplan FARM 2021}{farm/vp.pdf}
  \bilaga{Verksamhetsplan FnollK 2021}{fnollk/vp.pdf}
  
  \bilaga{Proposition om Justering av FIF:s verksamhet}{prop/a.pdf}
  \bilaga{Proposition om Omskapandet av Game Boy som sektionsförening}{prop/b.pdf}
  \bilaga{Proposition om Uppstädning av styrdokument}{prop/c.pdf}
  \bilaga{Proposition om Miljö- och hållbarhetsåtgärder}{prop/d.pdf}
  \bilaga{Proposition om Budgetändring för köksrenovering och riktade satsningar}{prop/e.pdf}
  
  \bilaga{Motion om Tilltal av talmanspresidiet}{motion/a.pdf}
  \bilaga{Motionssvar till Tilltal av talmanspresidiet}{motion/asvar.pdf}
  \bilaga{Motion om Sektionsfärg}{motion/b.pdf}
  \bilaga{Motionssvar till Sektionsfärg}{motion/bsvar.pdf}
  \bilaga{Motion om Vice i FIF}{motion/c.pdf}
  \bilaga{Motionssvar till Vice i FIF}{motion/csvar.pdf}
\end{bilagor}

\end{document}