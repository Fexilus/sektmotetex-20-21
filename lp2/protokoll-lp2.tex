\documentclass[hidelinks]{sektionsmote}
\usepackage{digsig}
\usepackage{csquotes}

\title{Protokoll fört vid sektionsmöte}
\shorttitle{Sektionsmötesprotokoll}
\motesdag{10}
\motesmanad{12}
\motesar{2020}
\motesnr{02}
\motestid{17.19}
\motesplats{Zoom}
\verksamhetsar{20/21}

\makeheader

\begin{document}
\maketitle

\section{Mötets öppnande}
Mötet öppnas \tid av Fysikteknologsektionens Talman Ruben Seyer.


\section{Mötets behörighet och beslutsförighet}
Talman Ruben Seyer meddelar att mötet är utlyst korrekt och i tid.
Han frågar mötet om det anses vara behörigt och beslutsförigt.
\begin{beslut}
    \item anse mötet behörigt och beslutsförigt enligt stadgarna.
\end{beslut}


\section{Val av justerare}
Alexandru Golic och Axel Flordal nomineras till justerare.
\begin{beslut}
  \item välja Alexandru Golic och Axel Flordal till justerare.
\end{beslut}


\section{Val av rösträknare}
Alexandru Golic och Karin Hult nomineras till rösträknare.
\begin{beslut}
  \item välja Alexandru Golic och Karin Hult till rösträknare.
\end{beslut}


\section{Fastställande av föredragningslista}
Talman Ruben Seyer har glömt att ta med de bordlagda ansvarsfriheterna från förregående möte.
Han ber därför snällt mötet att vara behjälpliga, och yrkar därefter på att lägga till en sådan punkt som punkt §11.
\begin{beslut}
  \item lägga till punkt §11 Bordlagda ansvarsfriheter och ändra numreringen på efterkommande punkter.
  \item fastställa föredragningslistan i övrigt.
\end{beslut}


\section{Adjungeringar}
Inga adjungeringar föreligger.


\section{Föregående mötesprotokoll}
Sekreterare Felix Augustsson informerar om att det föregående protokollet justerades och anslogs i tid i enlighet med stadgan.
\begin{beslut}
    \item lägga föregående mötesprotokoll till handlingarna. 
\end{beslut}


\section{Uppföljning av beslut}
Inga beslut föreligger.


\section{Fastställande av beslut}
\subsection{Avsägelser}
\begin{beslut}
  \item fastställa sektionsstyrelsens beslut att entlediga:
  \begin{itemize}
      \item Oskar Molin som Nätmakare i Spidera.
  \end{itemize}
\end{beslut}


\section{Meddelanden}
\subsection{Sektionsstyrelsen}
Sektionsordförande Emelie Björkman säger att det har varit en lite annorlunda läsperiod.
Sektionsmedlemmar kan skriva till Styret om de har idéer kring aktiviteter i läsperiod 3.
Det finns även en enkät gällande sektionens kommande investeringar.
Tidigare i år har sektionen investerat pengar i både flipper och brädspel.
Sektionen har dock mer pengar att investera, och detta skulle kunna läggas på till exempel arkadspel, F-spexets överlevnad, FIF-utrustning, F6 ljudutrustning till Focus, SNF:s litteratur på Focus.

Sekreterare i Styret Joseph Löfving informerar om att Styret kommer underlätta för gemene sektionsmedlem att hålla koll på deras beslut.
Det har skapats en maillista som man kan skriva upp sig på för att få information om när styretmöteprotokoll läggs upp.
Mail-utskicken kommer även innehålla korta sammanfattningar av mötena.

Skyddsombud Emelie Sjögren säger att hon varit på skyddsrond med \enquote{några tjommar från högskolan}.
På skyddsronden gicks det mest igenom frågor som att byta lampor som inte fungerar, men en del av informationen var intressant.
GD kommer fortsätta att vara kallt, men ventilationen kommer dras ned för att istället anpassas till 50 personer under pandemin.
Detta skulle potentiellt kunna göra GD temporärt varmare.
Dessutom kommer högskolan göra om i FB, flytta upp eluttagen i Övergången ovanför soffkanterna och byta ut en del utrustning i KINSLS.

Sektionskassör David Winroth delger att första kvartalets resultat ser bra ut trotts pandemin.
Därför uppmanas teknologer att skicka in teknologäskningar till styret.

\subsection{Kårledningen}
Sektionens Kårledningskontakt Gabriel Aspegrén kan man komma till med frågor om kåren, hur man kontaktar högskolan och andra liknande frågor.
Han berättar att Student Voice har stängt, och kårledningen fick in många svar och har därmed ett bra underlag för vad de ska jobba med under våren.
Han informerar även om att Chalmers under hösten haft mer medianärvaro.
De har nappat på att uppmärksamma studenthälsan under pandemin, vilket han ser som positivt.
Han meddelar även att det skickats ut en medlemsundersökning till alla teknologer.
Den ligger till grund för hur Kårledningen arbetar, och är därmed viktig att svara på.
Han säger även att intervjuerna av nästa läsårs kårledning snart påbörjas.
Om man är intresserad av att söka finns information om kårledningen och intervjuer med tidigare ledamöter på Facebook.
Till sist vill han även påminna om att årets julfirande kommer ske digitalt.
På lördag kommer det vara ett digitalt julbord, till vilken det gått att köpa jultalrik.
Men även om man inte införskaffat jultalrik går det att följa den tillhörande julbords-streamen.
Även kårens Luciafirande kommer ske digitalt via en livestream.

\subsection{Kårfullmäktige (FuM)}
Kårfullmäktigeledamot Sara Kraamer (f17) vill informera snabbt om Kårfulmäktiges verksamhet.
Hon informerar om att FuM består av ett presidie och 35 ledamöter.
Ledamöterna väljs in under april i en omröstning som är öppen för alla teknologer.
De invalda går under det kommande året på Kårfullmäktigemöte 2 gånger per läsperiod.
Under mötet sker till exempel inval och större beslut om kårens verksamhet, som att se över om kåren ska skaffa en större idrottshall.
Mötet är öppet att kolla på, och information om hur man gör det samt protokoll finns på \href{https://chalmersstudentkar.se/union-council/}{kårens hemsida}.
Partier och kandidater i det kommande valet finns hos \href{https://www.facebook.com/fumval/}{Valnämnden}.

\subsection{Focumateriet}
Automatpirat Olof Cronquist säger att Focumateriet kommer bedriva oberoende journalistik på \newline\href{https://focumateriet.wordpress.com}{\texttt{focumateriet.wordpress.com}}.
På fredag kommer de även bedriva en Dominion-turnering.
Till sist presenterar han presidiets hattar som utlovades föregående möte.
Talman Ruben Seyer får en hjälm med tillhörande Trocadero och sugrör, Vice talman Martin Due får en Casino-keps och Sekreterare Felix Augustsson får en Fez med slöja.


\section{Bordlagda ansvarsfriheter}
Revisor Nils Patriksson säger att de tidigare problemen med bokföringen nu ser bra ut.
Han tillstyrker att SNF och Styret ska ansvarsfrias.
Förra gången såg allt inte helt bra ut, men nu är det fixat.
\begin{beslut}
  \item med godkännande lägga \hyperlink{bilagor/revisionsnf1920.pdf.1}{den reviderade revisionberätelsen} till handlingarna.
  \item bevilja tidigare Sektionsordförande Fredrik Meisingseth, tidigare Sektionskassör Tobias Gabrielii, tidigare Ordförande i SNF Albert Johansson och tidigare Kassör i SNF Anton Wikström ansvarsfrihet för det gångna verksamhetsåret.
\end{beslut}


\section{Fyllnadsval}
\subsection{Valberedningen}
Ordförande i Valberedningen Alexandru Golic informerar om att valberedningen valbereder folk.
I år kommer även FIF att valberedas, så det finns mer behov en tidigare av en fullsatt valberedning.

\subsubsection{Val av 1 ledamot}
Ingen söker.
\begin{beslut}
  \item vakantsätta 1 ledamot i valberedningen.
\end{beslut}


\section{Personval}
\subsection{FnollK}
\subsubsection{Val av ordförande}
Det finns två sökande:
\begin{itemize}
    \item Emma Ödman
    \item Isabella Tepp.
\end{itemize}

\paragraph{Emma Ödman} är 20, går första året på F och äger ett körkort.
Hon tycker att Mottagningen gjorde bra intryck och vill föra det vidare.\par
David Winroth undrar om Emma kan köra ett \enquote{Nollan ska vara tyst} eftersom hon sitter i en stor sal med eko.
Det kan hon.\par
Albert Westerlund undrar vad Emma tänker om att sitta i styret.
Hon tycker det låter kul.
Hon var med på ett, och det tyckte hon var bra.\par
Jacob Welander undrar vad hennes favoritvin är.
Emma svarar att det såklart är La Borne.\par
Emelie Björkman säger att Albert ställde frågan hon tänkte ställa, så hon frågar istället hur Emma föreställer sig Mottagningen 2021.
Emma hoppas att det inte längre kommer vara en pandemi, och hennes mål är att den ska vara bättre än Mottagningen 2020.\par
Axel Pantzare undrar om hon har fått körkortet hon äger i ett cornflakespaket, eller om hon faktiskt tagit det.
Hon informerar om att hon både tagit och fått körkort.\par
Infochef i FnollK Fredrik Skoglund vill veta vad Modulen fyller för funktion för FnollK.
Emma svarar att \enquote{utan modul, ingen mottagning}.\par
Richard Svensson undrar vad det sämsta under årets Mottagning var, eller kort sagt var FnollK brast.
Emma svarar att den största bristen var att de inte fixade ett vaccin.
Emma tycker även att nollbrickstillverkning i Euler inte var bra, och lovar att ingen ska behöva tillverka nollbrickor där nästa år om hon blir invald.\par
Organisatör i FnollK 2018 Jonas Bohlin säger att patetgänget undrar vad planen är när Phadderchefen misslyckas, något han påstår garanterat kommer hända.
Emma undrar vad Phadderchefer brukar vara dåliga på, och får som svar att de ofta kommer för sent och inte gör så mycket.
Emma säger att hennes lösning är att se till att välja rätt person.
Jonas vill ha en femstegsplan.
Emma svarar då att hennes fem steg är fem inval av Phadderchefer.
På så sätt har 5 phadderchefer, och har därmed alltid en reserv när det behövs.\par
Emelie Björkman undrar vad Emma ser fram emot.
Emma ser mest fram emot att träffa alla människor.
Hon vill lära känna folk och veta vad de kommer ifrån.\par
Simon Dovrén undrar om hon har ytterligare vallöften.
Emma säger att eftersom Nollan inte vet hur Mottagningen var förra året så kan hon lova att de kommer uppleva Mottagningen 2021 som den bästa Mottagningen någonsin.\par
Axel Pantzare undrar vilken den bästa puben i Göteborg är.
Emma svarar Soho; om man är fyra personer kan man sitta i en speciell soffa där.\par
Jonas Bohlin vill ställa en mer seriös ordförande-patet-fråga, och undrar därmed vad Emma tycker om relationen mellan Ordförande och Vice ordförande.
Emma tycker att den är viktigt eftersom man ska kunna avlasta sig med sin Vice ordförande.
Det ska inte vara jobbigt att be om hjälp.\par
Fredrik Skoglund säger att Emma om hon blir invald inte kommer vara i en grupp av 4, utan istället en grupp av 7.
Vad är då den bästa puben?
Emma förklarar att soffan är till för minst 4 personer, så Soho är fortfarande hennes svar.\par
Alexandru Golic vill fortsätta på samma tema, och undrar därför vart man ska gå om 200 Nollan ska gå på bar samtidigt.
Emma svarar Slottsskogen.\par
Adam Johansson undrar vad som var bäst med FIF under Mottagningen.
Emma svara sittande.\par
Organisatör i FnollK August Kälvesten undrar vilken av ledamotsposterna som är viktigast.
Emma tycker inte att man borde rangordna dem, eftersom de är bäst som helhet.\par
Fredrik Skoglund undrar vilket föremål som är viktigast: phaddergrupp, nolluppdrag, modulen eller tejp?
Emma svarar Mottagningen.

\paragraph{Isabella Tepp} är 20, går andra året på F och har körkort.
Hon söker FnollK eftersom ingen annan får ha makt.
Hon är överkommen av storhetsvansinne, och har beslutat att första steget är att ta över FnollK.
Till att börja med är Mottagningen helt inställd, vilket är det säkraste alternativet under pandemin.
Den ska dessutom heta Nollning; att den heter Mottagning är en scam.
Det ska dessutom bli mer krök, och alla ska slippa Noll-pesten.\par
David Winroth poängterar att man bara kan vara ordförande i antingen Djungelpatrullen eller FnollK.
Isabella svarar att hon är speciel, och att reglementet kan ändras för henne.\par
Organisatör i FnollK 2018 Jonas Bohlin undrar hur expansionsplanerna ser ut och vad planen för Raum 2.0 är.
Isabella undviker frågan genom att säga att Rustmästaren kommer presentera det snart, och att hon därför inte vill förekomma honom.\par
Ludwig Gustavsson undrar om Isabella inte äger ett körtkort, och vems körkort hon i sådana fall har.
Isabella svarar att hon även äger sitt körkort.\par
Richard Svensson tycker att Isabellas plan låter rimlig, och undrar om Mottagningen kommer ha lika många arr som Djungelpatrullen haft i läsperiod 2.
Isabella svarar att arren inte kommer vara för Nollan, utan för alla andra.
Det kommer vara maxade DuP:ar varje måndag, tisdag, onsdag och fredag efter att vaccinet kommit.\par
Ludvig Nordqvist undrar vad hon ska göra med pengarna hon inte kommer lägga på Mottagningen och hur mycket Slotts det räcker till.
Isabella svarar många.\par
Organisatör i FnollK August Kälvesten undrar vilken nuvarande ledamotspost i FnollK som är viktigast.
Isabella kan inga andra poster utom förtroendeposterna, så hon har ingen aning.\par
Isac Borghed undrar varför det inte kommer vara några maxade DuP:ar på torsdagar.
Isabella svarar att hon har expfys på torsdagar.\par
Ludwig Gustavsson undar hur hon ska ha plats med alla DuP:ar.
Isabella svarar att det finns en plan: eftersom Djungelpatrullen redan har en plan för att ta över sektionen kan de även ta över GU, och därmed Hilbert.\par
Ludvig Nordqvist undrar hur man kan centrera makten runt Djungelpatrullen, och hur man kan ha nyktert tänkande i vardagen.
Isabella svarar att makten centreras just nu.
Ludvig förtydligar att han vill veta hur det ska bli diktatur.
Isabella svarar att det ska bli mer tyranni åt folket.\par
Alexandru Golic tycker att Djungelpatrullen har storslagna planer, och undrar om den nya mottagningen kan vara i \textit{Slots}skogen.
\enquote{Absolut} svarar Isabella.

\begin{beslut}
  \item välja Emma Ödman till Ordförande i FnollK.
\end{beslut}

\subsubsection{Val av vice ordförande}
Det finns två sökande:
\begin{itemize}
    \item My Irenheim Jarlhed
    \item Gustav Persson.
\end{itemize}

\paragraph{My Irenheim Jarlhed} är 21, går första året på F och har körkort.
Hon söker för att Mottagningen var bra för henne.
Hon är inte från Göteborg, så när det var distansundervisning på grund av pandemin var det hennes enda chans att lära känna folk.
Hon vill därför anordna samma sak för andra.\par
Organisatör i FnollK 2018 Jonas Bohlin undrar hur hon tror att hon kommer samarbeta med nyinvalda Ordförande Emma Ödman.
My svarar att hon inte känner henne så bra sedan innan, men att hon tror att det kommer gå bra.\par
Emelie Björkman undrar vad hon ser fram emot.
My svarar att hon aldrig har gjort någonting såpass stort som FnollK, så det är häftigt.
Hon tillägger att det specifikt är hela processen med att anordna Mottagning som verkar häftig.\par
Några FnollK-pateter påstår att de inte vet vad en Vice ordförande gör.
My förklarar att en Vice ordförande avlastar sin Ordförande.\par
Infochef i FnollK Fredrik Skoglund påminner om att alla Vice ordförande döper sig efter en stjärna i universum, och undrar om My har några tankar kring vilken stjärna hon vill välja.
My svarar att hon har googlat lite, men inte bestämt sig än.\par
Emelie Björkman säger att man som Vice ordförande får ta Ordförandes plats när denne inte är närvarande.
Till expempel har Ordförande Carl Strandby har haft mycket expfys under hösten, men trotts det har Vice ordförande Mattias Arvidsson inte dykt upp.
My tycker att det är konstigt att Mattias inte varit där, och tillägger att det styretmöte hon varit på var spännande.\par
Fredrik Skoglund påstår att det ofta händer att varken Ordförande eller Vice ordförande går på styretmöte, så han undrar i FnollK som är Vice vice ordförande.
My svarar att hon tycker att det borde vara den som är mest taggad.\par
Jonas Bohlin undrar hur My ska göra en stadskupp för att hålla i trådarna bakom kulisserna.
My svara att hon har planerat mycket, och tänker slå till när Emma har expfys.

\paragraph{Gustav Persson} är 21, går andra året på F (vilket han poängterar är större än första året) och har körkort för både motorcykel och bil.
Han söker för att ska stärka sin makt.
Han driver att det ska ske en ombygnation av Nollrummet till nytt Raum, och att det är Nollan som ska bygga det.
\begin{ofraga}
  Alexandru Golic vill sätta en tidsbegränsning på 3 minuter på nuvarande utfrågningar.
  Talman Ruben Seyer utrycker att presidiet inte tycker att detta borde tas upp såpass sent i utfrågningsprocessen.
  Emelie Björkman håller med presidiet om att man inte bör införa en tidsbegränsning efter att en person redan blivit utfrågad.
  Jonas Bohlin tillägger att ordningsfrågan antagligen är längre än utfrågningen.
  Axel Prebensen kontrar med att det skulle varit snabbare klart om inte Jonas sagt någonting.
  \begin{beslut}
    \item införa en tidsbegränsning på 3 minuter för utfrågningen.
  \end{beslut}
\end{ofraga}
Ordförande Carl Strandby säger att en Vice ordförande måste vara neutral för att hålla sams, och därmed inte kan vara kontroversiell.
Så varför är Gustavs smeknamn då Cop Killer?
Gustav svarar att det är klassificerat; man måste aspa Djungelpatrullen för att få reda på det.
Han tillägger dessutom att han brukar städa upp efter Isabella Tepp.\par
Ludvig Nordqvist säger att han älskar ordningsfrågor, och tycker att Gustav har bra idéer.
Han undrar hur man stoppar tiden så att det aldrig blir sämre.
Gustav svarar att man kan sitta i föreningar för evigt, och dessutom stänga ner alla andra föreningar.
Till exempel kan man ha en Djungelpatrull per decennium som eftersträvar det decenniets värderingar och tankesätt.\par
Infochef i FnollK Fredrik Skoglund påminner om att Ordförande i FnollK 2021 kommer vara Emma Ödman, och undrar därmed hur Gustav ska manipulera henne och få henne att gå över till den mörka sidan.
Gustav svarar att Isabella inte förlorade invalet till Ordförande; det var valfusk och felräkning under invalet, och därmed kommer Isabella vara Ordförande i FnollK 2021.\par
Ludvig Nordqvist undrar hur Gustav ska få Isabella invald.
Gustav svarar att det var felräkning av röster, så det kommer lösa sig själv, och är därmed inte en riktig fråga.

\begin{beslut}
  \item välja My Irenheim Jarlhed till Vice ordförande i FnollK.
\end{beslut}

\subsubsection{Val av kassör}
Det finns två sökande:
\begin{itemize}
    \item Erik Brusewitz
    \item Lowe Fareld Krågen.
\end{itemize}

\paragraph{Erik Brusewitz} är 22, går tredje året på F och har inte körkort.
Han söker eftersom det verkar asgött att arra Mottagning.
Tidigare har han inte haft tid på grund av arbete, men har nu sagt upp sig och har även prövat på att puffa.
Puffningen tyckte han var rolig, så han har nu bestämt sig för att söka.\par
David Winroth ponerar att budget ökar med 20k, och undrar vad Erik då lägger överskottet på.
Erik vill göra ljudsystemet i Nollraum maffigare.
Men han inser att det är inte är en del av Mottagningen, och ändrar därmed sitt svar till att han skulle hyrt en helikopter att köra runt Nollan i.\par
Albert Vesterlund säger att Erik inte kan heta Bruse om han sitter som Kassör, och undrar därför om han kan byta smeknamn till \$urkål eller \$äd.
Erik säger att surkål är den äckligaste maten som börjar på S, och att säd är den näst värsta maten som börjar på S.
Så han tycker att namnförslagen var lite oturliga, men byta namn kan han tänka sig.\par
Organisatör i FnollK August Kälvesten säger att Kassören har inflytande över budget, och undrar varför Erik vill ändra på tejpbudgeten mest.
Erik stammar lite, och säger till slut att han tycker tvärt om;
LoB-tejp är för dyrt, så kommande år blir det silvertejp från Clas Ohlson.\par
Kassör Elsa Danielsson ponerar att Erik ska göra att budget, det helt plötsligt dyker upp en pandemi.
Vad gör Erik då?
Erik svarar att han vill lägga pengarna på feta tröjor till aspar, eller införskaffa Cyberpunk till alla sektionsmedlemmar.\par
Axel Flordal vill åter ta upp frågan om smeknamn, och undrar hur Erik ställer sig till \$tångkorv och \$mulpaj.
Erik säger att man får tänka som en liten Nollan, och då är \$tångkorv är väldigt långt och svårt, så det är svårt.
Han har heller inte ätit stångkorv.
\$mulpaj tycker han man kan associera med pajas vilket inte är önskvärt, så det namnet hamnar långt ner på listan.\par
Isac Borghed säger att det är bra att vara med i 34:an, men att Erik tyvärr inte heter Fredrik.
Han undrar därför om Erik kan byta namn till Fredrik.
Erik lovar att gå till Skatteverket imorgon och byta namn till Fredrik om han blir invald.\par
Organisatör i FnollK 2018 Jonas Bohlin säger att en tidigare Kassör som han har på tråden ponerar att Phadderchefen köper 12 liter mjölk istället för 2 liter mjölk, och sedan gör det en gång till.
Vad gör Erik då?
Erik svarar mjölkhäv.
Eller så säljer man mjölk till Dragos och fakturerar honom.\par
Emelie Björkman undrar vad Erik ser fram emot.
Erik tror att det kommer vara kul att arrangera Mottagning och träffa Nollan.\par
Tobias Wallström undrar vad Erik lägger 20 tusen kronor på privat: en klocka, kaviar, skatteparadis eller politikermutor?
Erik svarar att politikermutor är det självklara valet, för politiker kan höja ens budget.\par
Joseph Löfving ponerar att Erik ska skriva sitt kandidatarbete med Joseph, och undrar då hur ska han då balansera tiden mellan FnollK och kandidatarbetet?
Erik säger att man får vara snabb med kandidatarbetet.
Arbetet med FnollK börjar ju först i januari, så om man gör färdigt kandidatarbetet över jullovet borde allt lösa sig.\par
Ludwig Gustavsson har hört att det är prat om att byta namn, men tycker att man om man ska byta namn måste ha någon att byta med.
Med vilken Fredrik kommer Erik byta namn, och varför är det Fredrik Skoglund?
Erik tycker att Fredrik Skoglund har använt klart sitt namn eftersom man bara behöver namnet Fredrik när man sitter i FnollK.
Erik tror dessutom att Fredrik är skön och kommer dela med sig av det goda.\par
Jonas Bohlin delar sin skärm, där det syns ett gäng Fnollk-pateter i ett separat zoom-samtal.
Han undrar vilka i det zoom-mötet som är FnollK-pateter, och be även Erik att säga vilka som är före detta Kassörer och vad de hade för smeknamn.
Erik lyckas lista ut att de flesta är pateter, och pekar även ut pateterna \$nigel och Mirriam som Kassörs-pateter.
Jonas Bohlin säger därefter att pateterna undrar om det kommer finnas streck-kexchoklad, och vad priset på sådana kommer vara.
Erik svarar att det kommer finnas kexchoklad bara för Jonas.\par
Ordförande Carl Strandby säger att han har haft en bra Kassör som har överseende med att han inte har koll på ekonomin.
Så han undrar hur mycket av ekonomiarbetet Erik kommer lägga över på Emma?
Erik svarar att Emma kommer ha fullt upp med att bossa runt FnollK, så Erik kommer göra sitt arbete själv.
Men han kommer diskutera de relevanta ekonomiska frågorna med Emma.\par
Jonas Bohlin undrar återigen vad kexchokladen kommer kosta i svenska kronor.
Erik svarar att om man säljer för 30:- till alla andra kan man sälja för 5:- till Jonas.\par
Filip Rydin säger att Nollan-tröjor är en stor utgift, och undrar därför vem Erik vill ska vara huvudsponsor?
Erik säger att han kanske har bränt broarna med Consat, och tycker därför att man borde fråga CERN om de vill sponsra tröjorna.\par
Fredrik Skoglund säger att en del av modulen är sponsrad av Gibraltar pizzeria, men att Infochefen inte brukar våga gå och ta betalt de 500 kronor som det kostar.
Kan Erik tänka sig att gå dit och samla in pengarna?
Erik tror att gubbarna på pizzerian är rädda för honom, och att det därför går bra.\par
Martin Due undrar om Erik kan lova att FnollK kommer köpa phadder-tröjor till nästa Mottagning.
Erik säger att det är farligt att lova saker, men att han kan han lova att skaffa phadder-tröjor.

\paragraph{Lowe Fareld Krågen} är 20, går andra året på TM och har samt äger körkort.
Han säger att det är uppenbart varför han söker: det är dags att Djungelpatrullen tar över.
Lowe kommer investera i ett off-campus Raum genom att gå in på Blocket och köpa Sveriges längsta husvagn.
Han kommer lägga hela Mottagningsbudgeten på det inköpet.\par
Ludvig Nordqvist säger att man som Kassör måste räkna ut hur mycket Slotts varje Nollan får om man skrotar Mottagningen.
Lowe svarar att Nollan inte ska få någon Slotts, utan den ska istället gå till sittande och pateter.\par
Tarek Alhaskir säger att den andra sökande har lovat att köpa phadder-tröjor, och undrar därför vad Lowe kan erbjuda.
Lowe svarar att Slotsen pris kommer att sänkas med 2/5.\par
Didrik Palmqvist säger att altruism är bra, att pengar är lycka, och att det ska vara mer sprit åt 2:an till 5:an.
Han undrar därför om det blir det mer sprit med Lowe?
Lowe svarar ja.

\begin{beslut}
  \item välja Erik Brusewitz till Kassör i FnollK.
\end{beslut}
Tarek Alhaskir önskar föra till protokollet att Erik Brusewitz lovat att gå till Skatteverket och byta namn till Fredrik imorgon.\par
Martin Due önskar föra till protokollet att Erik Brusewitz lovat att införskaffa phadder-tröjor till Mottagningen 2021.

\subsubsection{Val av 0--4 ledamöter}
\hyperlink{bilagor/nomfnollk.pdf.1}{Valberedningens gruppnominering} till FnollK väljs in automatiskt i enlighet med reglementet då de nominerade förtroendeposterna valts in.
Valberedningen läser upp gruppens nominering.

\begin{ofraga}
  Vice talman Martin Due yrkar på att ajournera mötet till 20.50 för matpaus.\par
  Gustav Hallberg yrkar på att ajournera mötet i 5 minuter för att man bara behöver hämta maten och kan äta den under mötet.\par
  Sekreterare Felix Augustsson vill ha en matpaus där man hinner äta, eftersom ha inte kan föra protokoll och äta samtidigt.\par
  Tobias Wallström frågar hur lång tid Felix tror sig behöva.
  Felix svarar att han helst vill ha så lång paus som Martin yrkade på, men att han behöver åtminstone 15 minuter.\par
  Elias Stenhede yrkar på att ajournera mötet i 16 minuter.
  \begin{beslut}
    \item ajournera mötet i 16 minuter till klockan 20.33.
  \end{beslut}
\end{ofraga}

\subsection{FARM}
\subsubsection{Val av ordförande}
Det finns en sökande:
\begin{itemize}
    \item Eric Carlsson.
\end{itemize}

\paragraph{Eric Carlsson} är 22, går fjärde året på F och har körkort.
Han säger att det FARM gör är viktigt när man senare i utbildningen går ut eller letar efter exjobb.
Han hoppas och tror att FARM-mässan blir av nästa år.
Han vill dessutom skaffa nya företagskontakter, samt skapa kanaler för att skapa nya kontakter.
Ett sätt att göra det är att kontakta andra högskolors arbetsmarknadsgrupper eller att använda nya kanaler som LinkedIn.
Han vill även att FARM gör mer PR så att det inte blir få sökande nästa år, och att FARM ska arbeta mer mot sektionsmedlemmar.
Han berättar vidare att han aspade FARM, men inte sökte eftersom det inte går att bedriva verksamhet med bara två personer.
Nu har det ändrat sig, så därför söker han.\par
Informationsansvarig Albert Vesterlund säger att man får sitta i Styret som Ordförande, och undrar vad Eric tycker om det.
Eric tror att det kan gå bra, han har en del erfarenhet av föreningar på sektionen på han suttit i bland annat F6.
Han har även suttit i kommitté centralt på kåren i 1,5 år.\par
Ordförande Richard Svensson påstår att Programansvariga för F Jonathan Weidow säger att Ordförande i FARM är den viktigaste posten på sektionen.
Eric svarar \enquote{In Weidow we trust}.\par
Eventansvarig i FARM Alexander Malmquist påminner om att Djungelpatrullen var för altruism, och undrar därför om Eric kan tänka sig något liknande.
Eric svarar att eftersom FARM driver in pengar till sektionen är de redan tillräckligt altruistiska.\par
Richard Svensson undrar vilket Erics favoritföretag är.
Eric svarar att det beror på infallsvinkeln, men han har hört att Systembolaget är rätt bra.

\begin{beslut}
  \item välja Eric Carlsson till ordförande i FARM.
\end{beslut}

\subsubsection{Val av vice ordförande}
Det finns en sökande:
\begin{itemize}
    \item Hugo Spencer.
\end{itemize}

\paragraph{Hugo Spencer} är 17, går första året på F och har inte körkort förutom till symaskin.
Han söker eftersom FARM:s arbete är väldigt viktigt för att Fysikteknologer ska ha bra möjligheter att hitta exjobb och möjligheter efter det.
Han tror det kommer bli kul att sitta tillsammans med Eric och de andra eventuellt sökande.\par
Albert Vesterlund säger att man som Vice ordförande behöver ta över när Ordförande inte kan göra uppgifter, till exempel om Ordförande har Corona och är på sjukhus.
Hugo svarar att han kan tänka sig att göra det.
Han var med på ett styretmöte och kan tänka sig att göra det igen.
Han anser även att posterna i FARM är lite mindre fasta, så det är lätt att ta över.\par
Samuel Martinsson har hört att Hugo har 3 symaskinskörkort, och undrar hur det kan hjälpa honom som Vice ordförande.
Hugo svarar att symaskinskörkort är en viktig kompetens, men har inte tänkt så mycket på hur det kan hjälpa honom i rollen som Vice ordförande.\par
Emelie Björkman undrar om Hugo har ett drömföretag att jobba med, eller något speciellt projekt han vill dra i.
Hugo vill höra vilka företag studenterna vill prata med, och uppmanar folk att kontakta honom om han blir invald.
Han har svårt att välja ett specifikt företag eftersom han är entusiastisk kring många.\par
PR-ansvarig i FARM Elias Stenhede Johansson undrar vilket företag som Hugo inte vill ha på Nollans tröjor.
Hugo tycker att det enda kriteriet är att företaget ska vara relevant.
McDonalds skulle till exempel behöva betala dyrt för att få sponsra tröjorna.\par
Didrik Palmqvist säger att FARM har många bra kontakter och undrar hur företagen kan sponsra annan verksamhet på sektionen, mer specifikt att Djungelpatrullen.
Hugo svarar att om man hittar rätt företag så kan det finnas möjligheter att hitta samarbeten via FARM.

\begin{beslut}
  \item välja Hugo Spencer till Vice ordförande i FARM.
\end{beslut}

\subsubsection{Val av kassör}
Det finns en sökande:
\begin{itemize}
    \item Samuel Martinsson.
\end{itemize}

\paragraph{Samuel Martinsson} är 19, går första året på TM och har ett körkort som är hans eget.
Han söker för att FARM:s arbete är viktigt, och han tror att han kan hjälpa till att få verksamheten att fortgå så bra som möjligt.\par
Kassör Filip Rydin säger att Kassören är ostansvarig, och undrar om den sökande har ambitioner gällande området.
Samuel tycker att ost är viktigt, och kommer såklart lägga en stor del av tiden på att säkerställa ostbehovet.\par
Ordförande Richard Svensson säger att FARM tycker om att tjäna mycket pengar, och undrar vad den sökande har tjänat på en månad.
Samuel svarar cirka 16 tusen, men att han som Kassör kommer ha en ambition att tjäna mer per månad.\par
Fredrik Skoglund påminner att nyinvalda Kassören i FnollK Erik Brusewitz har lovat att phadder-tröjor kommer införskaffas.
Han undrar därför om det är realistiskt att skaffa tröjor till phaddrar.
Samuel tycker att det borde vara möjligt, och om det nu redan är utlovat så gäller det att hålla löftet.\par
Emelie Björkman undrar hur Samuel ska balansera SNF och FARM, eftersom han bara har suttit en av tre läsperioder av sitt år i SNF.
Samuel säger att han får klara det, eftersom posten i FARM behöver fyllas.\par
Veckobladerist Mikael Eriksson påstår att man som Årskursrepresentant tar på sig saker som ingen annan vill göra, och undrar därför om Samuel kommer kunna fortsätta göra det.
Samuel tror att han kommer kunna fortsätta som innan.\par
Vice ordförande Valter Schütz vill att Samuel ska rita en eurosymbol.
Efter lite teknikkrångel så får Samuel kontroll över skärmen.
Han grubblar ett tag över hur en eurosymbol ser ut, och ritar till slut en krusidull som man skulle kunna tolka som en eurosymbol.\par
Adam Johansson säger att den skånska dialekten borde höras mer, och undrar därför om Samuel kan bekräfta att Adam inte pratar skånska, vilket Samuel gör motvilligt.
Adam undrar även vad Samuels skånska bakgrund kan bidra med.
Samuel svarar att det sker mycket forskning i Skåne, och att Skåne har närhet till Europa.
Båda dessa saker tror Samuel ger honom en fördel.\par
Sektionskassör David Winroth säger att FARM vanligtvis drar in pengar, och att sektionen lätt går plus.
Han undrar därför vad Samuel tycker är en bra långsiktig investering för sektionen.
Samuel svarar att en investering i Focus inte kan vara fel.\par
Tobias Wallström säger att kassörer måste kunna pengar, och undrar därför vilken sedel i cirkulation som har högst värde.
Samuel har svårt att svara, för så bra koll har han inte på pengar.
Han tillägger att han förhoppningsvis hinner lära sig det innan han går på.\par
Erik Brusewitz undrar vilken byggarbetsplats Samuel sitter vid.
Samuel svarar att han sitter bredvid ett kravallstaket på Olofshöjd.\par
Tobias avslöjar att 1000 Schweiziska franc är den sedel i cirkulation med högst värde.\par
Filip Rydin undrar om Samuel har erfarenhet av bokföring.
Samuel svarar att han inte har det.

\begin{beslut}
  \item välja Samuel Martinsson till kassör i FARM.
\end{beslut}

\subsubsection{Val av 0--5 ledamöter}
Eftersom \hyperlink{bilagor/nomfarm-avslag.pdf.1}{Styret har valt att avslå} \hyperlink{bilagor/nomfarm.pdf.1}{valberedningens gruppnominering} går invalet av ledamöter till personinval.
Det finns en sökande:
\begin{itemize}
    \item Amandus Reimer.
\end{itemize}

\paragraph{Amandus Reimer} är 22, går andra året på F och har inte körkort.
Han söker FARM eftersom han tror att det är kul, viktig och har verklighetsförankring.
Han tycker om att FARM inte bara gör roliga saker utan även seriösa uppdrag.
Han är intresserad av att hålla på med PR.\par
Adam Johansson säger att Amandus har gjort succé med en julkalender med FIF, och undrar om det kommer fortsätta vara bra PR hos båda föreningar om Amandus blir invald i FARM.
Det tror Amandus.\par
Jacob Welander undrar vilket ölmärke Amandus skulle vilja vara sponsrad av.
Amandus svarar Elk Brew.\par
Albert Vesterlund undrar vad Amandus bästa tips för att göra bra PR är.
Amandus svarar att man ska göra något som får en själv att skratta eller le.\par
Adam Johansson är intresserad av Amandus lampa.
Amandus förklarar att det är en Philips Hue från Ikea, och att den har en massa färger som man kan ratta fram.
Han byter sedan färg på lampan från blå till turkos.\par
Vice ordförande Valter Schütz undrar hur Amandus kommer göra bra PR för FARM i syfte att få fler sökande.
Amandus tycker att man behöver visa vad man kan göra i FARM utan att ha ett internship eller liknande, och tona ner hur seriöst FARM är.
\paragraph{}
David Winroth yrkar på att välja in 3 ledamöter i FARM.

\begin{beslut}
  \item välja in 3 ledamöter till FARM.
  \item välja Amandus Reimer till ledamot i FARM och vakantsätta de resterande två posterna.
\end{beslut}


\section{Propositioner}
\subsection{Skapande av jämlikhetsråd}
Skyddsombud Emelie Sjögren ger bakgrunden att F är en av de få sektioner som inte har någon form av jämlikhetsgrupp.
\hyperlink{bilagor/propjaemf.pdf.1}{Förslaget} är därför att vi på F ska ha ett råd för att integrera jämlikhetsarbetet på sektionen med föreningarna, där SAMO agerar mötesordförande.
Sekreterare i Styret Joseph Löfving tillägger att det kommer vara 2-3 oberoende ledamöter i rådet.\par
Alexandru Golic undrar om de oberoende ledamöterna kommer väljas in under detta möte, men Joseph svarar att det invalet kommer ske i läsperiod 4.\par
Tarek Alhaskir undrar det inte kommer bli likt situationen med oberoende SAMO där ingen vill sitta på posten.
Emelie svarar att det inte kommer jobba med samma saker som oberoende SAMO.
Dessutom blir inte rådet tomt bara för att det inte finns oberoende ledamöter ett visst år.\par
Alexandru undrar vilka befogenheter rådet kommer ha.
Emelie svarar att rådet inte kommer ha någon makt, men att personerna i rådet kan förankra sina åsikter hos SAMO som sitter i Styret, och med representanterna från föreningarna.\par
Martin Due undrar hur SAMO:s arbetsbelastning kommer påverkas.
Emelie svarar att just nu är SAMO ensam ansvarig för jämställdhet, så genom att dela upp arbetet blir det förhoppningsvis mindre arbete för SAMO.

\begin{beslut}
  \item bifalla propositionen.
\end{beslut}

\subsection{Skapande av mastermottagningsansvarig}
Sekreterare i Styret Joseph Löfving informerar om att det sker en Mastermottagning varje år, och att denna anordnas per utbildningsområde vilket innebär att sektionens masterstudenter är en del av KFM:s Mastermottagning.
För tillfället tillsätts de ansvariga på olika sätt från olika sektioner.
\hyperlink{bilagor/propmma.pdf.1}{Propositionen} yrkar istället på att sektionens representant ska tillsättas av sektionsmötet i läsperiod 3.
Han tillägger att Styret i efterhand fått reda på att posten är arvoderad.
Sektionsordförande Emelie Björkman informerar om att postens i nuläget omfattar att hyra Wikanders en kväll, och vill poängtera att det inte en hel Mottagning som ska arrangeras av de ansvariga.\par
Alexandru Golic undrar hur invalet kommer skötas om vi inte bifaller propositionen.
Joseph svarar att alternativet är att göra som innan med inval på styretmöten, vilket han tycker är mindre demokratiskt.

\begin{beslut}
  \item bifalla propositionen.
\end{beslut}

\section{Motioner}
\subsection{God kännande av tal manna presidiets hattar}
Motionären till lika Kistväktare Jesper Jäghagen påminner om att Focumateriet har införskaffat fina hattar till Talmanspresidiet.
Men i framtiden vill Focumateriet ha obegränsad kreativitet, och \hyperlink{bilagor/motionfoc.pdf.1}{deras motion} yrkar därmed på att styret inte ska vara en del av hattvalsprocessen.\par
Sekreterare i Styret Joseph Löfving presenterar \hyperlink{bilagor/motionfoc-svar.pdf.1}{Styrets motionssvar}.
Styret menar att en sådant godkännande av hattarna som för tillfället krävs från Styret bara behövs om man inte litar på Focumateriet.
Eftersom sektionen redan låter Focumateriet sköta closed captions under sektionsmötet tycker Styret att sektionen bör bifalla motionen.\par
Sekreterare Felix Augustsson påpekar att de flesta hattar verkar kunna ta sig igenom Styrets godkännande, och att Talmanspresidiet därför står bakom motionen.\par
Jacob Welander vill påminna sektionen om att det förr stod mellan Hattar och Mössor.
Huvudbonader menar han har att göra med frihet och genomförande av demokrati i världen och Sverige, och även han står därför bakom bifall.\par
Tarek Alhaskir undrar hur Focumateriet ska göra för att utnyttja sin nya makt fullt ut, och föreslår att Talmanspresidiet skulle kunna få full mundering.
Talman Ruben Seyer flikar in att man inte får göra personangrepp på klädstil, inte ens när det handlar om Talmanspresidiets klädstil.
Jesper svarar Tarek att Focumateriet är rätt chill, så därför är hattar där de drar gränsen (om Talmanspresidiet inte vill ha kläder).\par
Vice talman Martin Due säger att han tycker om sin hatt och därför litar mycket på Focumateriet.
Han yrkar därför i egenskap av sektionsmedlem på bifall.

\begin{beslut}
  \item bifalla motionen i sin helhet.
\end{beslut}

\subsection{Sektionsstenen Einsten}
Motionären Jacob Welander säger att vi samlats idag för att diskutera en rad punkter, men att den viktigaste av dem är hans motion.
Därefter reciterar han hela \hyperlink{bilagor/motionfnollk.pdf.1}{motionen}.\par
Sekreterare i Styret Joseph Löfving säger att Styret uppskattar roliga motioner, men ser varken någon fördel eller nackdel med motionen.
Styret anser inte att de vet vad som är roligt eller inte och yrkar därmed inte på något i sitt \hyperlink{bilagor/motionfnollk.svar.pdf.1}{motionssvar}.\par
Tobias Wallström säger att Styret försökt ta bort onödiga funktionärsposter, så därmed anser han att lägga till en ny sådan är onödigt.\par
Simon Franklin tycker att motionen är bra.\par
Alexandru Golic yrkar på motionen ska bifallas, med med ändringen att de ansvariga för stenen ska heta frisörer, eftersom de måste kunna klippa.
Motionären Jacob Welander tycker det var klipskt sagt.\par
Tarek Alhaskir säger att det tidigare varit diskussioner om andra kul poster.
Enligt visa ord framkom det då att \enquote{man ska låta grabbarna ha kul}.
Han undrar även om stenen ska stå kvar.
Jacob svarar att det kan vara något den invalda kan jobba med.\par
Felix Augustsson påminner om att sektionsmötet i läsperiod 3 2020 avslutades 01.17.\par
Emelie Björkman säger i egenskap av sektionsmedlem att hon först var skeptisk och tyckte att sektionsmedlemmar istället kunde söka Kräldjursvårdare.
Hon tycker även egentligen att sektionshets är dåligt, men efter samtal med Sektionsordförande på M-sektionen som tyckte att motionen var rolig är hon nu för den.
Hon tillägger att även om posten försvinner om några år är det kul just nu.\par
Ilma Aase tycker att om M-sektionen ska ha en sten så ska vi också ha en sten.\par
David Winroth tycker att vi ska låta grabbarna sköta sin sten.\par
Sexmästare Ludwig Gustavsson säger att som representant för F6 kan han meddela att F6 är för motionen eftersom den hetsar en annan sektion.\par
Alexandru undrar vad skötaren ska göra och hur mycket pengar den ska få.
Han har bland annat hört att FnollK har 20 tusen kronor över.
Jacob svarar att teorin om en sten är det fina, inte stenen i sig.
En del av uppdraget kommer att vara att komma med förslag om utvecklingen av framtida ansvar, med anknytning till bevarandet och omskötandet av stenen.

\begin{beslut}
  \item bifalla motionen med ändringsyrkande:
  \begin{itemize}
    \item Den ansvariga för stenen kallas Frisör istället för Einstenskötare.
  \end{itemize}
\end{beslut}

\section{Övriga frågor}
\subsection{Längden av sektionsmöten i läsperiod 3}
Tarek Alhaskir anser att Felix Augustsson under förra punkten hade en viktig poäng kring längden av av sektionsmöten i läsperiod 3.
Han undrar därför om Talmanspresidiet har en plan för att få det att gå snabbare.
Talman Ruben Seyer svarar att Talmanspresidiet försöker få mötena att gå snabbare, men att de inte har några planer på tekniker som de inte redan använder sig av under det nuvarande mötet.\par
Sektionskassör David Winroth säger att Styret har planer på att flytta vissa inval, och att de planerna kommer presenteras i läsperiod 3.

\subsection{Mats matlagning}
Erik Jansson undrar vad Mats Richardson lagar för mat.
Mats svarar att det är en del av Focumateriets satsning, parallellt med Foc News under mötet.

\section{Dumvästutdelning}
Nomineringarna är som följer:
\begin{itemize}
  \item \textbf{Molly Sigfridsdotter} nomineras av Emma Ödman.\\
  Molly, som är 19, ska åka Danmarksfärja tillsammans med två kompisar för att köpa öl.
  När båten stannar i Danmark tänker hon gå av för att köpa sallad.
  Detta får man såklart inte göra under pågående pandemi, och en utskällning av båtpersonalen följer.
  När hon kommer tillbaka till Sverige börjar hon bära av öl, men blir markerad av en droghund.
  Drogmarkeringen beror bara på att Molly nyligen köpt väskan på second hand, men när tullen ser nu ser närmare ifrågasätter de ungdomarnas alkoholinnehav.
  Detta resulterar i en timmas förhör innan de släpps (tillsammans med ölen, vilket är dumt av tullen).

  \item \textbf{Talman Ruben Seyer} nomineras av Samuel Martinsson.\\
  Ruben glömde under punkt §10 och §12 att stänga av sin mikrofon, vilket gjorde att talarna inte hördes över Rubens frenetiska skrivande på sitt tangentbord.

  \item \textbf{Talman Ruben Seyer} nomineras av Linnea Hallin.\\
  Ruben insåg först 1,5 timmar in på sektionsmötet att Focaderon som kom med hans nya hatt inte var helt barnvänlig.
  Han behövde därmed hålla i mötet under icke-optimala omständigheter.

  \item \textbf{Jacob Welander} nomineras av Emma Ödman.\\
  Jacob, som är 19, ska åka Danmarksfärja tillsammans med två kompisar för att köpa öl.
  Men Jacob glömmer sin legitimation, och får därmed inte gå på båten.

  \item \textbf{Talman Ruben Seyer} nomineras av David Winroth.\\
  Att skicka ut korrekta föredragslistor är något som David som Talman 19/20 kan intyga är av yttersta vikt.
  Men Ruben glömde i sin slutgiltiga föredragningslista att ta med de bordlagda ansvarsfriheterna från föregående möte, och behövde därmed få detta tillägg genomröstat med två tredjedelars majoritet.

  \item \textbf{Simon Järdnäs} nomineras av Leo Westin.\\
  Simon råkade skicka in sina phaddrars Mars-projekt som slutinlämning i Fysikingenjörens verktyg istället för sitt eget arbete. 

  \item \textbf{Erik Johansson} nomineras av Emma Ödman.\\
  Erik ska åka Danmarksfärja tillsammans med två kompisar som är under 20 för att köpa öl.
  När båten stannar i Danmark tänker han gå av för att köpa sallad.
  Detta får man såklart inte göra under pågående pandemi, och en utskällning av båtpersonalen följer.
  När han kommer tillbaka till Sverige ifrågasätter tullen efter lite komplikationer ungdomarnas alkoholinnehav.
  Som förklaring säger Erik \enquote{Det är till ett kalas med vår Chalmersklass}.
  Detta resulterar i en timmas förhör innan de släpps (tillsammans med ölen, vilket är dumt av tullen).

\end{itemize}

\begin{beslut}
  \item Erik Johansson var dummast.
\end{beslut}
Efter noggrann jämförelse med föregående dumvästinnehavare Ruben Frilund går mötet till beslut.
\begin{beslut}
  \item ge Erik Johansson Dumvästen.
\end{beslut}

\section{Mötets avslutande}
Mötet avslutas 23.24 av Talman Ruben Seyer.

\clearpage
\section*{Signaturer}
\label{sec:sig}
\addcontentsline{toc}{section}{\nameref{sec:sig}}
En signatur på den här sidan avser hela sektionsmötesprotokollet \themote. Det sker elektroniskt via digitalt ID. Kontakta talmanspresidiets sekreterare på \href{mailto:talman.sekr@ftek.se}{\texttt{talman.sekr@ftek.se}} vid kontroll eller övriga frågor. 

\vspace{4cm}
\begin{center}
  \makebox[12cm][c]{
    \parbox{12cm}{
      \signatur{Ruben Seyer}{Talman} \hfill
      \signatur{Felix Augustsson}{Sekreterare}
      \newline
      \signatur{Alexandru Golic}{Justerare} \hfill
      \signatur{Axel Flordal}{Justerare}
      \vspace{1.5em}
    }
  }
\end{center}

\clearpage
\begin{bilagor}
  \bilaga{Beslut att fastställa}{beslut.pdf}

  \bilaga{Reviderad revisionsberättelse Studienämnden (SNF) 19/20}{revisionsnf1920.pdf}

  \bilaga{Nomineringar FnollK 2021}{nomfnollk.pdf}
  
  \bilaga{Nomineringar FARM 2021}{nomfarm.pdf}
  \bilaga{Gällande avslaget av gruppnomineringen för FARM 2021}{nomfarm-avslag.pdf}
  
  \bilaga{Proposition om Skapande av jämlikhetsråd}{propjaemf.pdf}
  \bilaga{Proposition om Skapande av mastermottagningsansvarig}{propmma.pdf}
  
  \bilaga{Motion om God kännande av tal manna presidiets hattar}{motionfoc.pdf}
  \bilaga{Motionssvar till God kännande av tal manna presidiets hattar}{motionfoc-svar.pdf}
  \bilaga{Motion om Sektionsstenen Einsten}{motionfnollk.pdf}
  \bilaga{Motionssvar till Sektionstenen Einsten}{motionfnollk-svar.pdf}
\end{bilagor}

\end{document}