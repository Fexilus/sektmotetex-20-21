\documentclass[hidelinks]{sektionsmote}
\usepackage{digsig}
\usepackage{csquotes}
\usepackage{censor}
\usepackage[normalem]{ulem}

\title{Protokoll fört vid sektionsmöte}
\shorttitle{Sektionsmötesprotokoll}
\motesdag{6}
\motesmanad{5}
\motesar{2021}
\motesnr{04}
\motestid{17.20}
\motesplats{Zoom}
\verksamhetsar{20/21}

\makeheader

\begin{document}
\maketitle

\section{Mötets öppnande}
Mötet öppnas \tid av Fysikteknologsektionens Talman Ruben Seyer.


\section{Mötets behörighet och beslutsförighet}
Talman Ruben Seyer meddelar att mötet är utlyst korrekt och i tid.
Han frågar mötet om det anses vara behörigt och beslutsförigt.
\begin{beslut}
    \item anse mötet behörigt och beslutsförigt enligt stadgarna.
\end{beslut}


\section{Val av justerare}
Alexandru Golic och Albert Vesterlund nomineras till justerare.
\begin{beslut}
  \item välja Alexandru Golic och Albert Vesterlund till justerare.
\end{beslut}


\section{Val av rösträknare}
Ella Larsson och Emelie Sjögren nomineras till rösträknare.
\begin{beslut}
  \item välja Ella Larsson och Emelie Sjögren till rösträknare.
\end{beslut}


\section{Fastställande av föredragningslista}
Tarek Alhaskir yrkar på att flytta motion §15 b Sluta kräva inval innan motioner och propositioner till mellan punkt §17 Dumvästutdelning och punkt §18 Mötets avslutande.

Karin Hult yrkar på att flytta punkt §14 Propositioner och punkt §15 Motioner till mellan punkt §12 Preliminärverksamhetsplan och budget Sektionsstyrelsen och punkt §13 Personval.

Tarek jämkar sig med Karins yrkande.
\begin{beslut}
  \item fastställa föredragningslistan med följande ändringar:
  \begin{itemize}
    \item Flytta punkt §14 Propositioner till punkt §13 och punkt §15 Motioner till punkt §14, och ändra numreringen på efterkommande punkter.
  \end{itemize}
\end{beslut}


\section{Adjungeringar}
Inga adjungeringar föreligger.


\section{Föregående mötesprotokoll}
Sekreterare Felix Augustsson informerar om att det föregående protokollet justerades och anslogs i tid i enlighet med stadgan.
\begin{beslut}
    \item lägga föregående mötesprotokoll till handlingarna. 
\end{beslut}


\section{Uppföljning av beslut}
Inga beslut att följa upp.


\section{Fastställande av beslut}

\subsection{Fyllnadsval}
\begin{beslut}
  \item fastställa Sektionsstyrelsens beslut att välja:
  \begin{itemize}
      \item Åke Andersson som teknolog i Spidera
      \item Willem de Wilde som piff i Piff och Puff
      \item Pontus Gustafsson som teknolog i Spidera
      \item Mijo Thoresson som Game Boy i Game Boy
      \item Isac Borghed, Maja Rhodin och Vilgot Jansson som bakisar Bakisclubben
      \item Karin Hult som ledamot i Blodgruppen.
  \end{itemize}
\end{beslut}


\section{Meddelanden}

\subsection{Sektionsstyrelsen}
Skyddsombud Emelie Sjögren meddelar att resultatet av Studentbarometern är sammanställt, vilket går att hitta på internet.
Hon för även fram att man kan kontakta vice sektionsordförande Sofia Reiner om man har frågor om köksrenoveringen.

Sekreterare i Styret Joseph Löfving informerar om att Sektionsstyrelsen ändrat i Game Boy:s arbetsordnings så att det står att de är en sektionsförening.

Sektionskassör David Winroth berättar att sektionens ekonomi går bra, men att Sektionsstyrelsen på grund av prishöjning höjt budgeten för köksrenoveringen med 80 000kr från programbidraget.

\subsection{Kårledningen}
Sektionens kårledningskontakt Gabriel Aspegrén säger att en ny kårledning valts in.
Det finns en ny preliminär verksamhetsplan på Fullmäktiges hemsida (detta verkar inte stämma).
Där finns det även en statusrapport kring årets verksamhetsplanspunkter.
Han tillägger att man som teknolog kan äska pengar från Kårledningen.
Till sist nämner han att det kommer ske vaccinering i kårhuset.

\subsection{Focumateriet}
Kistväktare Jesper Jäghagen påminner om att det är \enquote{Safe Sex LV6} och att man bör hålla avstånd till andra.
Han informerar även om att det bedrivs oberoende journalistik kring det stekheta mötet på \newline\href{https://focumateriet.wordpress.com}{\texttt{focumateriet.wordpress.com}}.
Till sist nämner han att Focumateriet har billigaste lesken på campus.

\subsection{Studienämnden (SNF)}
Masteransvarig Tobias Gabrielii nämner att SNF har få nomineringar, men att man kan söka direkt på mötet om man vill engagera sig.
Han tycker att man borde göra det eftersom SNF är både kul och viktigt.


\section{Verksamhets- och revisionsberättelser}

\subsection{Fysikteknologsektionens Idrottsförening, FIF 2020}

\subsubsection{Verksamhetsberättelse}
Tidigare ordförande Adam Johansson berättar att FIF under året anpassat sin verksamhet till att vara på distans.
De arrangerade bland annat FIF--joggen och dans med programansvarig för TM Julie Rowlett.
De har även köpt material med pengar från programansvariga.
Han avslutar med att tacka Sektionsstyrelsen, pateter, de teknologer som medverkade på arrangemangen med flera.

Emelie Björkman undrar vilka material som köptes in.
Adam svarar att det bland annat köptes in badmintonutrustning.

\begin{beslut}
  \item med godkännande lägga \hyperlink{bilagor/fifvb.pdf.1}{verksamhetsberättelsen} till handlingarna.
\end{beslut}

\subsubsection{Revisionsberättelse}
Lekmannarevisor Jonathan Bengtsson berättar att det här är första året som FIF har haft ansvar för bokföring. 
Eftersom FIF har ett verksamhetsår som sträcker sig från april till mars innebär det att de endast haft en månad på sig till bokslut.
Därför är revisionen inte klar, och revisorerna yrkar därmed på att bordlägga revisionsberättelse och ansvarsfrihet.
Av samma anledning presenterar därför inte revisorerna någon revisionsberättelse.

\subsubsection{Ansvarsfrihet}
\begin{beslut}
  \item bordlägga frågan om ansvarsfrihet.
\end{beslut}

\section{Preliminär verksamhetsplan och budget Sektionsstyrelsen}

\subsection{Verksamhetsplan}
Sekreterare i Styret Joseph Löfving presenterar \hyperlink{bilagor/pvp.pdf.1}{Sektionsstyrelsens preliminära verksamhetsplan}.
Den är väldigt likt nuvarande verksamhetsplan, men några nya punkter har lagts till.
Det finns bland annat punkter om renoveringen av köket, utbildning av sektionsaktiva och att dokumentera positiva effekter av distansstudier.

Jack Vahnberg tycker att planen ser fin ut, men påpekar att punkten om psykosociala läget är åtminstone 4 år gammal.
Han undrar därför om det finns några konkreta planer kring punkten.
Skyddsombud Emelie Sjögren svarar att det för tillfället pågår en större omstrukturering av programmen.
Eftersom många av problemen har sin grund i utbildningarnas struktur är därför planen att under nästa år utvärdera hur ändringarna från en psykosocial synpunkt tillsammans med SNF.

\subsection{Budget}
Sektionskassör David Winroth presenterar \hyperlink{bilagor/budget.pdf.1}{den preliminära budgeten}.
Det är flera förändringar i budgeten, eftersom verksamheten antagligen kommer gå tillbaka att vara mer normal under nästkommande verksamhetsår.
Om pandemin fortsätter att tydligt påverka verksamheten kommer den slutgiltiga budgeten justeras för att reflektera detta.
Han menar även att det är värt att ha i åtanke att nuvarande budget bland annat förändrades under föregående sektionsmöte.
Några ändringar som är värde att fokusera på är att FnollK:s budget växer med 20 000kr till 200 000kr eftersom det är dyrare att ha en eventuellt uppdelad Mottagning, samt att Focumateriet får extra pengar för att upprusta Focus.

Tarek Alhaskir undrar varför inkomsterna från uthyrning av Focus budgeteras likadant som nuvarande år.
David svarar att nuvarande budget inte är anpassad efter pandemin.


\section{Propositioner}

\subsection{Uppstädning av styrdokument (2:a läsningen)}
Sekreterare i Styret Joseph Löfving presenterar \hyperlink{bilagor/prop/a.pdf.1}propositionen.

\begin{beslut}
  \item i andra läsningen bifalla \hyperlink{bilagor/prop/a.pdf.1}{propositionens} stadgeändringar.
\end{beslut}

\section{Motioner}

\subsection{Förflyttandet av invalet av memeposter samt sektionsnörd för underlättandet av sektionsmötet LP3}
Motionären Alexandru Golic presenterar \hyperlink{bilagor/motion/a.pdf.1}{motionen}.
Han tycker att sektionsmötena i läsperiod 3 är långa och smärtsamma.
Eftersom kräldjursvårdare och frisörer inte är viktiga för sektionen ska fungera menar han att dessa inval drar ut på mötet i onödan.
Sedan föreslår han även att invalet av sektionsnörd ska flyttas till läsperiod 1 eftersom Dragosföreläsningen då är över.

Sekreterare i Styret Joseph Löfving presenterar \hyperlink{bilagor/motion/asvar.pdf.1}{Sektionsstyrelsens motionssvar}.
Eftersom godkännande av motionen på nuvarande möte skulle innebära att de nuvarande berörda sektionsaktiva skulle få ett halverat verksamhetsår yrkar Sektionsstyrelsen på bordläggning av frågan.

Tobias Wallström håller med om att invalet av kräldjursvårdare och frisörer bör flyttas, men inte invalet av sektionsnörd.
Han menar att detta skulle påverka möjligheten att återinföra bonuspoäng från Dragostentan, eftersom sektionsnörden då skulle gått av innan tentaveckan i läsperiod 1.

Axel Flordal undrar om de nuvarande berörda sektionsaktiva kommer att få ett verksamhetsår som är ett och ett halvt år långt om motionen röstas igenom under nästa möte.
Talman Ruben Seyer svarar att det skulle vara effekten om mötet inte tar beslut som ändrar på situationen.

Tarek Alhaskir yrkar på att stryka alla delar av motionen som involverar sektionsnörden.

Alexandru säger att han efter kritiken känner sig tveksam till motionen.

Karin Hult tycker att det inte gör så mycket om kräldjursvårdaren och frisörerna bara sitter i ett halvår eftersom de ändå inte gör någonting.

Joseph menar att man då lika väl kan bordlägga.

Alexandru håller med om att bordläggning är det bästa alternativet.

\begin{beslut}
  \item bordlägga \hyperlink{bilagor/motion/a.pdf.1}{motionen} till första sektionsmötet 21/22.
\end{beslut}

\subsection{Sluta kräva inval innan motioner och propositioner}
Motionären Alexandru Golic presenterar \hyperlink{bilagor/motion/b.pdf.1}{motionen}.
Han tycker att propositionerna och motionerna under förra mötet presenterades för sent på kvällen.
Han menar att mötet då inte orkar diskutera frågorna tillräckligt mycket.
Hans åsikt är därför att talmanspresidiet ska få välja ordningen på inval och propositioner och motioner.

Sekreterare i Styret Joseph Löfving presenterar \hyperlink{bilagor/motion/bsvar.pdf.1}{Sektionsstyrelsens motionssvar}.
Eftersom Sektionsstyrelsen inte kan hitta någon motivering till det stående sektionsmötesbeslutet i frågan yrkar de på bifall.

Tarek Alhaskir undrar om det går att formulera om motionen så att det gamla beslutet istället rivs upp.
Talman Ruben Seyer svarar att det är presidiets tolkning att det är vad motionen i praktiken skulle innebära.

\begin{beslut}
  \item bifalla \hyperlink{bilagor/motion/b.pdf.1}{motionen} i sin helhet, det vill säga riva upp sektionsmötesbeslut 1992-04-09-a.
\end{beslut}

\subsection{Ändra Fabiolas stadga}
Motionären Ajša Ćuprija presenterar \hyperlink{bilagor/motion/c.pdf.1}{motionen}.
Fabiola vill vara mer inkluderande, och vill därför explicit lägga till icke-binära i beskrivningen av vilka Fabiola är ålagda att arrangera arrangemang för.

Sekreterare i Styret Joseph Löfving presenterar \hyperlink{bilagor/motion/csvar.pdf.1}{Sektionsstyrelsens motionssvar}.
Han säger att Sektionsstyrelsens beslut i frågan var det kortaste någonsin, och att de yrkar på bifall.

\begin{beslut}
  \item bifalla \hyperlink{bilagor/motion/c.pdf.1}{motionen} i sin helhet.
\end{beslut}


\section{Personval}

\subsection{Dragos}

\subsubsection{Val av Dragos}
Mr. Walker loggar in.
Han uppvisar till jubel och lovrop tio tigrars styrka, följt av en demonstration av hur man går på stadens gator som en vanlig man.
Efter en utfrågning, där varje svar får blodet att isa och varje rörelse blixten att stå stilla, går mötet till val.

\begin{beslut}
  \item välja in Mr. Walker till Dragos.
\end{beslut}

\subsection{Revisorer}

\subsubsection{Val av 2 lekmannarevisorer}
Det finns 1 sökande:
\begin{itemize}
    \item Tobias Gabrielii.
\end{itemize}

\paragraph{Tobias Gabrielii} är 22, f17 och har körkort.
Han söker för att någon nominerade honom och för att det är viktigt.
Helst skulle han vilja att det fanns två lekmannarevisorer.

\begin{beslut}
  \item välja Tobias Gabrielii till lekmannarevisor, samt vakantsätta en plats.
\end{beslut}

\subsection{Valberedningen}

\subsubsection{Val av 3--7 ledamöter}
Det finns 4 sökande:
\begin{itemize}
    \item Isabella Tepp
    \item Markus Utterström
    \item Joseph Löfving
    \item Mathias Arvidsson.
\end{itemize}

\paragraph{Isabella Tepp} är 20, f19 och har körkort.
Hon söker för att det var kul att vara representant för Djungelpatrullen i årets valberedning.
Hon tycker det är viktigt att hitta bra grupper, och tycker att det för traditionens skull borde vara hälften Djungelpatrullspateter i valberedningen.

Carl Strandby undrar vilka bullar Isabella vill bjuda på under valberedningen.
Isabella svarar att hon har fått berättat för sig att hon ska bjuda på vaniljbullar.

\paragraph{Markus Utterström} är 21, f19 och har körkort.
Även han var med som representant för Djungelpatrullen i årets valberedning.
Han tycker att valberedningen är ett kul och kreativt engagemang.
Han tycker även att det är viktigt att sektionen har bra kommittéer, och tror att han kan bidra med det.

\paragraph{Joseph Löfving} är 22, f18 och har inte körkort.
Han var med som representant för Sektionsstyrelsen i årets valberedning.
Han tycker att valberedningen är ett viktigt organ som är viktigt att fylla.

\paragraph{Mathias Arvidsson} är 22, f19 och har inte körkort.
Han söker eftersom han inte gett allt till sektionen än.
Han vill att sektionens kommittéer ska vara tillförlitliga och ha ett bra ansikte utåt.

\begin{beslut}
  \item välja in 7 ledamöter i valberedningen.
  \item välja Isabella Tepp, Markus Utterström, Joseph Löfving och Mathias Arvidsson till ledamöter i Valberedningen, samt vakantsätta 3 platser.
\end{beslut}

\subsection{Sektionsstyrelsen, kärnstyret}

\subsubsection{Val av sektionsordförande}
Det finns 1 sökande:
\begin{itemize}
    \item Carl Strandby.
\end{itemize}

\paragraph{Carl Strandby} är 24, f19 och har körkort.
Han söker eftersom han har en passion ledarskap och organisation.
Att göra det på en sektion som han älskar och med så stor potential ser han på som en chans att ge tillbaka.
Han vill arbeta för mindre administration, lägre engagemangströsklar och mer samarbete för att lösa problem.
Han tror att flera av sektionens problem, så som psykisk ohälsa, avhopp och långa sektionsmöten går att lösa genom att öppna upp ett förändringsarbete där kloka huvuden kan slås ihop.

Sektionsordförande 18/19 Jack Vahnberg tycker att det är kul att Carl vill hantera konkreta problem, men menar att en stor del av rollen handlar om att medla.
Han undrar därför hur Carl tycker att man behandlar situationer där folk tycker olika.
Carl svarar att han har erfarenheter av detta från FnollK.
Han ser att traditionsförändringar är ett område där många personer ofta har starka känslor.
Lösningen är då att jobba tillsammans, eftersom man ofta vill åt samma håll.
Han tycker om personligt ägandeskap i frågor, där medgivande är viktigare än konsensus.
Han upplever att det blir färre konflikter då.

Sektionsordförande Emelie Björkman undrar hur en bra ledare är, och hur Carl är som ledare.
Carl svarar att han tror att hans styrkor är att uppmuntra, delegera och ha tillit till att andra personer tar ansvar för sina uppdrag på egen hand.

Jack undrar om Carl har en åsikt om att valberedningens nomineringar nästan alltid väljs in på sektionsmöten.
Carl svarar att han har politisk bakgrund, och att det där finns mycket tillit till valberedningar.
Han tycker att det är viktigt att kunna ställa sig upp på sektionsmöten och kandidera, och att det är viktigt att valberedningen kan öppna upp diskussion kring inval innan mötet.

\begin{beslut}
  \item välja Carl Strandby till sektionsordförande.
\end{beslut}

\subsubsection{Val av vice sektionsordförande}
Det finns 1 sökande:
\begin{itemize}
    \item David Bååw.
\end{itemize}

\paragraph{David Bååw} är 22, f19 och har körkort.
Han söker för hjälpa sektionen i en mer seriös kapacitet.
Även om F6 var kul tror han att han kan göra mer, och anser sig ha bra insikt i hur det är att vara sektionsaktiv.

Sektionsordförande Emelie Björkman undrar om David har någon hjärtefråga.
David tycker att det är viktigt att de aktiva på sektionen syns utåt.
Han vill jobba med att utöka representationsmöjligheterna under arrangemang genom exempelvis flaggor, och att ge alla den synlighet som krävs för.

\begin{beslut}
  \item välja David Bååw till vice sektionsordförande.
\end{beslut}

\subsubsection{Val av sektionskassör}
Det finns 1 sökande:
\begin{itemize}
    \item Axel Flordal.
\end{itemize}

\paragraph{Axel Flordal} är 19, tm20 och har körkort.
Han tror att det kan vara roligt att sitta i Styret, och att det är viktigt att få sektionen att leva vidare och fungera.

Sektionskassör David Winroth undrar hur Axel känner för bokföring och långa bokföringskvällar.
Axel svarar att det kommer bli en utmaning, men att det kommer vara en bra erfarenhet att inte hinna med allt man vill, och att bokföra.

Ordförande i FnollK Emma Ödman undrar om FnollK:s budget kan höjas ytterligare.
Axel svarar att det alltid går att se över, men att han i nuvarande situation inte kan ge ett svar.

\begin{beslut}
  \item välja Axel Flordal till sektionskassör.
\end{beslut}

\subsubsection{Val av sekreterare}
Det finns 1 sökande:
\begin{itemize}
    \item Linnea Hallin.
\end{itemize}

\paragraph{Linnea Hallin} är 21, tm18 och har körkort.
Hon söker eftersom hon vill fortsätta vara sektionsaktiv efter att ha suttit i SNF i ett år.
Hon tror att det kan vara roligt att sitta i Sektionsstyrelsen.

Sekreterare i Styret Joseph Löfving påpekar att det bra att skriva snabbt som sekreterare, och undrar därför hur snabbt Linnea skriver.
Linnea svarar att hon skriver långsammare än hon önskar.

\begin{beslut}
  \item välja Linnea Hallin till sekreterare i Sektionsstyrelsen.
\end{beslut}

\subsubsection{Val av skyddsombud (SAMO)}
Det finns 1 sökande:
\begin{itemize}
    \item Emelie Lemann.
\end{itemize}

\paragraph{Emelie Lemann} är 21, tm19 och har körkort.
Hon söker för att ge sektionsmedlemmar en trivsam miljö.
Hon vill förbättra den psykisk hälsan på sektionen, och ser fram emot att arbeta med det nya JämF.

Skyddsombud Emelie Sjögren säger att det som skyddsombud finns mycket utrymme egna projekt, så som att vidareutveckla JämF.
Hon undrar därför vad Emelie Lemann är intresserad av.
Emelie Lemann svarar att hon vill se ett samarbete med kommittéer för att arrangera saker som får sektionsmedlemmar att bry sig som jämlikhet.
På så sätt går det att göra JämF fortsatt relevant.

Markus Utterström undrar om Emelie Lemann ser något problem med att hennes företrädare om hon blir invald kommer ha samma förnamn och kommittéhistorik.
Emelie Lemann svarar att det snarare är roligt än ett problem.

\begin{beslut}
  \item välja Emelie Lemann till skyddsombud.
\end{beslut}

\subsubsection{Val av informationsansvarig}
Det finns 1 sökande:
\begin{itemize}
    \item Mijo Thoresson.
\end{itemize}

\paragraph{Mijo Thoresson} är 20, f19 och har körkort.
Han söker eftersom det är kul att engagera sig.
Han tycker att de andra invalda är rimliga, och tror att det kommer bli kul att arbeta tillsammans under året.
Hans mål är att godkänna inlägg i Facebook-gruppen lika snabbt som sin föregångare.

Informationsansvarig Albert Vesterlund säger att det kan vara svårt att göra lämplig och effektiv reklam och informationsutskick.
Han undrar om Mijo har några tankar kring hur man når ut till internationella studenter och de som inte engagerar sig i sektionen.
Mijo svarar att Facebook brukar fungerar bra.
De går i slutändan inte att nå ut till de som inte är intresserade av information, men han tror att ett regelbundet användande av flödet på ftek kan bygga upp vana hos folk, och på så sätt bli en bredare informationskanal.
Han reflekterar även över att det kanske skulle vara bra att kanske fråga en internationell student om vad de tänker.

Tarek Alhaskir påminner att man som del av Sektionsstyrelsen ska förbereda motionssvar.
Han undrar därför hur Mijo ställer sig till \enquote{inte jätteseriösa motioner}.
Mijo tycker man som sektionsmedlem har rätt att skicka in vilka motioner man vill, och att Sektionsstyrelsen får man finna sig i det.
Om motionen inte känns seriös får man helt enkelt skriva det i motionssvaret.

Tarek frågar om Mijo tycker att det finns en gräns för hur många motioner som är för många?
Mijo tycker, i egenskap av någon som går på sektionsmöten att det kanske borde finnas det.
Men om någon tar sig tiden att skriva 100 motioner så är det bra att man får det.
Han avslutar med att säga att man inte borde missbruka denna möjlighet.

\begin{beslut}
  \item välja Mijo Thoresson till informationsansvarig.
\end{beslut}

\subsection{JämF}

\subsubsection{Val av 2--3 fristående ledamöter}
Det finns 2 sökande:
\begin{itemize}
    \item Ajša Ćuprija
    \item Ludvig Gustavsson.
\end{itemize}

\paragraph{Ajša Ćuprija} är 23, f17 och har körkort.
Hon säker eftersom jämlikhet är viktigt att ta upp och värt att jobba på.
Hon tycker att det är ett bra koncept att ha representanter i ett råd.
Hon vill under året se vad som görs i jämlikhetsarbetet på på andra sektioner och Chalmers centralt.

Skyddsombud Emelie Sjögren undrar om det är något specifikt Ajša vill dra i.
Ajša vill dels jobba med jämställdhet, och dels arbeta med att se över varför sektionen har den fördelning av bakgrunder den i nuläget har och hur detta påverkar.
Hon vill få sektionsmedlemmar att ifrågasätta det omkring sig, både genom utbildning och workshops.

Tarek Alhaskir undrar hur Ajša ser på att JämF och Fabiola kan överlappa lite när det kommer till jämställdhet.
Tycker hon att organen bör samarbeta eller att de ska hållas isär?
Ajša svarar att hon ser på Fabiola mer som ett nätverk som skapar gemenskap, medans JämF ökar synlighet av jämställdhetsfrågor.
Hon tillägger att JämF även arbetar med annan ojämlikhet så som socioekonomiska bakgrunder.

\paragraph{Ludvig Gustavsson} är 22, f19 och har körkort.
Han söker eftersom jämlikhet på sektionen är viktigt.
Han tror att JämF kan bidra till ökad jämlikhet, och han vill bidra till detta samt hjälpa rådets roll på sektionen att växa.

Skyddsombud Emelie Sjögren undrar om det är något specifikt Ludvig vill dra i.
Ludvig tycker att ett bra första steg är att lyfta jämlikhet som ämne och sprida kunskap genom till exempel lunchföreläsningar.

\begin{beslut}
  \item välja 3 fristående ledamöter i JämF.
  \item välja Ajša Ćuprija och Ludvig Gustavsson till fristående ledamöter i JämF, samt vakantsätta en plats.
\end{beslut}

\subsection{Talmanspresidiet}

\subsubsection{Val av talman}
Det finns 1 sökande:
\begin{itemize}
    \item Erik Broback.
\end{itemize}

\paragraph{Erik Broback} är 20, f19 och har körkort.
Han söker för att han tror att det kan vara kul, och för att sektionsmöten är viktiga.
Han tycker att det är viktigt att någon tar hand om mötena och håller i dem på ett bra sätt, vilket han tror att han både kan vara bra på och tycka om.

Tarek Alhaskir undrar om Erik har någon erfarenhet av att hålla i stora möten.
Han undrar även om Erik har någon taktik när han ska uttala namn, med tillägget att ingen på sektionen någonsin uttalat Alhaskir rätt.
Erik svarar att han inte har erfarenhet av stora möten, men av att leda mindre möten från gymnasietiden och som spelledare i rollspel.
Han har dock ingen bra taktik när det kommer till uttal.
Kanske kan det vara bra att plugga i förväg, be om ursäkt om man säger fel och sedan uttala bättre nästa gång.

Mats Rickardson påpekar att det är viktigt att dricka Focadero, och undrar hur Erik ställer sig till det?
Erik har inga problem med att dricka Focadero, men tycker att man bör göra det på ett bra sätt så att man kan fortsätta leda sektionsmötet på ett bra sätt.

David Winroth påpekar att man som talman inte bara leder möten, och undrar därför om det finns några styrdokument som Erik tycker bör förbättras?
Erik har inga stora ändringsförslag, men har några petiga små tydliggöranden och bättre åsyftningar han skulle vilja ha göra.

Emelie Björkman undrar om Erik har några reflektioner kring de fördelar som finns med digitala sektionsmöten, och undrar om han tänker använda sig av det även när mötena kan vara fysiska.
Erik svarar att även om sektionen blivit tvingad att ha digitala sektionsmöten bör vi dra lärdomar av erfarenheten.
Till exempel tror han att en större användning av det digitala röstsystemet kan göra mötena mer rättvisa.
Han tycker dock att det är bäst att ha dem på plats när det går.

Vice talman Martin Due undrar om Erik har ett mekaniskt tangentbord, och om han i sådana fall kan skaffa ett annat tangentbord för att höras bättre under digitala möten.
Erik har ett mekaniskt tangentbord, men planerar inte att skriva samtidigt som han pratar.

\begin{beslut}
  \item välja Erik Broback till talman.
\end{beslut}

\subsubsection{Val av vice talman}
Det finns inga sökande.

\subsubsection{Val av sekreterare}
Det finns 1 sökande:
\begin{itemize}
    \item Samuel Martinsson.
\end{itemize}

\paragraph{Samuel Martinsson} är 19, tm20 och har körkort.
Han söker eftersom han tycker att det är viktigt att ha ett oberoende organ som håller i sektionsmöten.

Alexandru Golic undrar om Samuel har hetat något annat innan, eftersom han sa att han \enquote{\dots heter Samuel Martinsson nu}.
Det har Samuel tyvärr inte.

\begin{beslut}
  \item välja Samuel Martinsson till sekreterare i talmanspresidiet.
\end{beslut}

\begin{ofraga}
  Sekreterare Felix Augustsson vill ajournera mötet i 30 minuter.
  \begin{beslut}
    \item ajournera mötet till 20.43.
  \end{beslut}
\end{ofraga}

\subsection{Focumateriet}

\subsubsection{Val av kapten}
Det finns 1 sökande:
\begin{itemize}
    \item Martin Bergström.
\end{itemize}

\paragraph{Martin Bergström} är 22, tm17 och har körkort.
Han söker eftersom han har suttit i ett år, och tror att han kan göra ett bra jobb som kap ten.
Hans plan är att expandera Focumateriets imperium.

Emelie Björkman undrar om Martin har erfarenhet av att sitta i Sektionsstyrelsen, och om han har några tankar kring hur han kan bidra till Sektionsstyrelsens arbete.
Martin svarar att han varit med tidigare, och därför har bra koll på hur det brukar vara.
Han tror med sin erfarenhet att han kommer vara den rimliga i nästa års Sektionsstyrelse.

Andreas Spetz undrar om Martin tror att han kan fylla kapten Gustav Hallbergs skor.
Martin svarar att Gustavs skor är stora som clownskor, och därför inte går att fylla.
Martin tillägger att han bara kan styra bra åt vänster, medans höger är lite klurigare.

\begin{beslut}
  \item välja Martin Bergström till kapten.
\end{beslut}

\subsubsection{Val av automatpirat}
Det finns 1 sökande:
\begin{itemize}
    \item Jesper Jäghagen.
\end{itemize}

\paragraph{Jesper Jäghagen} är 22, tm18 och har körkort.
Han söker för att fortsätta ge sektionen billigast lesk på campus, och för att det ska bli \enquote{gött häng nästa år}.

\begin{beslut}
  \item välja Jesper Jäghagen till automatpirat.
\end{beslut}

\subsubsection{Val av kistväktare}
Det finns 1 sökande:
\begin{itemize}
    \item Jonas Bohlin.
\end{itemize}

\paragraph{Jonas Bohlin} är 22, f17 och har körkort.
Han söker eftersom Focumateriet gör bra grejer och är viktigt.
Han tillägger att Jesper lämnar stora fotspår i bokföringsdepartementet, vilket Jonas ska göra sitt bästa för att fylla.

Arvid Andersson undrar hur det nu ska gå med Jonas förenings-hat-trick.
Jonas svarar att när han ser till den yngre generationen ser ljust ut.

Tobias Wallström undrar om Jonas kommer skicka Focumateriets pengar utomlands.
Jonas svarar att de flesta pengar redan är i ett bengaliskt konto, och att han ska investera resten i dogecoin.

Emelie Björkman undrar vad Jonas ska skicka Jesper att köpa till Focumama?
Jonas svarar att om man skicka Jesper har man redan gjort fel.
Det som saknas i Focumama är chokladbollar, såvida inget har ändrat sig sedan Jonas var på Focus.

\begin{beslut}
  \item välja Jonas Bohlin till kistväktare.
\end{beslut}

\subsubsection{Val av 0--5 ledamöter}
\hyperlink{bilagor/nom/foc.pdf.1}{Valberedningens gruppnominering} till Focumateriet väljs in automatiskt i enlighet med reglementet då de nominerade förtroendeposterna valts in.
Valberedningen läser upp gruppens nominering.

\subsection{F6}

\subsubsection{Val av sexmästare}
Det finns 1 sökande:
\begin{itemize}
    \item Jesper Bergström.
\end{itemize}

\paragraph{Jesper Bergström} är 21, f19 och har körkort.
Han söker eftersom hans förra år i FnollK var det roligaste han någonsin gjort.
Han tror att han i F6 kan få ett nytt perspektiv, särskilt i en ledarskapsroll som ordförande.
Vidare tycker han att F6 viktig eftersom de \enquote{är det där sköna} på sektionen.
Han tycker att ET-raj är fantastiska, och vill bidra till att andra teknologer ska få gå på sådana.
Han avslutar med att säga att han tror sig kunna bidra med sina tidigare erfarenheter, och att han kommer kunna utvecklas med genom att leda en grupp och sitta i Sektionsstyrelsen.

Emelie Björkman undrar hur Jesper tycker en bra ledare är, och hur Jesper planerar att bidra till Sektionsstyrelsen.
Jesper svarar att det är viktigt att lyssna eftersom alla inte är lika bekväma med att uttrycka åsikter.
Han tycker att det är viktigt att få till bra kompromisser, och tror att det kommer vara kul att lyssna på \sout{sektions}styrelsemöten.
Han tror att han kommer kunna bidra som idéspruta, och att han kan \enquote{tänka utanför lådan}.

Sexmästare Ludvig Gustavsson säger att man som ordförande ibland behöver be om ursäkt, och undrar vad Jesper tänker kring hur man gör det.
Jesper svarar att man kan bjuda på öl, acceptera sitt misstag och förmedla att man tänker undvika att det händer igen.

Joel Sandås säger att MK tyckte att glasbjudning var ett dåligt Nolluppdrag, men undrar vilken typ av glass Jesper skulle bjuda tillbaka med?
Jesper svarar glassbåtar.

Emma Ödman undrar om Jesper har några planer på att överlåta SP till FnollK?
Jesper svarar att även om vissa tror att han söker å FnollK:s vägnar kommer det inte på tal att FnollK ska göra något annat där än att vara på besök.

\begin{beslut}
  \item välja Jesper Bergström till sexmästare.
\end{beslut}

\subsubsection{Val av sexreterare}
Det finns 1 sökande:
\begin{itemize}
    \item Fredrik Skoglund.
\end{itemize}

\paragraph{Fredrik Skoglund} är 21, f19 och har körkort.
Han söker eftersom att han inte känner sig helt klar sen han gick av FnollK.
Han tyckte att F6 aspar var sköna, och tänkte att det då bara var att köra.

Sexreterare Jacob Burman säger att sitta som sexreterare är som att ta hand om 8 små dagisbarn i ett år, och undrar om Fredrik gillar att göra detta.
Fredrik svarar att han tror att det kan misstolkas om man säger att man gillar barn.
Hans förhoppning är att resten av kommittén kommer bestå av hyfsat ansvarstagande individer.
I värsta fall får han ryta till så skärper de sig, och säga snälla ord och sätta plåster på knän.

Emma Ödman undrar om Fredrik kommer rycka in på styrelsemöten vid eventuell försovning.
Fredrik svarar att han en gång i NollK fick gå på ett \sout{sektions}styrelsemöte, och att det var lite som att få lite whiplash.
Men i slutändan trivdes han bra och tyckte det var mysigt.

Karin Hult undrar om Fredrik kan se till att F6 har en test i Finform igen.
Fredrik svarar att man brukar be om texter från både F6 och Djungelpatrullen, och att det är ett svårt beslut att ta när man ger upp på att någonsin få texter.

\begin{beslut}
  \item välja Fredrik Skoglund till sexreterare.
\end{beslut}

\subsubsection{Val av kassör}
Det finns 1 sökande:
\begin{itemize}
    \item Filip Bergqvist.
\end{itemize}

\paragraph{Filip Bergqvist} är 19, f20 och har körkort till råga på allt.
Han tyckte att det var kul på F6 arr under Mottagningen.
Han vill dessutom bidra till sektionen och framför allt Nollan 2020 som har en hel del att ta igen.

Kassör i F6 Alexander Samuelsson säger att man som kassör får ta ställning till investeringsförslag som är mer eller mindre bra.
Han undrar därför hur Filip känner kring att köpa in en sockervaddsmaskin?
Filip tycker att man kan ha råd med det om man sparar sin veckopeng, men att det inte borde läggas mer än så på det.

Emelie Björkman kommenterar att de två föregående talarna inte kan hålla isär sektionsmöten och styrelsemöten, och undrar om Filip kan ta över om de helt förlorar det under året.
Hon undrar även vad han vill arrangera när arrangemangen inte behöver vara på distans längre?
Filip svarar att han vill arrangera sittningar, och mer specifikt en bastusittning.

Alexander undrar hur mycket Filip vill investera i aktien GME?
Filip säger att han om han får chansen behöver göra ett ställningstagande, men att han om det är möjligt vill investera 100\% av pengarna i aktien.

Tobias Wallström undrar vad Filip vill lägga budgeten under Cortègemästerskaps-sittningen på?
Filip svarar att det beror på vilka idéer de har, men att det skulle vara kull att göra om Gasquen till en sandbunker.

\begin{beslut}
  \item välja Filip Bergqvist till kassör i F6.
\end{beslut}

\subsubsection{Val av 0--6 ledamöter}
\hyperlink{bilagor/nom/f6.pdf.1}{Valberedningens gruppnominering} till F6 väljs in automatiskt i enlighet med reglementet då de nominerade förtroendeposterna valts in.
Valberedningen läser upp gruppens nominering.

\subsection{Djungelpatrullen}

\subsubsection{Val av överste}
Det finns 1 sökande:
\begin{itemize}
    \item Lowe Fareld Krågen.
\end{itemize}

\paragraph{Lowe Fareld Krågen} är 20, tm19 och har körkort.
Han söker eftersom han har haft kul under senaste året och känner att han har mer att ge.

Emelie Björkman undrar vad Lowe vill bidra med till Sektionsstyrelsen?
Lowe har lite tekniska problem, men får till slut fram att han tänker vara på mötena som representant för DP och att han tänker framföra sina patrullmäns talan.
Han vill se till att allt som sägs och blir av ska vara bra för alla.

Frida Krohn säger att Lowe hörs dåligt, och undrar om det beror på att han sitter på dass.
Lowe svarar att han sitter i en dusch.

Ledamot i Djungelpatrullen Elin Eklund säger att Isabella under året varit exemplarisk, och undrar om Lowe kan fylla hennes skor.
Lowe svarar att han ska köpa större skor.

\begin{beslut}
  \item välja Lowe Fareld Krågen till överste.
\end{beslut}

\subsubsection{Val av rustmästare}
Det finns 2 sökande:
\begin{itemize}
    \item Gustav Persson
    \item Emrik Östling.
\end{itemize}

\paragraph{Gustav Persson} är 21, f2019 och har körkort.
Han söker eftersom det gångna året inte var hur han ville att det skulle vara, och han därför vill sitta om.

Rustmästare 18/19 Thomas Johansen undrar hur mycket Gustav kan bänka.
Gustav svarar att han kanske kan bänka några kilo, men att han inte vet eftersom han inte gymmar.

\paragraph{Emrik Östling} är 20, f20 och har körkort.
Han söker eftersom han tycker att Focus behöver bli bättre.
Han tycker att Focus har fint läge och stor potential, och därför bara behöver fräschas upp lite.
Han vill även se till att FnollK kan vara i Focus under hela Mottagningen

Axel Flordal undrar om Emrik kan använda sin koppling till FnollK på något sätt.
Emrik svarar att han kan använda LoB--tejp.

Erik Brusewitz tycker att det är perfekt att Emrik kommer kunna prata med sig själv kring lokalbokningar.
Han tycker dock att både Lowe och Emrik är söta, och undrar därför om Emrik tror att Djungelpatrullen kommer bli för söt.
Emrik viftar lite med händerna och säger att det inte är några problem.

Ledamot i Djungelpatrullen Ludvig Nordqvist undrar om Emrik kan räkna upp 10 powertools.
Emrik räknar upp LoB--tejp ett antal gånger och ger sedan upp.

Joseph Löfving tycker att det är konstigt att Emrik säger \enquote{i Focus}.
Emrik svarar att han menade att \enquote{det var i fokus på Focus}.

Thomas Johansen undrar hur mycket Emrik kan bänka.
Emrik svarar 3 rullar LoB--tejp.

\begin{beslut}
  \item välja Gustav Persson till rustmästare.
\end{beslut}

\subsubsection{Val av skattmästare}
Det finns 2 sökande:
\begin{itemize}
    \item Gabriel Berggren
    \item Erik Brusewitz.
\end{itemize}

\paragraph{Gabriel Berggren} är 20, f20 och har körkort.
Han säger att han söker för att han kommit nära Djungelpatrullen och de andra nominerade, och för att han länge varit intresserad av ekonomi.

Arvid Andersson säger att det kan hända dumma saker, som till exempel att en patet reser till Karibien och ber Djungelpatrullen betala.
Han undrar hur Gabriel känner kring det.
Gabriel svarar att det kommer komma utmaningar, men att han kommer ta hand om det \enquote{head on}.

\paragraph{Erik Brusewitz} är 23, f18 och har inte körkort.
Han söker eftersom han tycker att Djungelpatrullen behöver en medlem som sitter i FnollK eftersom Sektionsstyrelsen, Focumateriet och F6 redan har medlemmar som suttit i FnollK.
Dessutom blev han nominerad av någon.
Han påstår att \enquote{Nollan visst inte ska sitta i DP}, och menar då att en Nollan då inte kan vara kassör.

Emma Ödman säger att Erik brukar ta på sig för mycket och är slarvig.
Hon undrar hur det ska fungera om han även sitter i Djungelpatrullen.
Erik påstår att det är lättare om han sitter i två föreningar, eftersom han då kan ha en gemensam pott för båda föreningarna.
Det skulle då gå att köpa FnollK-grejer med Djungelpatrulls-pengar och vice versa.

Joseph Löfving undrar om Erik kan ta hand om pengar två gånger om då Erik påstås ha sagt \enquote{Fuck vem har skrivit det här\dots Det är ju jag!}
Erik tycker att det i det fallet saknades en signatur på det han skrivit, vilket det inte gör på verifikat.

Erik Bivrin påpekar att Erik pratade om söthet tidigare, men att jäst dör vid hög sötma, vilket gör att alkohol inte produceras.
Han undrar därför hur Erik tror att han kommer passa in i kulturen.
Erik säger något om att det är bra att man inte väljer in sig själv, och han påstår att han nu rättar sina tidigare fel.

Arvid Andersson undrar hur det känns att inte komma med.
Erik tycker det är tråkigt.

Axel Flordal påminner om att man som kassör interagerar med andra kassörer på sektionen.
Erik skulle då få väldigt många interaktioner, vilket Axel tycker verkar svårt.
Erik svarar att han kan sova mindre och skaffa en till mobiltelefon för att lösa detta.
Då kan han ha meddelandena från Djungelpatrullen och FnollK på separata enheter.

Ludvig Nordqvist undrar hur många tolvor Erik kan klämma utan att bli klämd.
Erik svarar noll.

David Winroth frågar om det var bättre förr eller om det är bäst just nu, och undrar vilket overallsben man ska sy sitt namn på.
Erik svarar att det alltid är bäst.
Han anser även att det är givet att man ska sy på namnet på sidan.
Då kan man rotera 90 grader, och på så sätt representera båda föreningarna.

\begin{beslut}
  \item välja Gabriel Berggren till skattmästare.
\end{beslut}

\subsubsection{Val av 0--7 adjutanter}
\hyperlink{bilagor/nom/dp.pdf.1}{Valberedningens gruppnominering} till Djungelpatrullen väljs in automatiskt i enlighet med reglementet då de nominerade förtroendeposterna valts in.
Valberedningen läser upp gruppens nominering.

\subsection{Studienämnden (SNF)}

\subsubsection{Val av ordförande}
Det finns 1 sökande:
\begin{itemize}
    \item Albert Vesterlund.
\end{itemize}

\paragraph{Albert Vesterlund} är 20, tm19 och har körkort.
Han har gått på alla cocktailpartyn, och har varit kursutvärderare fler gånger än vad han gått läsperioder på Chalmers.
Han vill att utbildningen ska hålla kvalité, och att alla ha sektionsmedlemmar ska ha det kul.

Emelie Björkman säger att studiebevakningen av TM ibland fått kritik, och undrar om Albert har några tankar om det.
Albert svarar att det är viktigt att se till att båda utbildningarna har hög kvalité.
Han kommer bry sig om båda utbildningarna mycket, och vill inte ge fördel åt den ena eller andra.

\begin{beslut}
  \item välja Albert Vesterlund till ordförande i studienämnden.
\end{beslut}

\subsubsection{Val av vice ordförande}
Det finns inga sökande.

\subsubsection{Val av kassör}
Det finns 1 sökande:
\begin{itemize}
    \item Carin Lundqvist.
\end{itemize}

\paragraph{Carin Lundqvist} är 21, f18 och har körkort.
Hon är gammal spexare, och vill testa att göra något annat på sektionen.
Hon tycker att studier är bra, och vill jobba med psykisk ohälsa.
Hon söker kassör eftersom det är bra att ha i en studienämnd, även om veckobladerist skulle varit mer hennes stil.

\begin{beslut}
  \item välja Carin Lundqvist till kassör i studienämnden.
\end{beslut}

\subsubsection{Val av sekreterare}
Det finns 1 sökande:
\begin{itemize}
    \item Valter Schütz.
\end{itemize}

\paragraph{Valter Schütz} är 21, f19 och har körkort.
Han söker lite spontant eftersom han vill vara sektionsaktiv och det här är det sista invalet för kvällen.
Han tycker att det SNF gör är viktigt, och att sekreterare låter rätt bekvämt.

Carin Lundqvist undrar om Valter har någon erfarenhet av att vara sekreterare.
Valter svarar att han inte har det, men att någon gång måste vara första gången.

Albert Vesterlund undrar om Valter vill göra något annat i studienämnden än att anteckna på möten.
Valter svarar att han vill jobba för en bättre utbildning.

Årskursrepresentant åk. 1 Samuel Martinsson påstår att sekreteraren är citat-ansvarig, och därmed behöver vara snabb med att skriva ner citat.
Han undrar om Valter är det, och undrar om Valter kan dela citaten med pateter.
Valter svara att han skriver lagom snabbt, och att han kan skriva ner citat om det behövs.

\begin{beslut}
  \item välja Valter Schütz till sekreterare i studienämnden.
\end{beslut}

\subsubsection{Val av kandidatansvarig}
Det finns 1 sökande:
\begin{itemize}
    \item Ella Larsson.
\end{itemize}

\paragraph{Ella Larsson} är 22, f18 och har körkort.
Hon söker eftersom hon tycker om sektionen, och tycker att det är hög tid för henne att dra sitt strå till stacken.
Hon söker kandidatansvarig för att vara del av det kontinuerliga förbättringsarbetet, och för att se till att utbildningen håller hög kvalité.
Hon tror även att det kan bidra till hennes personliga utveckling.

Albert Vesterlund undrar om Ella har någon taktik för att få tag på kursutvärderare.
Ella svarar att hon tänker att man frågar på Facebook, och om det inte fungerar kan man hetsa folk privat.
Hon tycker att det är viktigt att ha kursutvärderare, eftersom de gör utbildningen lite bättre varje år.

\begin{beslut}
  \item välja Ella Larsson till kandidatansvarig.
\end{beslut}

\subsubsection{Val av masteransvarig}
Det finns inga sökande.

\subsubsection{Val av veckobladerist}
Det finns inga sökande.

\subsubsection{Val av matansvarig}
Det finns inga sökande.


\section{Övriga frågor}
Inga övriga frågor föreligger.


\section{Dumvästutdelning}
Nomineringarna är som följer:
\begin{itemize}
  \item \textbf{David Winroth} nomineras av Joseph Löfving.\newline
  När Game Boy skulle köpa in brädspel för pengar från programansvarig Jonathan Weidow köptes även ett stort fult plåtskåp in för att förvara spelen i.
  När resten av Sektionsstyrelsen uttryckte sitt missnöje kring situationen reagerade sektionskassör David Winroth med att vara upprörd tillbaka; David tyckte nämligen att Sektionsstyrelsen beslutade om inköpet tillsammans.
  Eftersom ingen förutom David kan minnas att detta någonsin skulle skett kan det därför anses vara Davids fel att det nu står ett stort, fult plåtskåp på Focus, där det redan råder platsbrist, för att David vill hålla sektionens brädspel inlåsta.

  \item \textbf{Simon Franklin} nomineras av Emma Ödman.\newline
  Som nyinvald phadderchef skulle Simon trycka sin overall.
  Efter att tillverkat en vinylutskrift av FnollK:s märke i 4 frustrerande timmar tog han fram tryckpressen från kassörsrummet.
  Simon la sin overall på plats och började trycka.
  Efter tiden som föreslagits i de medföljande instruktionerna släppte han på tryckpressen och märkte han att vinylen inte hade fastnat på tyget.
  Flera försök senare upptäckte han att han aldrig tryckt ner tryckpressen ordentligt.
  Plasten var vid det här laget insmält in i vinylen, men Simon valde att fortsätta trycka.
  Efter många frustrerande försök och några irriterade samtal bestämde Simon sig för att slita bort plasten.
  Resultatet blev en ful rygglogga.
  Till råga allt insåg han även att hans phaddergrupps-märke var synligt under pi-märket då han tryckt overallen med uppdragen dragkedja.

  \item \textbf{Joseph Löfving} nomineras av Simon Franklin.\newline
  Joseph hade som sekreterare ansvaret för att skriva tydliga instruktioner till sektionens tryckutrustning.
  Dessa instruktioner var bristfälliga på 3 sätt:
  \begin{itemize}
    \item Det stod att plagget skulle ligga platt, men det specificerades inte att dragkedjan behövde vara öppen.
    \item Det var inte tydligt vilken temperatur tryckpressen skulle vara inställd på, och olika källor antydde olika temperaturer.
    \item Det var inte tydligt att tryckpressen kunde tryckas tillräckligt långt för att låsa sig.
  \end{itemize}
  På grund av Josephs dåliga instruktioner blev ryggloggan på FnollK:s nyinvalda phadderchefs overall ful.

  \item \textbf{Simon Franklin} nomineras av Joseph Löfving.\newline
  Simon hade misslyckats med att trycka ryggloggan på sin overall.
  Istället för att stå för sitt misstag valde han att skylla på Sektionsstyrelsens sekreterare Joseph Löfving.
  Hans kritik av instruktionerna som Joseph skrivit kan anses felaktiga, eftersom:
  \begin{itemize}
    \item Dragkedjan på ett plagg måste vara öppen för att plagget ska kunna ligga platt.
    \item Det stod i instruktionerna att rätt temperatur var förinställd, vilket stämde.
    \item Ordet \enquote{tryckpress} borde göra det tydligt att plagg i maskinen ska både tryckas och pressas.
  \end{itemize}
  Det kan även tilläggas att instruktionerna är tydliga nog för att ett tiotal föreningar på sektionen ska ha kunnat trycka sina plagg utan problem.
  Det kan även tilläggas att Simon lämnade kassörsrummet upplåst efter att ha misslyckats med att följa de tydliga instruktionerna.

  \item \textbf{Niklas Johansson} nomineras av Alexandru Golic.\newline
  En fredagskväll ville Niklas äta chips, men de enda chipsen han hade hemma var mjuka.
  Han bestämde sig därför för att försöka grilla chipsen i sin ugn.
  Tio minuter senare känner han rökdoft i sin lägenhet, och går för att undersöka.
  Eftersom Niklas har grundläggande kunskaper inom om fysik tänkte han att det inte kunde brinna om det var kallt, och öppnade därför locket.
  Tyvärr vägs kylan av att öppna ett lock mer än väl upp av syretillförseln som detta orsakar, och Niklas behövde därför släcka den nyuppkomna branden i sin ugn genom att slänga på locket igen.
  Det kan även tilläggas att Niklas brandlarm inte reagerade på någon del av händelseförloppet.

  \item \textbf{David Winroth} nomineras av Mats Richardson.\newline
  David tyckte att rätt ställe att förvara sektionens spel var i ett låst skåp.
  Det gör det svårt för sektionsmedlemmarna att spela spel.

\end{itemize}

\begin{beslut}
  \item Simon Franklin var dummast.
\end{beslut}
Efter noggrann jämförelse med föregående dumvästinnehavare Hugo Rådegård går mötet till beslut.
\begin{beslut}
  \item ge Simon Franklin Dumvästen.
\end{beslut}

\section{Mötets avslutande}
Mötet avslutas 23.19 av Talman Ruben Seyer.

\clearpage
\section*{Signaturer}
\label{sec:sig}
\addcontentsline{toc}{section}{\nameref{sec:sig}}
En signatur på den här sidan avser hela sektionsmötesprotokollet \themote. Det sker elektroniskt via digitalt ID. Kontakta talmanspresidiets sekreterare på \href{mailto:talman.sekr@ftek.se}{\texttt{talman.sekr@ftek.se}} vid kontroll eller övriga frågor. 

\vspace{4cm}
\begin{center}
  \makebox[12cm][c]{
    \parbox{12cm}{
      \signatur{Ruben Seyer}{Talman} \hfill
      \signatur{Felix Augustsson}{Sekreterare}
      \newline
      \signatur{Alexandru Golic}{Justerare} \hfill
      \signatur{Albert Vesterlund}{Justerare}
      \vspace{1.5em}
    }
  }
\end{center}

\clearpage
\begin{bilagor}
  \bilaga{Beslut att fastställa}{bf.pdf}
  
  \bilaga{Verksamhetsberättelse FIF 2020}{fifvb.pdf}
    
  \bilaga{Preliminär verksamhetsplan Sektionsstyrelsen 21/22}{pvp.pdf}
  \bilaga{Preliminär budget 21/22}{budget.pdf}
  
  \bilaga{Nominering Kärnstyret 21/22}{nom/styret.pdf}
  \bilaga{Nominering Focumateriet 21/22}{nom/foc.pdf}
  \bilaga{Nominering F6 21/22}{nom/f6.pdf}
  \bilaga{Nominering Djungelpatrullen 21/22}{nom/dp.pdf}
  \bilaga{Nominering Studienämnden (SNF) 21/22}{nom/snf.pdf}
   
  \bilaga{Proposition om Uppstädning av styrdokument}{prop/a.pdf}
  
  \bilaga{Motion om Förflyttandet av invalet av memeposter samt sektionsnörd för underlättandet av sektionsmötet LP3}{motion/a.pdf}
  \bilaga{Motionssvar till Förflyttandet av invalet av memeposter samt sektionsnörd för underlättandet av sektionsmötet LP3}{motion/asvar.pdf}
  \bilaga{Motion om Sluta kräva inval innan motioner och propositioner}{motion/b.pdf}
  \bilaga{Motionssvar till Sluta kräva inval innan motioner och propositioner}{motion/bsvar.pdf}
  \bilaga{Motion om Ändra Fabiolas stadga}{motion/c.pdf}
  \bilaga{Motionssvar till Ändra Fabiolas stadga}{motion/csvar.pdf}
  \end{bilagor}

\end{document}